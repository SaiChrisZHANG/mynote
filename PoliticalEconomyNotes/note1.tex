\documentclass[twoside]{article}
\setlength{\oddsidemargin}{0 in}
\setlength{\evensidemargin}{0 in}
\setlength{\topmargin}{-0.6 in}
\setlength{\textwidth}{6.5 in}
\setlength{\textheight}{8.5 in}
\setlength{\headsep}{0.75 in}
\setlength{\parindent}{0 in}
\setlength{\parskip}{0.1 in}

\usepackage{url}
\usepackage{titlesec}
\setcounter{secnumdepth}{3}
\usepackage{palatino}
\usepackage{marginnote}
\usepackage{multirow}
\usepackage{multicol,array}
\usepackage{xltabular}
\usepackage{easybmat,bigdelim,arydshln}
\usepackage[authoryear,round]{natbib}
\usepackage{amssymb,amsmath,amsthm,amsfonts}
\usepackage{mathtools}
\usepackage{caption}
\usepackage{hyperref}
\usepackage{tcolorbox}
\tcbuselibrary{skins, breakable, theorems}
\usepackage{newpxtext,newpxmath}
\usepackage{longtable}
\usepackage{enumitem}
\makeatletter

\let\bar\overline

\setlist[itemize]{topsep=0pt,leftmargin=10pt,itemsep=-0.2em}
\usepackage{xcolor}
\usepackage{tikz}
\usepackage{pgfplots}
\pgfplotsset{compat = newest}
\usetikzlibrary{patterns,decorations.pathreplacing,decorations.markings}
\usepgfplotslibrary{fillbetween}

\hypersetup{
    colorlinks,
    citecolor=red,
    filecolor=black,
    linkcolor=violet,
    urlcolor=blue
}

\makeatletter
\renewcommand{\thefigure}{\thesection.\arabic{figure}}
\newtheoremstyle{indented}
  {3pt}% space before
  {3pt}% space after
  {\addtolength{\@totalleftmargin}{3.5em}
   \addtolength{\linewidth}{-3.5em}
   \parshape 1 3.5em \linewidth}% body font
  {}% indent
  {\bfseries}% header font
  {.}% punctuation
  {.5em}% after theorem header
  {}% header specification (empty for default)
\makeatother

\theoremstyle{definition}
\newtheorem{defin}{Definition}[section] % Creates a new counter, number within section
\newtheorem{prt}[defin]{Remark} 
\newtheorem{prts}[defin]{Remarks} % Again share defin's counter
\newtheorem{exmp}[defin]{Example} % etc.
\newtheorem{exmps}[defin]{Examples}
\newtheorem*{note}{Note}
\tcbuselibrary{theorems}

% use counter*=defin to make each tcbtheorem share defin's counter

\newtcbtheorem[use counter*=defin, number within=section]{definition}{Key takeaways}{enhanced, breakable,
    colback = white, colframe = red!55!black, colbacktitle = red!55!black, attach boxed title to top left = {yshift = -2.5mm, xshift = 3mm}, boxed title style = {sharp corners},fonttitle=\bfseries}{takeaway}

\newtcbtheorem[use counter*=defin, number within=section]{theorem}{Theorem}{enhanced, breakable,
    colback = white, colframe = blue!45!black, colbacktitle = blue!45!black, attach boxed title to top left = {yshift = -2.5mm, xshift = 3mm}, boxed title style = {sharp corners},fonttitle=\bfseries}{thm}
    
\newtcbtheorem[use counter*=defin, number within=section]{proposition}{Proposition}{enhanced, breakable,
    colback = white, colframe = teal, colbacktitle = teal, attach boxed title to top left = {yshift = -2.5mm, xshift = 3mm}, boxed title style = {sharp corners},fonttitle=\bfseries}{prop}

\newtcolorbox{example}[1]{enhanced, breakable, colback = white, colframe = orange!85!black, colbacktitle = orange!85!black, attach boxed title to top left = {yshift = -2.5mm, xshift = 3mm}, boxed title style = {sharp corners},fonttitle=\bfseries, title={Example: #1}}

\newtcbox{\myhl}[1][white]
  {on line, arc = 0pt, outer arc = 0pt,
    colback = #1!20!white, colframe = #1!50!black,
    boxsep = 0pt, left = 1pt, right = 1pt, top = 1pt, bottom = 1pt, boxrule = 0pt, bottomrule =0pt, toprule =0pt}
    
\newtcbox{\myhlrule}[1][white]
  {on line, arc = 0pt, outer arc = 0pt,
    colback = #1!20!white, colframe = #1!50!black,
    boxsep = 0pt, left = 1pt, right = 1pt, top = 1pt, bottom = 1pt, boxrule = 0pt, bottomrule =0.5pt, toprule =0.5pt}
%
% The following commands set up the lecnum (lecture number)
% counter and make various numbering schemes work relative
% to the lecture number.
%
\newcounter{lecnum}
\renewcommand{\thepage}{\thelecnum-\arabic{page}}
\renewcommand{\thesection}{\thelecnum.\arabic{section}}
\renewcommand{\theequation}{\thelecnum.\arabic{equation}}
\renewcommand{\thefigure}{\thelecnum.\arabic{figure}}
\renewcommand{\thetable}{\thelecnum.\arabic{table}}

\newcommand{\sidenotes}[1]{\marginnote{\raggedright\scriptsize#1}}
%
% The following macro is used to generate the header.
%
\newcommand{\lecture}[4]{
   \pagestyle{myheadings}
   \thispagestyle{plain}
   \newpage
   \setcounter{lecnum}{#1}
   \setcounter{page}{1}
   \noindent
   \begin{center}
   \framebox{
      \vbox{\vspace{2mm}
    \hbox to 6.28in { {\bf Political Economy
	\hfill } }
       \vspace{4mm}
       \hbox to 6.28in { {\Large \hfill Topic #1: #2  \hfill} }
       \vspace{2mm}
       \hbox to 6.28in { {\it \hfill by: #4} }
      \vspace{2mm}}
   }
   \end{center}
   \markboth{Week #1: #2}{Week #1: #2}

   {\bf Key takeaways}: {\begin{itemize}
       \item 
   \end{itemize}}

   {\bf Disclaimer}: {\it These notes are written by Sai Zhang (\href{emailto:saizhang.econ@gmail.com}{email me}), based on . Please do \textbf{NOT} distribute them online without permission.}
   \vspace*{4mm}
}
%

\tikzset{-stealth-/.style={decoration={
  markings,
  mark=at position #1 with {\arrow{stealth}}},postaction={decorate}}}

  \tikzset{tangent/.style={
    decoration={
        markings,% switch on markings
        mark=
            at position #1
            with
            {
                \coordinate (tangent point-\pgfkeysvalueof{/pgf/decoration/mark info/sequence number}) at (0pt,0pt);
                \coordinate (tangent unit vector-\pgfkeysvalueof{/pgf/decoration/mark info/sequence number}) at (1,0pt);
                \coordinate (tangent orthogonal unit vector-\pgfkeysvalueof{/pgf/decoration/mark info/sequence number}) at (0pt,1);
            }
    },
    postaction=decorate
},
use tangent/.style={
    shift=(tangent point-#1),
    x=(tangent unit vector-#1),
    y=(tangent orthogonal unit vector-#1)
},
use tangent/.default=1}

\begin{document}
\lecture{1}{Voter Turnout}{}{Sai Zhang}
%\footnotetext{These notes are partially based on those of Nigel Mansell.}

\section{Empirical Studies}

\subsection{Get-out-the-vote campaigns}
\subsubsection*{\citet*{gerber2000effects}}
\paragraph*{Main hypothesis} Personal canvassing mobilizes voters more effectively, decline in voter turn-out is related to decline in personal means of campaign.

\paragraph*{Literature}
Some related works are 
\begin{itemize}
    \item collective action and prosocial behavior (blood donations, recycling, good deeds): Christensen et al. (1998), Wang and Katzev (1990), Spaccarelli et al. (1989), Reams and Ray (1993)
    \item voter turn-out and contact with political organizations/candidates: 
    \begin{itemize}
        \item[-] Kramer (1970), Rosenstone and Hansen (1993). \textbf{\textit{Endogeneity issue}}: political contact is not exogenous: it could that those most likely to vote are also most likely to receive contact.
        \item[-] Adams and Smith (1980), Miller et al. (1981). \textbf{\textit{Small sample}}: results are not statistically reliable
    \end{itemize}
\end{itemize}

\paragraph*{Empirical details} This paper designed an experiment as in table \ref{tab:gerber2000design}, and
\begin{itemize}
    \item $N=29380$ (within 22077 households): randomize at HH level, analyze at individual level, NO clustering
    \item the baseline control group (no mail, no phone call, no in-person contact) is large ($N=10800$), due to budget limit
    \item specification: intent-to-treat, use the treatment assignment as an instrument
\end{itemize}

\begin{table}[ht]
\caption{Experiment Design of \citet{gerber2000effects}}\label{tab:gerber2000design}
\centering
    \begin{tabular}{ccccccc}
             & & \multicolumn{4}{c}{No. direct mailings sent} & \\ \cline{3-6}
            & & 0 & 1 & 2 & 3 & Total\\ 
        \hline
        \multirow{2}{*}{No phone call} & \textbf{without} in-person &  10800 & 2406 & 2588 & 2375 & 18169 \\
         & \textbf{with} in-person & 2686 & 519 & 625 & 627 & 4457\\
         \multirow{2}{*}{Phone call} & \textbf{without} in-person &  958 & 1451 & 1486 & 1522 & 5417 \\
         & \textbf{with} in-person & 217 & 385 & 352 & 383 & 1337\\
        \hline
        \multicolumn{2}{c}{Total} & 14661 & 4761 & 5051 & 4907 & 29380\\
    \end{tabular}
\end{table}

\paragraph*{Results} personal canvassing is very effective, while telephone and mail canvassing is limited. Face-to-face political activity is suggested to be very important. Declien in voter turnout might be attributed to the decrease of in-person campaigns. A competing hypothesis proposed by Rosenstone and Hansen (1993) where decline in voter turn-out is related to decline in the volume of mobilization was ruled out since ANES data shows no trend of decline in total mobilization.

\paragraph*{Questions and comments} Some of the questions I have for this study are 
\begin{itemize}
    \item Telephone and mail treatment are designed to be correlated, why? To show that even the two combined won't work as good as personal canvassing? Perhaps it's just what happened, unintentionally, a situation.
    \item Selection bias: The answering rate is only 28\% of the in-person treated group, only 32.1\% of the phone-call treated group. The results show that the bias is not big.
    \item Potential spillover effect: It's an ITT analysis, would be interesting to see the effects on the neighbors of the in-person contacted HHs.
    \item This is perhaps related to the rising of more polarized, social-media-star type of politicians? I guess it's related to the mechanism behind the effectiveness of in-person means, is it because the people would react more actively to things that they are feel? Or a hate towards the campaign means that they feel mistreated or dehumanized (email/messages)?
\end{itemize}
And some general comments:
\begin{itemize}
    \item \textbf{\underline{I like}}: bigger experiment, cleanly written, good explanation on ITT and instruments, well designed experiment
    \item \textbf{\underline{Not so sure}}: empirical strategy is a bit basic (not necessarily a bad thing), no understanding on the mechanism, generality is limited (bigger scale/more important elections would probably need other forms of nudges), no welfare analysis.
\end{itemize}

\subsubsection*{\citet*{gerber2008social}}
\paragraph*{Main hypothesis} Social pressure can serve as an important inducement to political participation. There are two competing channels:
\begin{itemize}
    \item \textbf{\underline{intrinsic}}: satisfaction from behaving according to a norm
    \item \textbf{\underline{extrinsic}}: incentive to comply to a norm
\end{itemize}

\paragraph*{Literature} In social psychology
\begin{itemize}
    \item complying motive: Cialdini and Goldstein (2004), Gerber and Rugers (2007)
    \item reactance (rejecting heavy-handed demand): Ringold et al. (1994)
\end{itemize}
And the effect of exposing voting records to neighbors: Cardy (2005), Ramirez (2005), Michelson (2005)

\paragraph*{Empirical details} The experiment was prior to Michigan August 2006 primary election, the grouping follows 
\begin{itemize}
    \item \textbf{\underline{control}}: no extra information
    \item \textbf{\underline{civic duty}}: adding "DO YOUR CIVIC DUTY - VOTE!"
    \item \textbf{\underline{Hawthorne}}: adding "YOU ARE BEING STUDIED"
    \item \textbf{\underline{self}}: 
    \item \textbf{\underline{neighbors}}:
\end{itemize}


\subsection*{\citet*{washington2006black}}

\section{Related Theories}



\newpage
\bibliographystyle{plainnat}
\bibliography{ref.bib}

\end{document}