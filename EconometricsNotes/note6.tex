\documentclass[twoside]{article}
\setlength{\oddsidemargin}{0 in}
\setlength{\evensidemargin}{0 in}
\setlength{\topmargin}{-0.6 in}
\setlength{\textwidth}{6.5 in}
\setlength{\textheight}{8.5 in}
\setlength{\headsep}{0.75 in}
\setlength{\parindent}{0 in}
\setlength{\parskip}{0.1 in}

\usepackage{url}
\usepackage{titlesec}
\setcounter{secnumdepth}{3}
\usepackage{palatino}
\usepackage{marginnote}
\usepackage{multirow}
\usepackage{easybmat,bigdelim,arydshln}
\usepackage[authoryear,round]{natbib}
\usepackage{amssymb,amsmath,amsthm,amsfonts}
\usepackage{mathtools}
%\usepackage{nicematrix}
\usepackage{arydshln}
\usepackage{caption}
\usepackage{hyperref}
\usepackage{tcolorbox}
\tcbuselibrary{skins, breakable, theorems}
\usepackage{newpxtext,newpxmath}
\usepackage{longtable}
\usepackage{enumitem}
\makeatletter

\let\bar\overline

\setlist[itemize]{topsep=0pt,leftmargin=10pt,itemsep=-0.2em}
\usepackage{xcolor}
\usepackage{tikz}
\usepackage{pgfplots}
\pgfplotsset{compat = newest}
\usetikzlibrary{patterns,decorations.pathreplacing,decorations.markings,fit,shapes.geometric,angles,quotes,arrows}
\usepgfplotslibrary{fillbetween}

\usepackage{ifthen}
\usepackage{tikz-3dplot}

\pgfdeclarelayer{ft}
\pgfdeclarelayer{bg}
\pgfsetlayers{bg,main,ft}

\hypersetup{
    colorlinks,
    citecolor=red,
    filecolor=black,
    linkcolor=violet,
    urlcolor=blue
}

\definecolor{myblue}{cmyk}{1,.72,0,.38}
\definecolor{mypurple}{cmyk}{.57,1,0,.58}
\definecolor{myred}{cmyk}{0,.88,.88,.58}
\definecolor{mygreen}{cmyk}{1,0,.69,.66}
\definecolor{myorange}{cmyk}{0,.58,100,.20}
\definecolor{glaucous}{rgb}{0.38, 0.51, 0.71}

\makeatletter
\renewcommand{\thefigure}{\thesection.\arabic{figure}}
\newtheoremstyle{indented}
  {3pt}% space before
  {3pt}% space after
  {\addtolength{\@totalleftmargin}{3.5em}
   \addtolength{\linewidth}{-3.5em}
   \parshape 1 3.5em \linewidth}% body font
  {}% indent
  {\bfseries}% header font
  {.}% punctuation
  {.5em}% after theorem header
  {}% header specification (empty for default)
\makeatother

\newcommand{\ind}{\perp\!\!\!\perp}

\theoremstyle{definition}
\newtheorem{defin}{Definition}[section] % Creates a new counter, number within section
\newtheorem{prt}[defin]{Remark} 
\newtheorem{prts}[defin]{Remarks} % Again share defin's counter
\newtheorem{exmp}[defin]{Example} % etc.
\newtheorem{exmps}[defin]{Examples}
\newtheorem*{note}{Note}
\tcbuselibrary{theorems}

% use counter*=defin to make each tcbtheorem share defin's counter

\newtcbtheorem[use counter*=defin, number within=section]{definition}{Definition}{enhanced, breakable,
    colback = white, colframe = red!55!black, colbacktitle = red!55!black, attach boxed title to top left = {yshift = -2.5mm, xshift = 3mm}, boxed title style = {sharp corners},fonttitle=\bfseries}{def}

\newtcbtheorem[use counter*=defin, number within=section]{theorem}{Theorem}{enhanced, breakable,
    colback = white, colframe = blue!45!black, colbacktitle = blue!45!black, attach boxed title to top left = {yshift = -2.5mm, xshift = 3mm}, boxed title style = {sharp corners},fonttitle=\bfseries}{thm}
    
\newtcbtheorem[use counter*=defin, number within=section]{proposition}{Proposition}{enhanced, breakable,
    colback = white, colframe = teal, colbacktitle = teal, attach boxed title to top left = {yshift = -2.5mm, xshift = 3mm}, boxed title style = {sharp corners},fonttitle=\bfseries}{prop}

\newtcbtheorem[use counter*=defin, number within=section]{lemma}{Lemma}{enhanced, breakable,
    colback = white, colframe = orange!80!black, colbacktitle = orange!80!black, attach boxed title to top left = {yshift = -2.5mm, xshift = 3mm}, boxed title style = {sharp corners},fonttitle=\bfseries}{lemma}

\newtcbtheorem[use counter*=defin, number within=section]{example}{Example}{enhanced, breakable,
    colback = white, colframe = yellow!60!black, colbacktitle = yellow!60!black, attach boxed title to top left = {yshift = -2.5mm, xshift = 3mm}, boxed title style = {sharp corners},fonttitle=\bfseries}{exmp}

\newtcbtheorem[use counter*=defin, number within=section]{assumption}{Assumption}{enhanced, breakable,
    colback = white, colframe = violet!60!white, colbacktitle = violet!60!white, attach boxed title to top left = {yshift = -2.5mm, xshift = 3mm}, boxed title style = {sharp corners},fonttitle=\bfseries}{assump}

\newtcbtheorem[use counter*=defin, number within=section]{algorithm}{Algorithm}{enhanced, breakable,
    colback = white, colframe = green!55!black, colbacktitle = green!55!black, attach boxed title to top left = {yshift = -2.5mm, xshift = 3mm}, boxed title style = {sharp corners},fonttitle=\bfseries}{algm}
%\newtcolorbox{example}[1]{enhanced, breakable, colback = white, colframe = orange!85!black, colbacktitle = orange!85!black, attach boxed title to top left = {yshift = -2.5mm, xshift = 3mm}, boxed title style = {sharp corners},fonttitle=\bfseries, title={Example: #1}}

\newtcbox{\myhl}[1][white]
  {on line, arc = 0pt, outer arc = 0pt,
    colback = #1!20!white, colframe = #1!50!black,
    boxsep = 0pt, left = 1pt, right = 1pt, top = 1pt, bottom = 1pt, boxrule = 0pt, bottomrule =0pt, toprule =0pt}
    
\newtcbox{\myhlrule}[1][white]
  {on line, arc = 0pt, outer arc = 0pt,
    colback = #1!20!white, colframe = #1!50!black,
    boxsep = 0pt, left = 1pt, right = 1pt, top = 1pt, bottom = 1pt, boxrule = 0pt, bottomrule =0.5pt, toprule =0.5pt}
%
% The following commands set up the lecnum (lecture number)
% counter and make various numbering schemes work relative
% to the lecture number.
%
\newcounter{lecnum}
\renewcommand{\thepage}{\thelecnum-\arabic{page}}
\renewcommand{\thesection}{\thelecnum.\arabic{section}}
\renewcommand{\theequation}{\thelecnum.\arabic{equation}}
\renewcommand{\thefigure}{\thelecnum.\arabic{figure}}
\renewcommand{\thetable}{\thelecnum.\arabic{table}}

\newcommand{\sidenotes}[1]{\marginnote{\raggedright\scriptsize#1}}
%
% The following macro is used to generate the header.
%
\newcommand{\lecture}[6]{
   \pagestyle{myheadings}
   \thispagestyle{plain}
   \newpage
   \setcounter{lecnum}{#1}
   \setcounter{page}{1}
   \noindent
   \begin{center}
   \framebox{
      \vbox{\vspace{2mm}
    \hbox to 6.28in { {\bf Econometrics
	\hfill \today} }
       \vspace{4mm}
       \hbox to 6.28in { {\Large \hfill Topic #1: #2  \hfill} }
       \vspace{2mm}
       \hbox to 6.28in { {\it #3 \hfill by #4} }
      \vspace{2mm}}
   }
   \end{center}
   \markboth{Week #1: #2}{Week #1: #2}

   {\bf Key points}: {#5}

   {\bf Disclaimer}: {\it #6}
   \vspace*{4mm}
}
%

\tikzset{-stealth-/.style={decoration={
  markings,
  mark=at position #1 with {\arrow{stealth}}},postaction={decorate}}}

  \tikzset{tangent/.style={
    decoration={
        markings,% switch on markings
        mark=
            at position #1
            with
            {
                \coordinate (tangent point-\pgfkeysvalueof{/pgf/decoration/mark info/sequence number}) at (0pt,0pt);
                \coordinate (tangent unit vector-\pgfkeysvalueof{/pgf/decoration/mark info/sequence number}) at (1,0pt);
                \coordinate (tangent orthogonal unit vector-\pgfkeysvalueof{/pgf/decoration/mark info/sequence number}) at (0pt,1);
            }
    },
    postaction=decorate
},
use tangent/.style={
    shift=(tangent point-#1),
    x=(tangent unit vector-#1),
    y=(tangent orthogonal unit vector-#1)
},
use tangent/.default=1}

\tikzstyle{terminator} = [rectangle, draw, thick, text centered, rounded corners, minimum height=2em]
\tikzstyle{process} = [rectangle, draw, thick, text centered, minimum height=2em]
\tikzstyle{decision} = [diamond, draw, thick, text centered, minimum width=3cm, minimum height=0.5cm]
\tikzstyle{data}=[trapezium, draw, thick, text centered, trapezium left angle=60, trapezium right angle=120, minimum height=2em]
\tikzstyle{arrow} = [thick,->,>=stealth]

\begin{document}
\lecture{6}{DID and TWFE}{}{Sai Zhang}{This note is on the causal panel data, building upon \citet{arkhangelsky2023causal}.}{This note is compiled by Sai Zhang.}
%\footnotetext{These notes are partially based on those of Nigel Mansell.}

\section{Panel Data Configurations}

\subsection{Data Types}
\subsubsection{Panel Data}
For observations on $N$ units, indexed by $i=1,\cdots,N$, over $T$ periods, indexed by $t=1,\cdots,T$, the outcome of interest is denoted by $Y_{it}$, the treatment $W_{it}$.
These observations may themselves consist of averages over more basic units:
\begin{align*}
    \mathbf{Y} &= \begin{pmatrix}
        Y_{11} & \cdots & Y_{1T}\\
        \vdots & \ddots & \vdots \\
        Y_{N1} & \cdots & Y_{NT}
    \end{pmatrix} &
    \mathbf{W} = \begin{pmatrix}
        W_{11} & \cdots & W_{1T}\\
        \vdots & \ddots & \vdots \\
        W_{N1} & \cdots & W_{NT}
    \end{pmatrix}
\end{align*}
we may also observe exogenous variables $X_{it}$ or $X_i$. Typically, we focus on a balanced panel where for all units $i=1,\cdots,N$ we observe outcomes for all $t=1,\cdots,T$.

\subsubsection{Grouped Repeated Cross-Section Data}
In a GRCS data, we have observations on $N$ units, each observed only once in period $T_i$ for unit $i$. Different units may be observed at diffrent points in time, $T_i$ typically takes on only a few values, with many units sharing the same value for $T_i$. The outcome $Y_i$ and treatment $W_i$ are indexed by the unit index $i$.
The set of units is \textbf{partitioned} into 2 or more groups, with the group that unit $i$ belongs to denoted by $G_i\in \mathcal{G}=\left\{1,2,\cdots,G\right\}$.

Define the average outcomes for each group-time-period pair:
\begin{equation*}
    \bar{Y}_{gt} \equiv \frac{\sum^N_{i=1} \mathbf{1}_{G_i=g,T_i=t}Y_i}{\sum^N_{i=1} \mathbf{1}_{G_i=g,T_i=t}}
\end{equation*}
for treatment 
\begin{equation*}
    \bar{W}_{gt} \equiv \frac{\sum^N_{i=1} \mathbf{1}_{G_i=g,T_i=t}W_i}{\sum^N_{i=1} \mathbf{1}_{G_i=g,T_i=t}}
\end{equation*}
then treat the $G\times T$ group averages $\bar{Y}_{gt}$ and $\bar{W}_{gt}$ as the unit of observation, then the grouped data is just a panel.
The major issue in practice is that the number of groups is very small comparing to proper panel data.

\subsubsection{Row and Column Exchangeable Data}
The data are doubly indexed by $i=1,\cdots,N$ and $j=1,\cdots,J$, with outcomes $Y_{ij}$. They are different from panel data in that there is \textbf{no time ordering} for the second index. Many methods developed for panel data are also applicable here.

\subsection{Shapes of Data Frames}
Panel data can also be loosely classified by the shape:
\begin{itemize}
    \item \myhl[myblue]{\textbf{Thin Frames} $(N\gg T)$}, where the number of cross-section units is large relative to the number of time periods:
    \begin{itemize}
        \item unit-specific parameters (individual FEs) \textbf{can not be estimated consistently} due to the short time series
        \item REs might be more suitable since they place a stocahstic structure on the individual components
    \end{itemize}
    \item \myhl[myblue]{\textbf{Fat Frames} $(N\ll T)$}, where the number of cross-section units is large relative to the number of time periods.
    \item \myhl[myblue]{\textbf{Square} $N\simeq T$}, where the number of units and time periods is comparable.
\end{itemize}

\subsection{Assignment Mechanisms}
\subsubsection{The General Case}
In the most general case, the treatment may vary both across units and over time, with units \textbf{switching} in and out of the treatment group:
\begin{equation*}
    \mathbf{W}^{\text{general}} = \begin{pmatrix}
        1&1&0&0&\cdots &1\\
        0&0&1&0&\cdots &0\\
        1&0&1&1&\cdots &0\\
        \vdots &\vdots & \vdots & \vdots & \ddots & \vdots \\
        1&0&1&0&\cdots &0
    \end{pmatrix}
\end{equation*}
This is more relevant for the RCED configurations, and for panel data of products and promotions as treatments.
The assumption on the absence/presence of \textbf{dynamic treatment} effects is very important.

\subsubsection{Single Treated Period}
One special case arises when a substantial number of units is treated, but these units are only treated \textbf{in the last period}
\begin{equation*}
    \mathbf{W}^{\text{last}} = \begin{pmatrix}
        0&0&0&0&\cdots &0\\
        0&0&0&0&\cdots &0\\
        0&0&0&0&\cdots &1\\
        \vdots &\vdots & \vdots & \vdots & \ddots & \vdots \\
        0&0&0&0&\cdots &1
    \end{pmatrix}
\end{equation*}
If $T$ is relatively small, this case is often analyzed as a cross-section problem, the lagged outcomes are used as exogenous covariates or pre-treatment variables to be adjusted.
Here, dynamic effects are not testable, nor do they matter since the shortness of the panel.
\begin{equation*}
    \mathbf{W}^{\text{last}} = \begin{pmatrix}
        0&0&0&0&\cdots &0\\
        0&0&0&0&\cdots &0\\
        0&0&0&0&\cdots &0\\
        \vdots &\vdots & \vdots & \vdots & \ddots & \vdots \\
        0&0&1&1&\cdots &0
    \end{pmatrix}
\end{equation*}
this setting is prominent in the original applications of the synthetic control literature, here $T$ is usually small.

\subsubsection{Single Treated Unit and Single Treated Period}
An extreme case is where only a single unit is treated, and it is only treated in a single period (typically the last). Normally, we focus on the effect for the single treated/time-period pair and construct prediction intervals.

\subsubsection{Block Assignment}
The case of block assignment is where a subset of units is treated every period after a common starting date:

\newpage
\bibliographystyle{plainnat}
\bibliography{ref.bib}

\end{document}