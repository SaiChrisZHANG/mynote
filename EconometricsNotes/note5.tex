\documentclass[twoside]{article}
\setlength{\oddsidemargin}{0 in}
\setlength{\evensidemargin}{0 in}
\setlength{\topmargin}{-0.6 in}
\setlength{\textwidth}{6.5 in}
\setlength{\textheight}{8.5 in}
\setlength{\headsep}{0.75 in}
\setlength{\parindent}{0 in}
\setlength{\parskip}{0.1 in}

\usepackage{url}
\usepackage{titlesec}
\setcounter{secnumdepth}{3}
\usepackage{palatino}
\usepackage{marginnote}
\usepackage{multirow}
\usepackage{easybmat,bigdelim,arydshln}
\usepackage[authoryear,round]{natbib}
\usepackage{amssymb,amsmath,amsthm,amsfonts}
\usepackage{mathtools}
%\usepackage{nicematrix}
\usepackage{arydshln}
\usepackage{caption}
\usepackage{hyperref}
\usepackage{tcolorbox}
\tcbuselibrary{skins, breakable, theorems}
\usepackage{newpxtext,newpxmath}
\usepackage{longtable}
\usepackage{enumitem}
\makeatletter

\let\bar\overline

\setlist[itemize]{topsep=0pt,leftmargin=10pt,itemsep=-0.2em}
\usepackage{xcolor}
\usepackage{tikz}
\usepackage{pgfplots}
\pgfplotsset{compat = newest}
\usetikzlibrary{patterns,decorations.pathreplacing,decorations.markings,fit,shapes.geometric,angles,quotes,arrows}
\usepgfplotslibrary{fillbetween}

\usepackage{ifthen}
\usepackage{tikz-3dplot}

\pgfdeclarelayer{ft}
\pgfdeclarelayer{bg}
\pgfsetlayers{bg,main,ft}

\hypersetup{
    colorlinks,
    citecolor=red,
    filecolor=black,
    linkcolor=violet,
    urlcolor=blue
}

\definecolor{myblue}{cmyk}{1,.72,0,.38}
\definecolor{mypurple}{cmyk}{.57,1,0,.58}
\definecolor{myred}{cmyk}{0,.88,.88,.58}
\definecolor{mygreen}{cmyk}{1,0,.69,.66}
\definecolor{myorange}{cmyk}{0,.58,100,.20}
\definecolor{glaucous}{rgb}{0.38, 0.51, 0.71}

\makeatletter
\renewcommand{\thefigure}{\thesection.\arabic{figure}}
\newtheoremstyle{indented}
  {3pt}% space before
  {3pt}% space after
  {\addtolength{\@totalleftmargin}{3.5em}
   \addtolength{\linewidth}{-3.5em}
   \parshape 1 3.5em \linewidth}% body font
  {}% indent
  {\bfseries}% header font
  {.}% punctuation
  {.5em}% after theorem header
  {}% header specification (empty for default)
\makeatother

\newcommand{\ind}{\perp\!\!\!\perp}

\theoremstyle{definition}
\newtheorem{defin}{Definition}[section] % Creates a new counter, number within section
\newtheorem{prt}[defin]{Remark} 
\newtheorem{prts}[defin]{Remarks} % Again share defin's counter
\newtheorem{exmp}[defin]{Example} % etc.
\newtheorem{exmps}[defin]{Examples}
\newtheorem*{note}{Note}
\tcbuselibrary{theorems}

% use counter*=defin to make each tcbtheorem share defin's counter

\newtcbtheorem[use counter*=defin, number within=section]{definition}{Definition}{enhanced, breakable,
    colback = white, colframe = red!55!black, colbacktitle = red!55!black, attach boxed title to top left = {yshift = -2.5mm, xshift = 3mm}, boxed title style = {sharp corners},fonttitle=\bfseries}{def}

\newtcbtheorem[use counter*=defin, number within=section]{theorem}{Theorem}{enhanced, breakable,
    colback = white, colframe = blue!45!black, colbacktitle = blue!45!black, attach boxed title to top left = {yshift = -2.5mm, xshift = 3mm}, boxed title style = {sharp corners},fonttitle=\bfseries}{thm}
    
\newtcbtheorem[use counter*=defin, number within=section]{proposition}{Proposition}{enhanced, breakable,
    colback = white, colframe = teal, colbacktitle = teal, attach boxed title to top left = {yshift = -2.5mm, xshift = 3mm}, boxed title style = {sharp corners},fonttitle=\bfseries}{prop}

\newtcbtheorem[use counter*=defin, number within=section]{lemma}{Lemma}{enhanced, breakable,
    colback = white, colframe = orange!80!black, colbacktitle = orange!80!black, attach boxed title to top left = {yshift = -2.5mm, xshift = 3mm}, boxed title style = {sharp corners},fonttitle=\bfseries}{lemma}

\newtcbtheorem[use counter*=defin, number within=section]{example}{Example}{enhanced, breakable,
    colback = white, colframe = yellow!60!black, colbacktitle = yellow!60!black, attach boxed title to top left = {yshift = -2.5mm, xshift = 3mm}, boxed title style = {sharp corners},fonttitle=\bfseries}{exmp}

\newtcbtheorem[use counter*=defin, number within=section]{assumption}{Assumption}{enhanced, breakable,
    colback = white, colframe = violet!60!white, colbacktitle = violet!60!white, attach boxed title to top left = {yshift = -2.5mm, xshift = 3mm}, boxed title style = {sharp corners},fonttitle=\bfseries}{assump}

\newtcbtheorem[use counter*=defin, number within=section]{algorithm}{Algorithm}{enhanced, breakable,
    colback = white, colframe = green!55!black, colbacktitle = green!55!black, attach boxed title to top left = {yshift = -2.5mm, xshift = 3mm}, boxed title style = {sharp corners},fonttitle=\bfseries}{algm}
%\newtcolorbox{example}[1]{enhanced, breakable, colback = white, colframe = orange!85!black, colbacktitle = orange!85!black, attach boxed title to top left = {yshift = -2.5mm, xshift = 3mm}, boxed title style = {sharp corners},fonttitle=\bfseries, title={Example: #1}}

\newtcbox{\myhl}[1][white]
  {on line, arc = 0pt, outer arc = 0pt,
    colback = #1!20!white, colframe = #1!50!black,
    boxsep = 0pt, left = 1pt, right = 1pt, top = 1pt, bottom = 1pt, boxrule = 0pt, bottomrule =0pt, toprule =0pt}
    
\newtcbox{\myhlrule}[1][white]
  {on line, arc = 0pt, outer arc = 0pt,
    colback = #1!20!white, colframe = #1!50!black,
    boxsep = 0pt, left = 1pt, right = 1pt, top = 1pt, bottom = 1pt, boxrule = 0pt, bottomrule =0.5pt, toprule =0.5pt}
%
% The following commands set up the lecnum (lecture number)
% counter and make various numbering schemes work relative
% to the lecture number.
%
\newcounter{lecnum}
\renewcommand{\thepage}{\thelecnum-\arabic{page}}
\renewcommand{\thesection}{\thelecnum.\arabic{section}}
\renewcommand{\theequation}{\thelecnum.\arabic{equation}}
\renewcommand{\thefigure}{\thelecnum.\arabic{figure}}
\renewcommand{\thetable}{\thelecnum.\arabic{table}}

\newcommand{\sidenotes}[1]{\marginnote{\raggedright\scriptsize#1}}
%
% The following macro is used to generate the header.
%
\newcommand{\lecture}[6]{
   \pagestyle{myheadings}
   \thispagestyle{plain}
   \newpage
   \setcounter{lecnum}{#1}
   \setcounter{page}{1}
   \noindent
   \begin{center}
   \framebox{
      \vbox{\vspace{2mm}
    \hbox to 6.28in { {\bf Econometrics
	\hfill \today} }
       \vspace{4mm}
       \hbox to 6.28in { {\Large \hfill Topic #1: #2  \hfill} }
       \vspace{2mm}
       \hbox to 6.28in { {\it #3 \hfill by #4} }
      \vspace{2mm}}
   }
   \end{center}
   \markboth{Week #1: #2}{Week #1: #2}

   {\bf Key points}: {#5}

   {\bf Disclaimer}: {\it #6}
   \vspace*{4mm}
}
%

\tikzset{-stealth-/.style={decoration={
  markings,
  mark=at position #1 with {\arrow{stealth}}},postaction={decorate}}}

  \tikzset{tangent/.style={
    decoration={
        markings,% switch on markings
        mark=
            at position #1
            with
            {
                \coordinate (tangent point-\pgfkeysvalueof{/pgf/decoration/mark info/sequence number}) at (0pt,0pt);
                \coordinate (tangent unit vector-\pgfkeysvalueof{/pgf/decoration/mark info/sequence number}) at (1,0pt);
                \coordinate (tangent orthogonal unit vector-\pgfkeysvalueof{/pgf/decoration/mark info/sequence number}) at (0pt,1);
            }
    },
    postaction=decorate
},
use tangent/.style={
    shift=(tangent point-#1),
    x=(tangent unit vector-#1),
    y=(tangent orthogonal unit vector-#1)
},
use tangent/.default=1}

\tikzstyle{terminator} = [rectangle, draw, thick, text centered, rounded corners, minimum height=2em]
\tikzstyle{process} = [rectangle, draw, thick, text centered, minimum height=2em]
\tikzstyle{decision} = [diamond, draw, thick, text centered, minimum width=3cm, minimum height=0.5cm]
\tikzstyle{data}=[trapezium, draw, thick, text centered, trapezium left angle=60, trapezium right angle=120, minimum height=2em]
\tikzstyle{arrow} = [thick,->,>=stealth]

\begin{document}
\lecture{5}{Two-Way Cluster-Robust (TWCR) Standard Errors}{}{Sai Zhang}{The validity of Two-Way Cluster-Robust (TWCR) standard errors}{This note is compiled by Sai Zhang.}
%\footnotetext{These notes are partially based on those of Nigel Mansell.}

\section{One-Way Clustering}
First, consider the case of one-way clustering. The linear model with one-way clustering $$ y_{ig} = \mathbf{x}_{ig}\boldsymbol{\beta} + u_{ig} $$
where $i$ denotes the $i$th of the $N$ individuals in the sample, $j$ denotes the $g$th of the $G$ clusters, assume that
\begin{itemize}
    \item $\mathbb{E}\left[u_{ig}\mid \mathbf{x}_{ig}\right] =0$
    \item error independence across clusters: for $i\neq j$
    \begin{equation}\label{eq:error_independence}
        \mathbb{E}\left[ u_{ig} u_{jg'}\mid \mathbf{x}_{ig},\mathbf{x}_{jg'} \right] = 0
    \end{equation}
    unless $g=g'$, that is, errors for individuals within the same cluster may be correlated.
\end{itemize}
Grouping observations by cluster, get
$$
\mathbf{y}_g = \mathbf{X}_g \boldsymbol{\beta} + \mathbf{u}
$$
where $\mathbf{X}_g$ has dimension $N_g\times K$ and $\mathbf{y}_g$ has dimension $N_g \times 1$, with $N_g$ observations in cluster $g$. 
Stacking over cluster, get the matrix form of the model
$$
\mathbf{y=X}\boldsymbol{\beta}+\mathbf{u}
$$
with $\mathbf{y,u}$ being $N\times 1$ vectors, $\mathbf{X}$ being an $N\times K$ matrix. OLS estimator gives 
\begin{equation}\label{eq:OLSest}
    \hat{\boldsymbol{\beta}} = \left(\mathbf{X'X}\right)^{-1}\mathbf{X'y}=\left( \sum^G_{g=1}\mathbf{X}_g'\mathbf{X}_g \right)^{-1} \sum^G_{g=1}\mathbf{X}'_g\mathbf{y}_g
\end{equation}
then, by CLT, we have that $\sqrt{G} \left(\hat{\boldsymbol{\beta}}-\boldsymbol{\beta}\right) \xrightarrow{d} \mathcal{N}(0,\boldsymbol{\Sigma})$ where the variance matrix of the limit normal distribution $\boldsymbol{\Sigma}$ is 
\begin{equation}\label{eq:limit_varcovmat}
    \left( \lim_{G\rightarrow\infty}\frac{1}{G}\sum^G_{g=1} \mathbf{E}\left[\mathbf{X}'_g\mathbf{X}_g\right] \right)^{-1} \left(\lim_{G\rightarrow\infty}\frac{1}{G}\sum^G_{g=1} \mathbf{E}\left[\mathbf{X}'_g\mathbf{u}'_g\mathbf{u}_g\mathbf{X}_g\right] \right) \times \left( \lim_{G\rightarrow\infty}\frac{1}{G}\sum^G_{g=1} \mathbf{E}\left[\mathbf{X}'_g\mathbf{X}_g\right]  \right)^{-1}
\end{equation}
If the primary source of clustering is due to group-level common shocks, a useful approximation is that for the $j$th regressor, the default OLS variance estimate based on $s^2 \left(\mathbf{X'X}\right)^{-1}$ should be inflated by $\tau_j \simeq 1+\rho_{x_j}\rho_u\left(\bar{N}_g -1\right)$, where 
\begin{itemize}
    \item $s$ is the estimated standard deviation of the error
    \item $\rho_{x_j}$ is a measure of within-cluster correlation of $x_j$
    \item $\rho_u$ is the within-cluster error correlation 
    \item $\bar{N}_g$ is the average cluster size
\end{itemize}
It's easy to see the $\tau_j$ can be large even with small $\rho_u$ \citep{kloek1981ols,scott1982effect,moulton1990illustration}. If assume the model for the cluster error variance matrices $\boldsymbol{\Omega}_g = \mathbb{V}\left[\mathbf{u}_g \mid \mathbf{X}_g\right] = \mathbb{E}\left[\mathbf{u}_g\mathbf{u}_g'\mid \mathbf{X}_g\right]$, 
and there is a consistent estimate $\hat{\boldsymbol{\Omega}}_g$ of $\boldsymbol{\Omega}_g$, we can estimate $\mathbb{E}\left[\mathbf{X}_g'\mathbf{u}_g\mathbf{u}_g'\mathbf{X}_g\right] = \mathbb{E}\left[\mathbf{X}_g'\boldsymbol{\Omega}_g \mathbf{X}_g\right]$ via GLS.

\paragraph*{Cluster-robust variance matrix estimate} consider 
\begin{equation}\label{eq:oneway_clurob}
    \hat{\mathbb{V}} \left[\hat{\boldsymbol{\beta}}\right] = \left(\mathbf{X'X}\right)^{-1}\left(\sum^G_{g=1} \mathbf{X}_g'\hat{\mathbf{u}}_g \hat{\mathbf{u}}_g' \mathbf{X}_g \right) \left(\mathbf{X'X}\right)^{-1}
\end{equation}
where $\hat{\mathbf{u}}_g = \mathbf{y}_g - \mathbf{X}_g\hat{\boldsymbol{\beta}}$. This estimate is consistent if $$ G^{-1}\sum^G_{g=1}\mathbf{X}_g'\hat{\mathbf{u}}_g \hat{\mathbf{u}}_g' \mathbf{X}_g - G^{-1}\sum^G_{g=1}\mathbb{E}\left[ \mathbf{X}_g' \mathbf{u}_g \mathbf{u}_g' \mathbf{X}_g \right] \xrightarrow{\mathrm{p}} \mathbf{0} $$ as $G\rightarrow \infty$. 
An informal presentation of Eq.(\ref{eq:oneway_clurob}) is to rewrite the central matrix as 
\begin{equation}\label{eq:onewayclu_centralmat}
    \hat{\mathbf{B}} = \sum^G_{g=1} \mathbf{X}_g'\hat{\mathbf{u}}_g \hat{\mathbf{u}}_g' \mathbf{X}_g = \mathbf{X}'\begin{bmatrix}
        \hat{\mathbf{u}}_1\hat{\mathbf{u}}_1' & \mathbf{0} & \cdots & \mathbf{0}\\
        \mathbf{0} & \hat{\mathbf{u}}_2\hat{\mathbf{u}}_2' & & \vdots \\
        \vdots & & \ddots & \mathbf{0} \\
        \mathbf{0} & \cdots & & \hat{\mathbf{u}}_G\hat{\mathbf{u}}_G'
    \end{bmatrix}\mathbf{X} = \mathbf{X}'\left(\hat{\mathbf{u}}\hat{\mathbf{u}}' \otimes \mathbf{S}^G \right) \mathbf{X} 
\end{equation}
where $\otimes$ denotes element-wise multiplication. The $(p,q)$th element of this matrix is 
\begin{equation*}
    \sum^N_{i=1}\sum^N_{j=1}x_{ia}x_{jb}\hat{u}_i\hat{u}_j \cdot \mathbf{1}\left(i,j\text{ in the same cluster}\right)
\end{equation*}
with $\hat{u}_i = y_i - \mathbf{x}'_i \hat{\boldsymbol{\beta}}$.

$\mathbf{S}^G$ is an $N\times N$ indicator matrix with $\mathbf{S}_{ij}^G=1$ only if the $i$th and $j$th observation belong to the same cluster: it zeros out a large amount of $\hat{\mathbf{u}}\hat{\mathbf{u}}'$ (asymptotically equivalently, ${\mathbf{u}}{\mathbf{u}}'$), specifically, only $\sum^G_{g=1}N_g^2$ out of $N^2 = \left(\sum^G_{g=1}N_g\right)^2$ terms are not zero (sub-matrices on the diagonal). Asymptotically
\begin{itemize}
    \item for fixed $N_g$, $\frac{1}{{N^2}}\sum^G_{g=1}{N^2_g}\xrightarrow{G\rightarrow\infty}0$
    \item for balanced clusters $N_g = N/G$, $\frac{1}{{N^2}}\sum^G_{g=1}{N^2_g} = \frac{1}{G} \xrightarrow{G\rightarrow\infty}0$
\end{itemize}


A strand of literature popularizes 
this method:
\begin{itemize}
    \item \citet{liang1986longitudinal}: in a generalized estimatin equations setting
    \item \citet{arellano1987computing}: fixed effects estimator in linear panel models
    \item \citet{hansen2007asymptotic}: asymptotic theory for panel data where $T\rightarrow\infty$ in addition to $N\rightarrow\infty$ (or $N_g\rightarrow\infty$ in addition to $G\rightarrow\infty$ in the notation above).
\end{itemize}

\section{Two-Way Clustering}
Now, consider the case of two-way clustering, 
$$
y_{i,gh} = \mathbf{x}'_{i,gh}\boldsymbol{\beta} + u
$$
where each observation may belong to \textbf{two} dimension of groups: group $g\in \left\{1,\cdots,G\right\}$ and $h\in \left\{ 1,\cdots,H \right\}$, and for $i\neq j$
\begin{equation}\label{eq:twoway_errors}
    \mathbb{E} \left[ u_{i,gh} u_{j,g'h'} \mid \mathbf{x}_{i,gh},\mathbf{j,g'h'} \right] = 0
\end{equation}
unless $g=g'$ or $h=h'$, that is, errors for individuals within the same group (along either $g$ or $h$) may be correlated.

\paragraph*{Cluster-robust variance matrix estimate} extending the one-way clustering case, keep elements of $\hat{\mathbf{u}}\hat{\mathbf{u}}'$ where the $i$th and $j$th observations share a cluster in \myhl[myblue]{\textbf{any}} dimension, then similar to Eq.(\ref{eq:onewayclu_centralmat})
\begin{equation}\label{eq:twowayclu_centermat}
    \hat{\mathbf{B}} = \mathbf{X}'\left(\hat{\mathbf{u}}\hat{\mathbf{u}}' \otimes \mathbf{S}^{GH}\right)\mathbf{X}
\end{equation}
here $\mathbf{S}^{GH}$ is an $N\times N$ indicator matrix with $\mathbf{S}^{GH}_{ij}=1$ only if the $i$th and $j$th observation share any cluster, the $(p,q)$th element of this matrix is 
$$
\sum^N_{i=1}\sum^N_{i=1}x_{ia}x_{jb}\hat{u}_i\hat{u}_j \cdot \mathbf{1} \left(i,j\text{ share any cluster}\right)
$$
$\hat{\mathbf{B}}$ can also be presented in one-way cluster-robust fashion:
\begin{align}\label{eq:twowayclu_centmat}
    \hat{\mathbf{B}} &= \mathbf{X}'\left( \hat{\mathbf{u}}\hat{\mathbf{u}}' \otimes \mathbf{S}^{GH} \right)\mathbf{X} = \mathbf{X}'\left( \hat{\mathbf{u}}\hat{\mathbf{u}}' \otimes \mathbf{S}^G \right)\mathbf{X} + \mathbf{X}'\left( \hat{\mathbf{u}}\hat{\mathbf{u}}' \otimes \mathbf{S}^H \right)\mathbf{X} - \mathbf{X}'\left( \hat{\mathbf{u}}\hat{\mathbf{u}}' \otimes \mathbf{S}^{G\cap H} \right)\mathbf{X}
\end{align}
where $\mathbf{G}^{GH} =\mathbf{G}^G+ \mathbf{G}^H - \mathbf{G}^{G\cap H} $, with 
\begin{itemize}
    \item $\mathbf{G}^G$: $\mathbf{G}^G_{ij}=1$ only if the $i$th and $j$th observation belong to the same cluster $g\in \left\{1,2,\cdots,G\right\}$
    \item $\mathbf{G}^H$: $\mathbf{G}^H_{ij}=1$ only if the $i$th and $j$th observation belong to the same cluster $h\in \left\{1,2,\cdots,H\right\}$
    \item $\mathbf{G}^{G\cap H}$: $\mathbf{G}^{G\cap H}_{ij}=1$ only if the $i$th and $j$th observation belong to \textbf{both} the same cluster $g\in \left\{1,2,\cdots,G\right\}$ and the same cluster $h\in \left\{1,2,\cdots,H\right\}$
\end{itemize}
then, similar to one-way clustering case,
\begin{align}
    \hat{\mathbb{V}}\left[\hat{\boldsymbol{\beta}}\right] =& \left(\mathbf{X'X}\right)^{-1}\mathbf{X}'\left(\hat{\mathbf{u}}\hat{\mathbf{u}}'\otimes \mathbf{S}^G\right)\mathbf{X}\left(\mathbf{X'X}\right)^{-1} \\ \nonumber
    &+ \left(\mathbf{X'X}\right)^{-1}\mathbf{X}'\left(\hat{\mathbf{u}}\hat{\mathbf{u}}'\otimes \mathbf{S}^H\right)\mathbf{X}\left(\mathbf{X'X}\right)^{-1} \\ \nonumber
    &- \left(\mathbf{X'X}\right)^{-1}\mathbf{X}'\left(\hat{\mathbf{u}}\hat{\mathbf{u}}'\otimes \mathbf{S}^{G\cap H}\right)\mathbf{X}\left(\mathbf{X'X}\right)^{-1}
\end{align}
that is,
\begin{equation}\label{eq:twowayclu_decomp}
    \hat{\mathbb{V}}\left[\hat{\boldsymbol{\beta}}\right] = \hat{\mathbb{V}}^G\left[\hat{\boldsymbol{\beta}}\right] + \hat{\mathbb{V}}^H\left[\hat{\boldsymbol{\beta}}\right] - \hat{\mathbb{V}}^{G\cap H}\left[\hat{\boldsymbol{\beta}}\right]
\end{equation}
each of Eq.(\ref{eq:twowayclu_decomp}) can be separately computed by OLS of $\mathbf{y}$ on $\mathbf{X}$, with variance matrix estimates $\hat{\mathbb{V}}$ based on 
\begin{itemize}
    \item[i] clustering on $g\in \left\{1,2,\cdots, G\right\}$
    \item[ii] clustering on $h \in \left\{1,2,\cdots, H\right\}$
    \item[iii] clustering on $(g,h)\in \left\{(1,1),\cdots,(G,H)\right\}$
\end{itemize}

\paragraph*{Practical considerations} It is required to know what \textit{ways} will be potentially important for clustering, which can be tested via checking the dimension of correlations in the errors. There are several ways to test 
\begin{itemize}
    \item estimate sample covariances of $\mathbf{X}'\hat{\mathbf{u}}$ within dimensions, test the null that the \myhl[myblue]{\textbf{average}} of such covariances is 0: rejecting this null is sufficient (not necessary) to reject the null of no clustering \citep{white1980heteroskedasticity}
    \item for \myhl[myblue]{\textbf{small samples}}, Eq. (\ref{eq:oneway_clurob}) is baised downwards. This is corrected (in Stata) by replacing $\hat{\mathbf{u}}_g$ with $\sqrt{c}\hat{\mathbf{u}}_g$, where $c = \frac{G}{G-1}\frac{N-1}{N-K}\simeq \frac{G}{G-1}$. For two-way clustering (Eq. \ref{eq:twowayclu_centmat}), there are 2 ways of correction:
    \begin{itemize}
        \item choose correction terms for each of the 3 components: $$c_1= \frac{G}{G-1}\frac{N-1}{N-K}, c_2= \frac{H}{H-1}\frac{N-1}{N-K},c_3=\frac{I}{I-1}\frac{N-1}{N-K}$$ with $I$ being the number of unique clusters determined by $G\cap H$
        \item choose a constant terms for all components: $$c=\frac{J}{J-1}\frac{N-1}{N-K}$$ with $J=\min(G,H)$
    \end{itemize}
    \item \myhl[myblue]{\textbf{Var-cov matrix not positive-semidefinite}}: $\hat{\mathbb{V}}\left[\hat{\boldsymbol{\beta}}\right]$ might have negative elements on the diagonal (Eq. \ref{eq:twowayclu_decomp}), informly, this is more likely to  arise when clustering is done over the same groups as the fixed effects. One way to address this issue is using \textit{eigendecomposition} technique:
    $$
    \hat{\mathbb{V}}\left[\hat{\boldsymbol{\beta}}\right] = \mathbf{U}\boldsymbol{\Lambda}\mathbf{U}'
    $$
    where 
    \begin{itemize}
        \item $\mathbf{U}$ containing the eigenvectors of $\hat{\mathbf{V}}$
        \item $\boldsymbol{\Lambda} = \mathrm{diag}\left[\lambda_1,\cdots,\lambda_d\right]$ contains the eigenvalues of $\hat{\mathbf{V}}$ 
    \end{itemize}
    then create $\boldsymbol{\Lambda}^+ = \mathrm{diag} \left[\lambda_1^+,\cdots,\lambda_d^+\right]$ with $\lambda^+_j=\max\left(0,\lambda_j\right)$ and use $\hat{\mathbf{V}}^+\left[\hat{\boldsymbol{\beta}}\right]=\mathbf{U}\boldsymbol{\Lambda}^+\mathbf{U}'$ as the estimate
\end{itemize}

\newpage
\bibliographystyle{plainnat}
\bibliography{ref.bib}

\end{document}