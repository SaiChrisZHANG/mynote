\documentclass[twoside]{article}
\setlength{\oddsidemargin}{0 in}
\setlength{\evensidemargin}{0 in}
\setlength{\topmargin}{-0.6 in}
\setlength{\textwidth}{6.5 in}
\setlength{\textheight}{8.5 in}
\setlength{\headsep}{0.75 in}
\setlength{\parindent}{0 in}
\setlength{\parskip}{0.1 in}

\usepackage{url}
\usepackage{titlesec}
\setcounter{secnumdepth}{3}
\usepackage{palatino}
\usepackage{marginnote}
\usepackage{multirow}
\usepackage{easybmat,bigdelim,arydshln}
\usepackage[authoryear,round]{natbib}
\usepackage{amssymb,amsmath,amsthm,amsfonts}
\usepackage{mathtools}
%\usepackage{nicematrix}
\usepackage{arydshln}
\usepackage{caption}
\usepackage{hyperref}
\usepackage{tcolorbox}
\tcbuselibrary{skins, breakable, theorems}
\usepackage{newpxtext,newpxmath}
\usepackage{longtable}
\usepackage{enumitem}
\makeatletter

\let\bar\overline

\setlist[itemize]{topsep=0pt,leftmargin=10pt,itemsep=-0.2em}
\usepackage{xcolor}
\usepackage{tikz}
\usepackage{pgfplots}
\pgfplotsset{compat = newest}
\usetikzlibrary{patterns,decorations.pathreplacing,decorations.markings,fit,shapes.geometric,angles,quotes,arrows}
\usepgfplotslibrary{fillbetween}

\usepackage{ifthen}
\usepackage{tikz-3dplot}

\pgfdeclarelayer{ft}
\pgfdeclarelayer{bg}
\pgfsetlayers{bg,main,ft}

\hypersetup{
    colorlinks,
    citecolor=red,
    filecolor=black,
    linkcolor=violet,
    urlcolor=blue
}

\definecolor{myblue}{cmyk}{1,.72,0,.38}
\definecolor{mypurple}{cmyk}{.57,1,0,.58}
\definecolor{myred}{cmyk}{0,.88,.88,.58}
\definecolor{mygreen}{cmyk}{1,0,.69,.66}
\definecolor{myorange}{cmyk}{0,.58,100,.20}
\definecolor{glaucous}{rgb}{0.38, 0.51, 0.71}

\makeatletter
\renewcommand{\thefigure}{\thesection.\arabic{figure}}
\newtheoremstyle{indented}
  {3pt}% space before
  {3pt}% space after
  {\addtolength{\@totalleftmargin}{3.5em}
   \addtolength{\linewidth}{-3.5em}
   \parshape 1 3.5em \linewidth}% body font
  {}% indent
  {\bfseries}% header font
  {.}% punctuation
  {.5em}% after theorem header
  {}% header specification (empty for default)
\makeatother

\newcommand{\ind}{\perp\!\!\!\perp}

\theoremstyle{definition}
\newtheorem{defin}{Definition}[section] % Creates a new counter, number within section
\newtheorem{prt}[defin]{Remark} 
\newtheorem{prts}[defin]{Remarks} % Again share defin's counter
\newtheorem{exmp}[defin]{Example} % etc.
\newtheorem{exmps}[defin]{Examples}
\newtheorem*{note}{Note}
\tcbuselibrary{theorems}

% use counter*=defin to make each tcbtheorem share defin's counter

\newtcbtheorem[use counter*=defin, number within=section]{definition}{Definition}{enhanced, breakable,
    colback = white, colframe = red!55!black, colbacktitle = red!55!black, attach boxed title to top left = {yshift = -2.5mm, xshift = 3mm}, boxed title style = {sharp corners},fonttitle=\bfseries}{def}

\newtcbtheorem[use counter*=defin, number within=section]{theorem}{Theorem}{enhanced, breakable,
    colback = white, colframe = blue!45!black, colbacktitle = blue!45!black, attach boxed title to top left = {yshift = -2.5mm, xshift = 3mm}, boxed title style = {sharp corners},fonttitle=\bfseries}{thm}
    
\newtcbtheorem[use counter*=defin, number within=section]{proposition}{Proposition}{enhanced, breakable,
    colback = white, colframe = teal, colbacktitle = teal, attach boxed title to top left = {yshift = -2.5mm, xshift = 3mm}, boxed title style = {sharp corners},fonttitle=\bfseries}{prop}

\newtcbtheorem[use counter*=defin, number within=section]{lemma}{Lemma}{enhanced, breakable,
    colback = white, colframe = orange!80!black, colbacktitle = orange!80!black, attach boxed title to top left = {yshift = -2.5mm, xshift = 3mm}, boxed title style = {sharp corners},fonttitle=\bfseries}{lemma}

\newtcbtheorem[use counter*=defin, number within=section]{example}{Example}{enhanced, breakable,
    colback = white, colframe = yellow!60!black, colbacktitle = yellow!60!black, attach boxed title to top left = {yshift = -2.5mm, xshift = 3mm}, boxed title style = {sharp corners},fonttitle=\bfseries}{exmp}

\newtcbtheorem[use counter*=defin, number within=section]{assumption}{Assumption}{enhanced, breakable,
    colback = white, colframe = violet!60!white, colbacktitle = violet!60!white, attach boxed title to top left = {yshift = -2.5mm, xshift = 3mm}, boxed title style = {sharp corners},fonttitle=\bfseries}{assump}

\newtcbtheorem[use counter*=defin, number within=section]{algorithm}{Algorithm}{enhanced, breakable,
    colback = white, colframe = green!55!black, colbacktitle = green!55!black, attach boxed title to top left = {yshift = -2.5mm, xshift = 3mm}, boxed title style = {sharp corners},fonttitle=\bfseries}{algm}
%\newtcolorbox{example}[1]{enhanced, breakable, colback = white, colframe = orange!85!black, colbacktitle = orange!85!black, attach boxed title to top left = {yshift = -2.5mm, xshift = 3mm}, boxed title style = {sharp corners},fonttitle=\bfseries, title={Example: #1}}

\newtcbox{\myhl}[1][white]
  {on line, arc = 0pt, outer arc = 0pt,
    colback = #1!20!white, colframe = #1!50!black,
    boxsep = 0pt, left = 1pt, right = 1pt, top = 1pt, bottom = 1pt, boxrule = 0pt, bottomrule =0pt, toprule =0pt}
    
\newtcbox{\myhlrule}[1][white]
  {on line, arc = 0pt, outer arc = 0pt,
    colback = #1!20!white, colframe = #1!50!black,
    boxsep = 0pt, left = 1pt, right = 1pt, top = 1pt, bottom = 1pt, boxrule = 0pt, bottomrule =0.5pt, toprule =0.5pt}
%
% The following commands set up the lecnum (lecture number)
% counter and make various numbering schemes work relative
% to the lecture number.
%
\newcounter{lecnum}
\renewcommand{\thepage}{\thelecnum-\arabic{page}}
\renewcommand{\thesection}{\thelecnum.\arabic{section}}
\renewcommand{\theequation}{\thelecnum.\arabic{equation}}
\renewcommand{\thefigure}{\thelecnum.\arabic{figure}}
\renewcommand{\thetable}{\thelecnum.\arabic{table}}

\newcommand{\sidenotes}[1]{\marginnote{\raggedright\scriptsize#1}}
%
% The following macro is used to generate the header.
%
\newcommand{\lecture}[6]{
   \pagestyle{myheadings}
   \thispagestyle{plain}
   \newpage
   \setcounter{lecnum}{#1}
   \setcounter{page}{1}
   \noindent
   \begin{center}
   \framebox{
      \vbox{\vspace{2mm}
    \hbox to 6.28in { {\bf Econometrics
	\hfill \today} }
       \vspace{4mm}
       \hbox to 6.28in { {\Large \hfill Topic #1: #2  \hfill} }
       \vspace{2mm}
       \hbox to 6.28in { {\it #3 \hfill by #4} }
      \vspace{2mm}}
   }
   \end{center}
   \markboth{Week #1: #2}{Week #1: #2}

   {\bf Key points}: {#5}

   {\bf Disclaimer}: {\it #6}
   \vspace*{4mm}
}
%

\tikzset{-stealth-/.style={decoration={
  markings,
  mark=at position #1 with {\arrow{stealth}}},postaction={decorate}}}

  \tikzset{tangent/.style={
    decoration={
        markings,% switch on markings
        mark=
            at position #1
            with
            {
                \coordinate (tangent point-\pgfkeysvalueof{/pgf/decoration/mark info/sequence number}) at (0pt,0pt);
                \coordinate (tangent unit vector-\pgfkeysvalueof{/pgf/decoration/mark info/sequence number}) at (1,0pt);
                \coordinate (tangent orthogonal unit vector-\pgfkeysvalueof{/pgf/decoration/mark info/sequence number}) at (0pt,1);
            }
    },
    postaction=decorate
},
use tangent/.style={
    shift=(tangent point-#1),
    x=(tangent unit vector-#1),
    y=(tangent orthogonal unit vector-#1)
},
use tangent/.default=1}

\tikzstyle{terminator} = [rectangle, draw, thick, text centered, rounded corners, minimum height=2em]
\tikzstyle{process} = [rectangle, draw, thick, text centered, minimum height=2em]
\tikzstyle{decision} = [diamond, draw, thick, text centered, minimum width=3cm, minimum height=0.5cm]
\tikzstyle{data}=[trapezium, draw, thick, text centered, trapezium left angle=60, trapezium right angle=120, minimum height=2em]
\tikzstyle{arrow} = [thick,->,>=stealth]

\begin{document}
\lecture{5}{Cluster-Robust Standard Errors}{}{Sai Zhang}{The validity of Two-Way Cluster-Robust (TWCR) standard errors}{This note is compiled by Sai Zhang.}
%\footnotetext{These notes are partially based on those of Nigel Mansell.}

\section{One-Way Clustering}\label{sec:oneway_cluster}
First, consider the case of one-way clustering. The linear model with one-way clustering $$ y_{ig} = \mathbf{x}_{ig}\boldsymbol{\beta} + u_{ig} $$
where $i$ denotes the $i$th of the $N$ individuals in the sample, $j$ denotes the $g$th of the $G$ clusters, assume that
\begin{itemize}
    \item $\mathbb{E}\left[u_{ig}\mid \mathbf{x}_{ig}\right] =0$
    \item error independence across clusters: for $i\neq j$
    \begin{equation}\label{eq:error_independence}
        \mathbb{E}\left[ u_{ig} u_{jg'}\mid \mathbf{x}_{ig},\mathbf{x}_{jg'} \right] = 0
    \end{equation}
    unless $g=g'$, that is, errors for individuals within the same cluster may be correlated.
\end{itemize}
Grouping observations by cluster, get
$$
\mathbf{y}_g = \mathbf{X}_g \boldsymbol{\beta} + \mathbf{u}
$$
where $\mathbf{X}_g$ has dimension $N_g\times K$ and $\mathbf{y}_g$ has dimension $N_g \times 1$, with $N_g$ observations in cluster $g$. 
Stacking over cluster, get the matrix form of the model
$$
\mathbf{y=X}\boldsymbol{\beta}+\mathbf{u}
$$
with $\mathbf{y,u}$ being $N\times 1$ vectors, $\mathbf{X}$ being an $N\times K$ matrix. OLS estimator gives 
\begin{equation}\label{eq:OLSest}
    \hat{\boldsymbol{\beta}} = \left(\mathbf{X'X}\right)^{-1}\mathbf{X'y}=\left( \sum^G_{g=1}\mathbf{X}_g'\mathbf{X}_g \right)^{-1} \sum^G_{g=1}\mathbf{X}'_g\mathbf{y}_g
\end{equation}
then, by CLT, we have that $\sqrt{G} \left(\hat{\boldsymbol{\beta}}-\boldsymbol{\beta}\right) \xrightarrow{d} \mathcal{N}(0,\boldsymbol{\Sigma})$ where the variance matrix of the limit normal distribution $\boldsymbol{\Sigma}$ is 
\begin{equation}\label{eq:limit_varcovmat}
    \left( \lim_{G\rightarrow\infty}\frac{1}{G}\sum^G_{g=1} \mathbf{E}\left[\mathbf{X}'_g\mathbf{X}_g\right] \right)^{-1} \left(\lim_{G\rightarrow\infty}\frac{1}{G}\sum^G_{g=1} \mathbf{E}\left[\mathbf{X}'_g\mathbf{u}'_g\mathbf{u}_g\mathbf{X}_g\right] \right) \times \left( \lim_{G\rightarrow\infty}\frac{1}{G}\sum^G_{g=1} \mathbf{E}\left[\mathbf{X}'_g\mathbf{X}_g\right]  \right)^{-1}
\end{equation}
If the primary source of clustering is due to group-level common shocks, a useful approximation is that for the $j$th regressor, the default OLS variance estimate based on $s^2 \left(\mathbf{X'X}\right)^{-1}$ should be inflated by $\tau_j \simeq 1+\rho_{x_j}\rho_u\left(\bar{N}_g -1\right)$, where 
\begin{itemize}
    \item $s$ is the estimated standard deviation of the error
    \item $\rho_{x_j}$ is a measure of within-cluster correlation of $x_j$
    \item $\rho_u$ is the within-cluster error correlation 
    \item $\bar{N}_g$ is the average cluster size
\end{itemize}
It's easy to see the $\tau_j$ can be large even with small $\rho_u$ \citep{kloek1981ols,scott1982effect,moulton1990illustration}. If assume the model for the cluster error variance matrices $\boldsymbol{\Omega}_g = \mathbb{V}\left[\mathbf{u}_g \mid \mathbf{X}_g\right] = \mathbb{E}\left[\mathbf{u}_g\mathbf{u}_g'\mid \mathbf{X}_g\right]$, 
and there is a consistent estimate $\hat{\boldsymbol{\Omega}}_g$ of $\boldsymbol{\Omega}_g$, we can estimate $\mathbb{E}\left[\mathbf{X}_g'\mathbf{u}_g\mathbf{u}_g'\mathbf{X}_g\right] = \mathbb{E}\left[\mathbf{X}_g'\boldsymbol{\Omega}_g \mathbf{X}_g\right]$ via GLS.

\paragraph*{Cluster-robust variance matrix estimate} consider 
\begin{equation}\label{eq:oneway_clurob}
    \hat{\mathbb{V}} \left[\hat{\boldsymbol{\beta}}\right] = \left(\mathbf{X'X}\right)^{-1}\left(\sum^G_{g=1} \mathbf{X}_g'\hat{\mathbf{u}}_g \hat{\mathbf{u}}_g' \mathbf{X}_g \right) \left(\mathbf{X'X}\right)^{-1}
\end{equation}
where $\hat{\mathbf{u}}_g = \mathbf{y}_g - \mathbf{X}_g\hat{\boldsymbol{\beta}}$. This estimate is consistent if $$ G^{-1}\sum^G_{g=1}\mathbf{X}_g'\hat{\mathbf{u}}_g \hat{\mathbf{u}}_g' \mathbf{X}_g - G^{-1}\sum^G_{g=1}\mathbb{E}\left[ \mathbf{X}_g' \mathbf{u}_g \mathbf{u}_g' \mathbf{X}_g \right] \xrightarrow{\mathrm{p}} \mathbf{0} $$ as $G\rightarrow \infty$. 
An informal presentation of Eq.(\ref{eq:oneway_clurob}) is to rewrite the central matrix as 
\begin{equation}\label{eq:onewayclu_centralmat}
    \hat{\mathbf{B}} = \sum^G_{g=1} \mathbf{X}_g'\hat{\mathbf{u}}_g \hat{\mathbf{u}}_g' \mathbf{X}_g = \mathbf{X}'\begin{bmatrix}
        \hat{\mathbf{u}}_1\hat{\mathbf{u}}_1' & \mathbf{0} & \cdots & \mathbf{0}\\
        \mathbf{0} & \hat{\mathbf{u}}_2\hat{\mathbf{u}}_2' & & \vdots \\
        \vdots & & \ddots & \mathbf{0} \\
        \mathbf{0} & \cdots & & \hat{\mathbf{u}}_G\hat{\mathbf{u}}_G'
    \end{bmatrix}\mathbf{X} = \mathbf{X}'\left(\hat{\mathbf{u}}\hat{\mathbf{u}}' \otimes \mathbf{S}^G \right) \mathbf{X} 
\end{equation}
where $\otimes$ denotes element-wise multiplication. The $(p,q)$th element of this matrix is 
\begin{equation*}
    \sum^N_{i=1}\sum^N_{j=1}x_{ia}x_{jb}\hat{u}_i\hat{u}_j \cdot \mathbf{1}\left(i,j\text{ in the same cluster}\right)
\end{equation*}
with $\hat{u}_i = y_i - \mathbf{x}'_i \hat{\boldsymbol{\beta}}$.

$\mathbf{S}^G$ is an $N\times N$ indicator matrix with $\mathbf{S}_{ij}^G=1$ only if the $i$th and $j$th observation belong to the same cluster: it zeros out a large amount of $\hat{\mathbf{u}}\hat{\mathbf{u}}'$ (asymptotically equivalently, ${\mathbf{u}}{\mathbf{u}}'$), specifically, only $\sum^G_{g=1}N_g^2$ out of $N^2 = \left(\sum^G_{g=1}N_g\right)^2$ terms are not zero (sub-matrices on the diagonal). Asymptotically
\begin{itemize}
    \item for fixed $N_g$, $\frac{1}{{N^2}}\sum^G_{g=1}{N^2_g}\xrightarrow{G\rightarrow\infty}0$
    \item for balanced clusters $N_g = N/G$, $\frac{1}{{N^2}}\sum^G_{g=1}{N^2_g} = \frac{1}{G} \xrightarrow{G\rightarrow\infty}0$
\end{itemize}


A strand of literature popularizes 
this method:
\begin{itemize}
    \item \citet{liang1986longitudinal}: in a generalized estimatin equations setting
    \item \citet{arellano1987computing}: fixed effects estimator in linear panel models
    \item \citet{hansen2007asymptotic}: asymptotic theory for panel data where $T\rightarrow\infty$ in addition to $N\rightarrow\infty$ (or $N_g\rightarrow\infty$ in addition to $G\rightarrow\infty$ in the notation above).
\end{itemize}

\section{Two-Way Clustering}\label{sec:twoway_cluster}
Now, consider the case of two-way clustering, 
$$
y_{i,gh} = \mathbf{x}'_{i,gh}\boldsymbol{\beta} + u
$$
where each observation may belong to \textbf{two} dimension of groups: group $g\in \left\{1,\cdots,G\right\}$ and $h\in \left\{ 1,\cdots,H \right\}$, and for $i\neq j$
\begin{equation}\label{eq:twoway_errors}
    \mathbb{E} \left[ u_{i,gh} u_{j,g'h'} \mid \mathbf{x}_{i,gh},\mathbf{j,g'h'} \right] = 0
\end{equation}
unless $g=g'$ or $h=h'$, that is, errors for individuals within the same group (along either $g$ or $h$) may be correlated.

\paragraph*{Cluster-robust variance matrix estimate} extending the one-way clustering case, keep elements of $\hat{\mathbf{u}}\hat{\mathbf{u}}'$ where the $i$th and $j$th observations share a cluster in \myhl[myblue]{\textbf{any}} dimension, then similar to Eq.(\ref{eq:onewayclu_centralmat})
\begin{equation}\label{eq:twowayclu_centermat}
    \hat{\mathbf{B}} = \mathbf{X}'\left(\hat{\mathbf{u}}\hat{\mathbf{u}}' \otimes \mathbf{S}^{GH}\right)\mathbf{X}
\end{equation}
here $\mathbf{S}^{GH}$ is an $N\times N$ indicator matrix with $\mathbf{S}^{GH}_{ij}=1$ only if the $i$th and $j$th observation share any cluster, the $(p,q)$th element of this matrix is 
$$
\sum^N_{i=1}\sum^N_{i=1}x_{ia}x_{jb}\hat{u}_i\hat{u}_j \cdot \mathbf{1} \left(i,j\text{ share any cluster}\right)
$$
$\hat{\mathbf{B}}$ can also be presented in one-way cluster-robust fashion:
\begin{align}\label{eq:twowayclu_centmat}
    \hat{\mathbf{B}} &= \mathbf{X}'\left( \hat{\mathbf{u}}\hat{\mathbf{u}}' \otimes \mathbf{S}^{GH} \right)\mathbf{X} = \mathbf{X}'\left( \hat{\mathbf{u}}\hat{\mathbf{u}}' \otimes \mathbf{S}^G \right)\mathbf{X} + \mathbf{X}'\left( \hat{\mathbf{u}}\hat{\mathbf{u}}' \otimes \mathbf{S}^H \right)\mathbf{X} - \mathbf{X}'\left( \hat{\mathbf{u}}\hat{\mathbf{u}}' \otimes \mathbf{S}^{G\cap H} \right)\mathbf{X}
\end{align}
where $\mathbf{G}^{GH} =\mathbf{G}^G+ \mathbf{G}^H - \mathbf{G}^{G\cap H} $, with 
\begin{itemize}
    \item $\mathbf{G}^G$: $\mathbf{G}^G_{ij}=1$ only if the $i$th and $j$th observation belong to the same cluster $g\in \left\{1,2,\cdots,G\right\}$
    \item $\mathbf{G}^H$: $\mathbf{G}^H_{ij}=1$ only if the $i$th and $j$th observation belong to the same cluster $h\in \left\{1,2,\cdots,H\right\}$
    \item $\mathbf{G}^{G\cap H}$: $\mathbf{G}^{G\cap H}_{ij}=1$ only if the $i$th and $j$th observation belong to \textbf{both} the same cluster $g\in \left\{1,2,\cdots,G\right\}$ and the same cluster $h\in \left\{1,2,\cdots,H\right\}$
\end{itemize}
then, similar to one-way clustering case,
\begin{align}
    \hat{\mathbb{V}}\left[\hat{\boldsymbol{\beta}}\right] =& \left(\mathbf{X'X}\right)^{-1}\mathbf{X}'\left(\hat{\mathbf{u}}\hat{\mathbf{u}}'\otimes \mathbf{S}^G\right)\mathbf{X}\left(\mathbf{X'X}\right)^{-1} \\ \nonumber
    &+ \left(\mathbf{X'X}\right)^{-1}\mathbf{X}'\left(\hat{\mathbf{u}}\hat{\mathbf{u}}'\otimes \mathbf{S}^H\right)\mathbf{X}\left(\mathbf{X'X}\right)^{-1} \\ \nonumber
    &- \left(\mathbf{X'X}\right)^{-1}\mathbf{X}'\left(\hat{\mathbf{u}}\hat{\mathbf{u}}'\otimes \mathbf{S}^{G\cap H}\right)\mathbf{X}\left(\mathbf{X'X}\right)^{-1}
\end{align}
that is,
\begin{equation}\label{eq:twowayclu_decomp}
    \hat{\mathbb{V}}\left[\hat{\boldsymbol{\beta}}\right] = \hat{\mathbb{V}}^G\left[\hat{\boldsymbol{\beta}}\right] + \hat{\mathbb{V}}^H\left[\hat{\boldsymbol{\beta}}\right] - \hat{\mathbb{V}}^{G\cap H}\left[\hat{\boldsymbol{\beta}}\right]
\end{equation}
each of Eq.(\ref{eq:twowayclu_decomp}) can be separately computed by OLS of $\mathbf{y}$ on $\mathbf{X}$, with variance matrix estimates $\hat{\mathbb{V}}$ based on 
\begin{itemize}
    \item[i] clustering on $g\in \left\{1,2,\cdots, G\right\}$
    \item[ii] clustering on $h \in \left\{1,2,\cdots, H\right\}$
    \item[iii] clustering on $(g,h)\in \left\{(1,1),\cdots,(G,H)\right\}$
\end{itemize}

\paragraph*{Practical considerations} It is required to know what \textit{ways} will be potentially important for clustering, which can be tested via checking the dimension of correlations in the errors. There are several ways to test 
\begin{itemize}
    \item estimate sample covariances of $\mathbf{X}'\hat{\mathbf{u}}$ within dimensions, test the null that the \myhl[myblue]{\textbf{average}} of such covariances is 0: rejecting this null is sufficient (not necessary) to reject the null of no clustering \citep{white1980heteroskedasticity}
    \item for \myhl[myblue]{\textbf{small samples}}, Eq. (\ref{eq:oneway_clurob}) is baised downwards. This is corrected (in Stata) by replacing $\hat{\mathbf{u}}_g$ with $\sqrt{c}\hat{\mathbf{u}}_g$, where $c = \frac{G}{G-1}\frac{N-1}{N-K}\simeq \frac{G}{G-1}$. For two-way clustering (Eq. \ref{eq:twowayclu_centmat}), there are 2 ways of correction:
    \begin{itemize}
        \item choose correction terms for each of the 3 components: $$c_1= \frac{G}{G-1}\frac{N-1}{N-K}, c_2= \frac{H}{H-1}\frac{N-1}{N-K},c_3=\frac{I}{I-1}\frac{N-1}{N-K}$$ with $I$ being the number of unique clusters determined by $G\cap H$
        \item choose a constant terms for all components: $$c=\frac{J}{J-1}\frac{N-1}{N-K}$$ with $J=\min(G,H)$
    \end{itemize}
    \item \myhl[myblue]{\textbf{Var-cov matrix not positive-semidefinite}}: $\hat{\mathbb{V}}\left[\hat{\boldsymbol{\beta}}\right]$ might have negative elements on the diagonal (Eq. \ref{eq:twowayclu_decomp}), informly, this is more likely to  arise when clustering is done over the same groups as the fixed effects. One way to address this issue is using \textit{eigendecomposition} technique:
    $$
    \hat{\mathbb{V}}\left[\hat{\boldsymbol{\beta}}\right] = \mathbf{U}\boldsymbol{\Lambda}\mathbf{U}'
    $$
    where 
    \begin{itemize}
        \item $\mathbf{U}$ containing the eigenvectors of $\hat{\mathbf{V}}$
        \item $\boldsymbol{\Lambda} = \mathrm{diag}\left[\lambda_1,\cdots,\lambda_d\right]$ contains the eigenvalues of $\hat{\mathbf{V}}$ 
    \end{itemize}
    then create $\boldsymbol{\Lambda}^+ = \mathrm{diag} \left[\lambda_1^+,\cdots,\lambda_d^+\right]$ with $\lambda^+_j=\max\left(0,\lambda_j\right)$ and use $\hat{\mathbf{V}}^+\left[\hat{\boldsymbol{\beta}}\right]=\mathbf{U}\boldsymbol{\Lambda}^+\mathbf{U}'$ as the estimate
\end{itemize}

\section{Multiway Clustering}
\citet{cameron2011robust} extended the framework\footnote{Also proposed by \citet{thompson2011simple}.} to allow clustering in $D$ dimensions, then we can do the following reframing
\begin{itemize}
    \item $G_d$: the number of clusters in dimension $d\in \left\{ 1,2,\cdots,D \right\}$
    \item $D-$vector $\boldsymbol{\delta}_i = \boldsymbol{\delta}(i)$, with funciton $\boldsymbol{\delta}:\left\{1,2,\cdots,N\right\}\rightarrow \bigtimes ^D_{d=1}\left\{1,2,\cdots, G_d\right\}$ lists the cluster membership in each dimension of each observation
\end{itemize}
then we have 
$$
\mathbf{1}\left[i,j\text{ shares a clustser}\right] = 1 \Leftrightarrow \delta_{id}=\delta_{jd}
$$
for some $d\in \left\{1,2, \cdots, D\right\}$, where $\delta_{id}$ denotes the $d$th element of $\boldsymbol{\delta}_i$. Also 
\begin{itemize}
    \item $D-$vector $\mathbf{r}$: define the set $$ R\equiv \left\{ \mathbf{r}:r_d\in\left\{0,1\right\},d=1,2,\cdots,D,\mathbf{r\neq 0} \right\} $$
    elements of the set $R$ can be used to index all cases where 2 observations share a cluster in at least one dimension. Define the function 
    $$
    \mathbf{I_r} (i,j) \equiv \mathbf{1} \left[r_d\delta_{id} = r_d \delta_{jd},\forall d\right]
    $$
    which indicates whether observations $i$ and $j$ have identical cluster menbership for \myhl[myblue]{\textbf{all}} dimensions $d$ s.t. $r_d=1$.
    Then we have a \textit{aggregate} identifier
    $$
    \mathbf{I}(i,j) = 1 \Leftrightarrow \mathbf{I_r}(i,j)=1\text{ for some }\mathbf{r}\in R
    $$
    i.e., 2 observations share \myhl[myblue]{\textbf{at least}} one dimension.
\end{itemize}
The define the $2^D-1$ matrices
\begin{equation}\label{eq:centermat_multiclu}
    \tilde{\mathbf{B}}_{\mathbf{r}}\equiv \sum^N_{i=1}\sum^N_{j=1}\mathbf{x}_i\mathbf{x}_j' \hat{u}_i\hat{u}_j\mathbf{I_r}(i,j)
\end{equation}
with $\mathbf{r}\in R$.

\paragraph*{Var-cov matrix estimator} consider, similarly, an estimator 
\begin{equation}
    \hat{\mathbb{V}}\left[\hat{\boldsymbol{\beta}}\right] = \left(\mathbf{X'X}\right)^{-1}\tilde{\mathbf{B}} \left(\mathbf{X'X}\right)^{-1} \equiv  \left(\mathbf{X'X}\right)^{-1}  \left(\sum_{\left\Vert \mathbf{r} \right\Vert =k,\mathbf{r}\in R} (-1)^{k+1}\tilde{\mathbf{B}}_r \right)  \left(\mathbf{X'X}\right)^{-1}
\end{equation}
where cases of clustering on an odd number of dimensions are added, those of clustering on an even number of dimensions are subtracted. Consider the case of $D=3$,
\begin{equation*}
    \left(\tilde{\mathbf{B}}_{(1,0,0)} + \tilde{\mathbf{B}}_{(0,1,0)} + \tilde{\mathbf{B}}_{(0,0,1)} \right) - \left( \tilde{\mathbf{B}}_{(1,1,0)} + \tilde{\mathbf{B}}_{(1,0,1)} + \tilde{\mathbf{B}}_{(0,1,1)} \right) + \tilde{\mathbf{B}}_{(1,1,1)}
\end{equation*}
$\tilde{\mathbf{B}}$ is identical to $\hat{\mathbf{B}}$ defined analogically as in Eq.(\ref{eq:twowayclu_centmat}), since 
\begin{itemize}
    \item no observation pair with $\mathbf{I}(i,j)=0$: this is immediate, since $\mathbf{I}(i,j)=0 \Leftrightarrow \mathbf{I_r}(i,j)=0,\forall \mathbf{r}$
    \item the covariance term corresponding to each observation pair with $\mathbf{I}(i,j)=1$ is included \myhl[myblue]{\textbf{exactly once}} in $\tilde{\mathbf{B}} $: by inclusion-exclusion principle for set cardinality
    $$\mathbf{I}(i,j) \Rightarrow \sum_{\left\Vert \mathbf{r} \right\Vert=k,\mathbf{r}\in R}(-1)^{k+1}\mathbf{I_r}(i,j)=1 $$
\end{itemize} 

\paragraph*{Curse of dimensionality} this could arise in a setting with \textbf{many dimensions} of clustering, and in which one or more dimensions have \textbf{few} clusters\footnote{The square design (each dimension has the same number of clusters) with orthogonal dimensions has the \textbf{least} independence of observations.}.
\citet{cameron2011robust} suggested an ad-hoc rule of thumb for approximating sufficient numbers of clusters.

\subsection{Non-linear Estimators}
\paragraph*{$m$-Estimators}
Consider an $m$-estimator that solves
$$
\sum^N_{i=1}\mathbf{h}_i\left(\hat{\boldsymbol{\theta}}\right) = \mathbf{0}
$$
under standard assumptions, $\hat{\boldsymbol{\theta}}$ is asymptotically normal with estimated variance matrix 
\begin{equation}\label{eq:m-est_varmat}
    \hat{\mathbb{V}}\left[\hat{\boldsymbol{\theta}}\right] = \hat{\mathbf{A}}^{-1}\hat{\mathbf{B}}{\hat{\mathbf{A}^{\prime}}^{-1}}
\end{equation}
where $\hat{\mathbf{A}} = \sum_i \left. \frac{\partial \mathbf{h}_i}{\partial \boldsymbol{\theta}'}\right\vert _{\hat{\boldsymbol{\theta}}}$ and $\hat{\mathbf{B}}$ is an estimate of $\mathbb{V}\left[\sum_i\mathbf{h}_i\right]$.

\begin{itemize}
    \item \myhl[myblue]{\textbf{one-way clustering}} $\hat{\mathbf{B}} = \sum^G_{g=1}\hat{\mathbf{h}}_g\hat{\mathbf{h}}_g'$ where $\hat{\mathbf{h}}_g = \sum^{N_g}_{i=1}\hat{\mathbf{h}}_{ig}$, clustering may not lead to parameter inconsistency, depending on whether $\mathbb{E}\left[\mathbf{h}_i(\boldsymbol{\theta})\right]= \mathbf{0}$ with clustering
    \begin{itemize}
        \item \textbf{population-averaged approach}: assum $\mathbf{E}\left[y_{ig}\mid \mathbf{x}_{ig}\right] = \Phi \left( \mathbf{x}_{ig}'\boldsymbol{\beta} \right)$
        \item \textbf{random effects approach}: let $y_{ig}=1$ if $y^*_{ig} > 0$ where $y^*_{ig}=\mathbf{x}_{ig}'\boldsymbol{\beta}+\epsilon_g + \epsilon_{ig}$, where 
        \begin{itemize}
            \item idiosyncratic error $\epsilon_{ig}\sim \mathcal{N}(0,1)$
            \item cluster-specific error $\epsilon_g \sim \mathcal{N}(0,\sigma^2_g)$
        \end{itemize}
        then we have the alternative moment condition $$ \mathbb{E}\left[y_{ig}\mid \mathbf{x}_{ig}\right] = \Phi\left(\frac{\mathbf{x}_{ig}'\boldsymbol{\beta}}{\sqrt{1+\sigma^2_g}}\right) $$ 
    \end{itemize}
    \item \myhl[myblue]{\textbf{multiway clustering}} replacing $\hat{u}_i\mathbf{x}_i$ in Eq.(\ref{eq:centermat_multiclu}) with $\hat{\mathbf{h}}_i$, then we have 
    \begin{align*}
        \hat{\mathbb{V}} \left[\hat{\boldsymbol{\theta}}\right] &= \hat{\mathbf{A}}^{-1} \tilde{\mathbf{B}} \hat{\mathbf{A}'}^{-1}
    \end{align*}
    where 
    \begin{align*}
        \hat{\mathbf{A}} &= \sum_i \left. \frac{\partial \mathbf{h}_i}{\partial \boldsymbol{\theta}'}\right\vert _{\hat{\boldsymbol{\theta}}} & \tilde{\mathbf{B}} &= \sum_{\left\Vert \mathbf{r} \right\Vert =k,\mathbf{r}\in R} (-1)^{k+1}\tilde{\mathbf{B}}_r & \tilde{\mathbf{B}}_r &\equiv \sum^N_{i=1}\sum^N_{j=1}\hat{\mathbf{h}}_i\hat{\mathbf{h}'}_j\mathbb{I}_{\mathbf{r}}(i,j)
    \end{align*}
    with $\mathbf{r}\in R$\footnote{This multiway clustering can be implemented using several one-way clustered bootstraps. Each of the one-way cluster robust matrices is estimated by a pairs cluster bootstrap that resamples with replacement from the appropriate cluster dimension. They are then combined as if they had been estimated analytically \citep{cameron2011robust}.}.
\end{itemize} 

\paragraph*{GMM estimation} Consider an example of over-identified models: linear two stage least squares with more instruments than endogenous regressors, we have 
$$
\hat{\boldsymbol{\theta}} = \arg\min_{\boldsymbol{\theta}} Q(\boldsymbol{\theta}) =  \arg\min_{\boldsymbol{\theta}} \left(\sum^N_{i=1}\mathbf{h}_i(\boldsymbol{\theta})\right)'\mathbf{W}\left(\sum^N_{i=1}\mathbf{h}_i(\boldsymbol{\theta})\right)
$$
where $\mathbf{W}$ is a symmetric positive definite weighting matrix. Under standard regularity conditions, $\hat{\boldsymbol{\theta}}$ is asymptotically normal, with estimated variance matrix 
\begin{equation*}
    \hat{\mathbb{V}}\left[\hat{\boldsymbol{\theta}}\right] = \left(\hat{\mathbf{A}}'\mathbf{W}\hat{\mathbf{A}}\right)^{-1}\hat{\mathbf{A}}'\mathbf{W}\tilde{\mathbf{B}}\mathbf{W}\hat{\mathbf{A}}\left(\hat{\mathbf{A}}'\mathbf{W}\hat{\mathbf{A}}\right)^{-1}
\end{equation*}
again, $\hat{\mathbf{A}} = \sum_i \left. \frac{\partial \mathbf{h}_i}{\partial \boldsymbol{\theta}'}\right\vert _{\hat{\boldsymbol{\theta}}} $, and $\tilde{\mathbf{B}}$ is an estimate of $\mathbb{V}\left[\sum_i \mathbf{h}_i\right]$.

\section{Menzel (2021): Asymptotic Gaussianity}
One key of TWCR inference is the asymptotic Gaussianity, \citet{menzel2021bootstrap} pointed out the potential non-Gaussianity of the limit distribution.
Still, consider a random array ($Y_{it}$) indexed by two dimensions by $i=1,\cdots,N$ and $t=1,\cdots,T$. Clusters are sampled independently at random from an infinite population,
but otherwise \textbf{unrestricted} in dependence within each row $\mathbf{Y}_{i\cdot} \coloneq \left(Y_{i1}\cdots,Y_{iT}\right)$ and within each column $\mathbf{Y}_{\cdot t}\coloneq \left(Y_{1t},\cdots,Y_{Nt}\right)$.

\subsection{Distribution of Sample Average}
First, consider 
$$
\bar{Y}_{NT} \coloneq \frac{1}{NT}\sum^N_{i=1}\sum^T_{t=1}Y_{it}
$$
and approximate the asymptotic distribution regardless of whether, or what type of, cluster-dependence is present.

\paragraph*{3 scenarios} of the array $(Y_{it})$
\begin{itemize}
    \item \myhl[myblue]{\textbf{no cluster-dependence}}: $(Y_{it})$ are mutually independent, CLT at a rate of $(NT)^{-1/2}$ applies (under regularity conditions)
    \item \myhl[myblue]{\textbf{correlation within clusters}}: the convergence rate of $(Y_{it})$ is determined by the number of relevant clusters 
    \item \myhl[myblue]{\textbf{non-separable models of heterogeneity (dependence with clusters, even uncorrelated)}}\footnote{This is specific to clustering in 2 or more dimensions.}: The asymptotic behavior is non-standard
\end{itemize}
Consider 2 examples:
\begin{itemize}
    \item \myhl[myblue]{\textbf{Additive factor model}}
    $$ Y_{it} = \mu + \alpha_i + \gamma_t + \epsilon_{it} $$
    where $\mu$ is a constant, and $\alpha_i,\gamma_i,\epsilon_{it}$ are zero-mean i.i.d. random variables for $i=1,\cdots,N$ and $t=1,\cdots,T$ with bounded second moments, and $N=T$. Based on a standard central limit theory, we have 
    \begin{itemize}
        \item in the \underline{non-degenerate} case with $\mathrm{Var}(\alpha_i) >0$ or $\mathrm{\gamma_t}>0$, the sample distribution $$ \sqrt{N}\left(\bar{Y}_{NT}-\mathbb{E}\left[Y_{it}\right]\right) \xrightarrow{d} \mathcal{N}\left(0,\mathrm{Var}(\alpha_i) + \mathrm{Var}(\gamma_t)\right) $$
        \item in the \underline{degenerate} case of \underline{no clustering} with $\mathrm{Var}(\alpha_i) = \mathrm{Var}(\gamma_t) =0$, the sample distribution $$ \sqrt{NT}\left(\bar{Y}_{NT}-\mathbb{E}\left[Y_{it}\right]\right) \xrightarrow{d} \mathcal{N}\left(0,\mathrm{Var}(\epsilon_{it})\right) $$ 
    \end{itemize}
    if marginal distributions of $\alpha_i,\gamma_t,\epsilon_{it}$ are known, we can simulate from the joint distribution of $\left(Y_{it}\right)$ by sampling the individual components at random, a bootstrap procedure would be consistent. If \textbf{unknown}, consider estimators 
    \begin{align*}
        \hat{\alpha}_i & \coloneq \frac{1}{T}\sum^T_{t=1}\left(Y_{it}-\bar{Y}_{NT}\right) = \alpha_i + \frac{1}{T}\sum^T_{t=1} \left(\epsilon_{it}-\bar{\epsilon}_{NT}\right)\\
        \hat{\gamma}_t & \coloneq \frac{1}{N}\sum^N_{i=1}\left(Y_{it}-\bar{Y}_{NT}\right) = \gamma_t + \frac{1}{N}\sum^N_{i=1} \left(\epsilon_{it}-\bar{\epsilon}_{NT}\right)\\
        \hat{\epsilon}_{it} & \coloneq Y_{it} - \bar{Y}_{NT} -\hat{\alpha}_i -\hat{\gamma}_t
    \end{align*}
    then use these empirical distributions for estimation and form a bootstrap sample 
    $$ Y^*_{it} \coloneq \bar{Y}_{NT} + \alpha^*_i + \gamma^*_t + \epsilon^*_{it} $$
    by drawing from these estimators and obtain $\bar{Y}^*_{NT}\coloneq \frac{1}{NT}\sum^N_{i=1}\sum^T_{t=1}Y^*_{it}$, and verify the conditional variances of the bootstrap distribution given the sample:
    \begin{align*}
        \frac{1}{N}\sum^N_{i=1}\left(\hat{\alpha}_i - \frac{1}{N}\sum^N_{j=1}\hat{\alpha}_j\right)^2 - \left[\mathrm{Var}(\alpha_i) + \frac{\mathrm{Var}(\epsilon_{it})}{T}\right] & \xrightarrow{p} 0\\
        \frac{1}{T}\sum^N_{i=1}\left(\hat{\gamma}_t - \frac{1}{T}\sum^T_{s=1}\hat{\gamma}_t\right)^2 - \left[\mathrm{Var}(\gamma_t) + \frac{\mathrm{Var}(\epsilon_{it})}{N}\right] & \xrightarrow{p} 0
    \end{align*}
    then the bootstrap distribution is
    \begin{itemize}
        \item in the \underline{non-degenerate} case, $$ \sqrt{N}\left( \bar{Y}^*_{NT}-\bar{Y}_{NT} \right) \xrightarrow{d} \mathcal{N}\left(0,\mathrm{Var}(\alpha_i) + \mathrm{Var}(\gamma_t)\right) $$
        the estimation error $\hat{\alpha}_i$ does \textbf{NOT} affect the asymptotic variance.
        \item in the \underline{degenerate} case, $$ \sqrt{NT}\left(\bar{Y}^*_{NT}-\bar{Y}_{NT}\right) \xrightarrow{d} \mathcal{N}\left(0,3\mathrm{Var}(\epsilon_{it})\right) $$
        asymptotically overestimates the variance of the sampling distribution, leading to inconsistency of this naive bootstrapping procedure.
    \end{itemize}
    \item \myhl[myblue]{\textbf{Non-Gaussian limit distribution}} $$ Y_{it} = \alpha_i \gamma_t + \epsilon_{it} $$
    where $\alpha_i,\gamma_t,\epsilon_{it}$ are independently distributed with $\mathbb{E}\left[\epsilon_{it}\right] = 0,\mathrm{Var}(\alpha_i) = \sigma^2_{\alpha},\mathrm{Var}(\gamma_t) = \sigma^2_{\gamma}, \mathrm{Var}(\epsilon)_{it}=\sigma^2_{\epsilon}$. \textbf{If} $\mathbb{E}\left[\alpha_i\right] = \mathbb{E}\left[\gamma_t\right] = 0$, then CLT and Continuous Mapping Theorem (CMT) imply 
    \begin{align*}
        \sqrt{NT} \cdot \bar{Y}_{NT} =& \frac{1}{\sqrt{NT}} \sum^N_{i=1}\sum^T_{t=1} \left(\alpha_i \gamma_t +\epsilon_{it}\right)\\
        =& \left( \frac{1}{\sqrt{N}} \sum^N_{i=1}\alpha_i \right) \left(\frac{1}{\sqrt{T}} \sum^T_{t=1}\gamma_t \right) + \frac{1}{\sqrt{NT}}\sum^N_{i=1}\sum^T_{t=1}\epsilon_{it}\\
        \xrightarrow{d}& \sigma_{\alpha}\sigma_{\gamma}Z_1Z_2 + \sigma_{\epsilon}Z_3
    \end{align*}
\end{itemize}
then even without correlation within clusters, non-separable heterogeneity can still generate dependence in $2^{\text{nd}}$ or higher moments in the limiting distribution\footnote{2 major issues arise:\begin{itemize}
    \item The limiting distribution needs \textbf{not} be Gaussian: plug-in asymptotic inference based on the normal distribution is invalid 
    \item It only comes from two-or-more-dimension cluster dependence, not single-dimension cluster dependence.
\end{itemize}
}.

\subsection{Menzel (2021)'s Bootstrap procedure}\label{subsec:menzel_bootstrap}
\subsubsection{Notation}
For the array $\left(Y_{it}\right)_{i,t}$, denote 
\begin{itemize}
    \item $\mathbb{P}$: joint distribution of $\left(Y_{it}\right)_{i,t}$
    \item $\mathbb{P}_{NT}$: drifting DGP indexed by $N,T$
    \item $\mathbb{P}^*_{NT}$: bootstrap distribution for $\left(Y^*_{it}\right)$ given the realizations $\left(  Y_{it}:i=1,\cdots,N; t= 1,\cdots,T  \right)$
    \item respective distributions $\mathbb{E},\mathbb{E}_{NT},\mathbb{E}^*_{NT}$
\end{itemize}

\subsubsection{Inference: Sample Mean}
First, consider the assumption of \textit{separate exchangeability}
\begin{assumption}{Separate Exchangeability}{sep_exchange}
    \begin{itemize}
        \item A \textbf{separately exchangeable} array is an infinite array $\left(Y_{it}\right)_{i,t}$ such that for any integers $\tilde{N},\tilde{T}$ and permutations $\pi_1:\left\{1,\cdots,\tilde{N}\right\}\rightarrow \left\{1,\cdots,\tilde{N}\right\}$ and $\pi_2:\left\{1,\cdots,\tilde{T}\right\}\rightarrow \left\{1,\cdots,\tilde{T}\right\}$, we have 
        $$ \left(Y_{\pi_1(i),\pi_2(t)}\right)_{i,t} \overset{d}{=} \left(Y_{it}\right)_{i,t} $$
        such an array is called \textbf{dissociated} if for any $N_0,T_0\geq 1$, $\left(Y_{it}\right)^{i=N_0,t=T_0}_{i=1,t=1}$ is independent of $\left(Y_{it}\right)_{i>N_0,t>T_0}$.
        \item For dyadic data, consider the alternative assumption \textbf{jointly exchangeable} arrays $\left(Y_{ij}\right)_{i,j}$ satisfying 
        $$ \left(Y_{\pi(i),\pi(j)}\right)_{i,j} \overset{d}{=} \left(Y_{ij}\right)_{i,j} $$
        for any permutation $\pi$ on $\left\{1,\cdots,\tilde{N}\right\}$, in addition, $\left(Y_{ij}\right)^{N_0}_{i,j=1}$ is independent of $\left(Y_{ij}\right)_{i,j>N_0}$
    \end{itemize}
\end{assumption}
This assumption can be interpreted as rows (and columns) corresponding to units that are drawn independently from a common population, where we then observe the joint outcome for every row-column pair, consider the requirements in the following applications
\begin{itemize}
    \item \myhl[myblue]{\textbf{DiD/matched data}}: the units corresponding to either dimension of the sample to represent independent draws from a common, infinite population 
    \item \myhl[myblue]{\textbf{non-exhaustively matched data}}: only observe joint outcomes for a posibly self-selected subset of unit pairs, sample selection should be (jointly or separately) exchangeable
    \item \myhl[myblue]{\textbf{U-/V-statistics}}: the kernel $Y_{i_1,\cdots,i_D} \coloneq h\left(X_{i_1},\cdots,X_{i_D}\right) $ evaluated at i.i.d. observations $X_1,\cdots,X_N$ forms a dissociated, jointly exchangeable array
    \item \myhl[myblue]{\textbf{Network}}: unlabeled\footnote{\textit{Unlabeled}: nodel identifiers do not carry any significance for the statistical model.} data implies finite exchangeability, the sampled graph has joint (\textit{infinite}) exchangeability if it is a subgraph of an infinite graph
\end{itemize}
Directly from Assumption \ref{assump:sep_exchange}, any dissociated separately exchangeable array can be represented as 
\begin{equation*}
    Y_{it} = f\left(\alpha_i,\gamma_t,\epsilon_{it}\right)
\end{equation*}
for some function $f(\cdot)$ where $\alpha_1,\cdots,\alpha_N$, $\gamma_1,\cdots,\gamma_T$, $\epsilon_{11},\cdots,\epsilon_{NT}$ are mutually independent, uniformly distributed random variables.

\paragraph*{Projection} now, decompose the array $\left(Y_{it}\right)_{i,t}$ as 
\begin{align*}
    Y_{it} &= b + a_i + g_t + w_{it} & \mathbb{E}\left[w_{it}\mid a_i,g_t\right]&=0
\end{align*}
where $a_i$ and $g_t$ are mean-zero and mutually independent, s.t. the joint distribution of $Y_{it}$ can then be expanded as 
\begin{align*}
    Y_{it} =& \mathbb{E}\left[Y_{it}\right] + \left(\mathbb{E}\left[Y_{it}\mid \alpha_i\right] - \mathbb{E}\left[Y_{it}\right] \right) + \left( \mathbb{E}\left[Y_{it}\mid \gamma_t\right] - \mathbb{E}\left[Y_{it}\right] \right) \\
    &+ \left( \mathbb{E}\left[Y_{it}\mid \alpha_i,\gamma_t\right] - \mathbb{E}\left[Y_{it}\mid \alpha_i\right] - \mathbb{E}\left[Y_{it}\mid \gamma_t\right] + \mathbb{E}\left[Y_{it}\right]\right) + \left( Y_{it}-\mathbb{E}\left[Y_{it}\mid \alpha_i,\gamma_t\right] \right) \\
    \eqcolon & b+ a_i + g_t + v_{it} + e_{it}
\end{align*}
with
\begin{itemize}
    \item $e_{it} = Y_{it} - \mathbb{E}\left[Y_{it}\mid \alpha_i,\gamma_t\right]$
    \item $a_i = \mathbb{E}\left[Y_{it}\mid\alpha_i \right]-\mathbb{E}\left[Y_{it}\right]$, $g_t= \mathbb{E}\left[Y_{it}\mid \gamma_t\right]-\mathbb{E}\left[Y_{it}\right]$
    \item $v_{it} = \mathbb{E}\left[Y_{it}\mid \alpha_i,\gamma_t\right] - \mathbb{E}\left[Y_{it}\mid \alpha_i\right] - \mathbb{E}\left[Y_{it}\mid \gamma_t\right] + \mathbb{E}\left[Y_{it}\right]$
    \item $b= \mathbb{E}\left[Y_{it}\right]$
\end{itemize}
here, 
\begin{itemize}
    \item temporal and cross-sectional units were drawn independently: $a_1,\cdots,a_N$ and $g_1,\cdots,g_T$ are independent of each other.
    \item by construction, $\mathbb{E}\left[e_{it}\mid a_i,g_t,v_{it}\right]=0$, $\mathbb{E}\left[v_{it}\mid a_i\right] = \mathbb{E}\left[v_{it}\mid g_t\right]=0$
    \item $e_{it}$, $(a_i,g_t)$ and $v_{it}$ are \textbf{uncorrelated}
\end{itemize}
then, rewrite the sample mean as 
\begin{align*}
    \hat{Y}_{NT} &= b+\bar{a}_N + \bar{g}_T + \bar{v}_{NT} + \bar{e}_{NT}\\
    &\coloneq b + \frac{1}{N}\sum^N_{i=1}a_i + \frac{1}{T}\sum^T_{t=1}g_t + \frac{1}{NT}\sum^T_{t=1}\sum^N_{i=1}v_{it} + \frac{1}{NT}\sum^T_{t=1}\sum^N_{i=1}e_{it}
\end{align*}
and the unconditional variances of the projections with 
\begin{align*}
    \sigma^2_a&\coloneq \mathrm{Var}(a_i) & \sigma^2_g&\coloneq \mathrm{Var}(g_t) & \sigma^2_v&\coloneq \mathrm{Var}(v_{it}) & \sigma^2_e&\coloneq \mathrm{Var}(e_{it})   
\end{align*}
let $w_{it}\coloneq v_{it}+ e_{it}$, and denote its variance by $\sigma^2_w = \mathrm{Var}(w_{it})$. Then, assume integrability 
\begin{assumption}{Integrability}{integrability}
    Let $Y_{it} = f(\alpha_i,\gamma_t,\epsilon_{it})$, where $\alpha_i,\gamma_t,\epsilon_{it}$ are random arrays with elements i.i.d. drawn from $[0,1]$ uniform distribution, assume 
    \begin{itemize}
        \item $a_i/\sigma_a$, $g_t/\sigma_g$, $v_{it}/\sigma_v$, $e_{it}/\sigma_e$ are well-defined and have bounded moments up to the order $4+\delta$ for some $\delta>0$, whenever the respective variances $\sigma^2_a,\sigma^2_g,\sigma^2_v,\sigma^2_e$ are non-zero.
        \item $\sigma^2_a + \sigma^2_g >0$, or $\sigma^2_v + \sigma^2_e > 0$
    \end{itemize}
\end{assumption}

\paragraph*{Low-rank approximation}
Consider the row/column projection
\begin{equation*}
    \bar{v}_{NT} \equiv \frac{1}{NT}\sum^T_{t=1}\sum^N_{i=1}\left(\mathbb{E}\left[Y_{it}\mid \alpha_i,\gamma_t\right] -\mathbb{E}\left[Y_{it}\mid \alpha_i\right] -\mathbb{E}\left[Y_{it}\mid \gamma_t\right] +\mathbb{E}\left[Y_{it}\right] \right) \eqcolon \frac{1}{NT}\sum^T_{t=1}\sum^N_{i=1}v(\alpha_i,\gamma_t)
\end{equation*}
as a generalized U-statistic with a kerel $v(\alpha,\gamma)$ evaluated at the samples $\alpha_1,\cdots,\alpha_N$ and $\gamma_1,\cdots,\gamma_t$. There are 2 major issues w.r.t. characterizing the distribution of $\bar{Y}_{NT}$
\begin{itemize}
    \item the presence of the projection error $e_{it}$
    \item the factors $\alpha_i,\gamma_t$ are not observable
\end{itemize}
Define,
$$
v(\alpha,\gamma) \coloneq \mathbb{E}\left[Y_{it}\mid \alpha_i=\alpha,\gamma_t=\gamma\right] - \mathbb{E}\left[Y_{it}\mid \alpha_i=\alpha\right] - \mathbb{E}\left[Y_{it}\mid \gamma_t=\gamma\right] + \mathbb{E}\left[Y_{it}\right]
$$
under Assumption \ref{assump:integrability}, we have compact integral operators
\begin{align*}
    S(u)(g)&=\int v(a,g)u(a)F_{\alpha}(\mathrm{d}a) & S^*(u)(a)&=\int v(a,g)u(g)F_{\gamma}(\mathrm{d}g)
\end{align*}
where $F_{\alpha},F_{\gamma}$ are the marginal distributions corresponding to the joint $F_{\alpha\gamma}$ of $\alpha_i,\gamma_t$. Then the low-rank approximation is
\begin{equation}\label{eq:low-rank_approximation}
    v(\alpha,\gamma) = \sum^{\infty}_{k=1}c_k\phi_k(\alpha)\psi_k(\gamma)
\end{equation}
under the $L_2(F_{\alpha\gamma})$ norm on the space of smooth functions of $(\alpha,\gamma)\in[0,1]^2$. Here 
\begin{itemize}
    \item $(c_k)_{k\geq 1}$: a sequence of singular values, $\lim\left\vert c_k \right\vert \rightarrow 0$
    \item $\left(\phi_k\left(\cdot\right)\right)_{k\geq 1}$ and $\left(\psi_k\left(\cdot\right)\right)_{k\geq 1}$: orthonormal bases for $L_2\left([0,1],F_{\alpha}\right)$ and $L_2\left([0,1],F_{\gamma}\right)$:
    \begin{itemize}
        \item By construction: $$ \mathbb{E}\left[v(a,\gamma_t)\right] = \mathbb{E}\left[v(\alpha_i,g)\right] =0,\forall a,g\in[0,1] \Rightarrow \mathbb{E}\left[\phi_k(\alpha_i)\right] = \mathbb{E}\left[\psi_k(\gamma_t)\right] = 0,\forall k =1,2,\cdots $$
        \item the basis functions are orthonormal and $\alpha_i$ and $\gamma_t$ are independent, then $\forall K<\infty$ $$ \mathrm{Cov}\left[ \left(\phi_1(\alpha_i),\psi_1(\gamma_t), \cdots, \phi_K(\alpha_i),\psi_K(\gamma_t) \right) \right] $$ is the $2K$-dimensional identity matrix
        \item $\left(\phi_1(\alpha_i),\cdots, \phi_K(\alpha_i)\right)$ can be correlated with $a_i$: $\sigma_{ak}\coloneq \mathrm{Cov}\left(a_i,\phi_k(\alpha_i)\right)$
        \item $\left(\psi_1(\gamma_t),\cdots, \psi_K(\gamma_t)\right)$ can be correlated with $g_t$: $\sigma_{gk}\coloneq \mathrm{Cov}\left(g_t,\psi_k(\gamma_t)\right)$
    \end{itemize}
\end{itemize}
with this representation of Eq.(\ref{eq:low-rank_approximation}), we have\footnote{The limiting distribution of this term is not Gaussian, but can be represented as a linear combination of independent chi-squared random variables. This type of distributions is known as Wiener/Gaussian chaos.} 
\begin{equation*}
    \frac{1}{NT}\sum^N_{i=1}\sum^T_{t=1}v(\alpha_i,\gamma_t) = \sum^{\infty}_{k=1}c_k \left(\frac{1}{N}\sum^N_{i=1}\phi_k(\alpha_i)\right) \left(\frac{1}{T}\sum^T_{t=1}\psi_k(\gamma_t)\right)
\end{equation*}
and the second-order projection term can also be represented as a function of \textbf{countably many} sample averages of \textbf{i.i.d. mean-zero} random variables.

\begin{assumption}{Eigenfucntions and coefficients in the spectral represention (\ref{eq:low-rank_approximation})}{restrictions_on_lowrankapprox}
    The function $v(\alpha,\gamma)\coloneq \mathbb{E}\left[Y_{it}\mid \alpha_i=\alpha,\gamma_t=\gamma\right] - \mathbb{E}\left[Y_{it}\mid \alpha_i=\alpha\right] - \mathbb{E}\left[Y_{it}\mid \gamma_t=\gamma\right] + \mathbb{E}\left[Y_{it}\right]$ admits a spectral representation
    $$
    v(\alpha,\gamma) = \sum^{\infty}_{k=1}c_k\phi_k(\alpha)\psi_k(\gamma)
    $$
    under the $L_2(F_{\alpha\gamma})$ norm. And 
    \begin{itemize}
        \item the singular values are uniformly bounded by a square summable null sequence $\bar{c}_k$: $c_k\leq \bar{c}_k,\forall k=1,2,\cdots$, where $\sum^{\infty}_{k=1}c^2_k< \infty$
        \item $\forall k=1,2,\cdots$, the first 3 moments of the eigenfunctions $\phi_k(\alpha_i)$ and $\psi_k(\gamma_t)$ are bounded by a constant $B>0$
    \end{itemize}
\end{assumption}
To summarize the two assumptions
\begin{itemize}
    \item Assumption \ref{assump:sep_exchange} guarantees the pointwise consistency of the bootstrap
    \item Assumption \ref{assump:restrictions_on_lowrankapprox} gives the uniform consistency of the bootstrap: it imposes common bounds on moments and singular values and restricts the set of joint distribution $F$ to a \myhl[myblue]{\textbf{uniformity}} class\footnote{Here, the sequence $\mathbf{c}\coloneq \left(\bar(c)_k\right)_{k\geq 0}$ controls the magnitude of the error from a finite-dimensional approximation to $v(\alpha,\gamma)$.}.
\end{itemize}

\subsubsection{Bootstrap procedure}
For the sample mean $\bar{Y}_{NT} - \mathbb{E}\left[Y_{it}\right]$, the limiting distribution depends on the scale parameters:
\begin{itemize}
    \item If observations are independent across rows and columns: $\sqrt{NT}\left(\bar{Y}_{NT}-\mathbb{E}\left[Y_{it}\right]\right) \xrightarrow{d} \mathcal{N}\left(0,\sigma^2_e\right)$
    \item If $N=T$, within-cluster covariances are bounded from 0 in \textbf{at least one dimension}: $\sqrt{N}\left(\bar{Y}_{NT}-\mathbb{E}\left[Y_{it}\right]\right) \xrightarrow{d}\mathcal{N}\left(0,\sigma^2_a+\sigma^2_g\right)$
\end{itemize}
The bootstrap procedure should then be adaptive for both degenerate and non-degenerate cases. For the expansion
\begin{align}\label{eq:sample_mean_decomp}
    Y_{it} =& \mathbb{E}\left[Y_{it}\right] + \left(\mathbb{E}\left[Y_{it}\mid \alpha_i\right] - \mathbb{E}\left[Y_{it}\right] \right) + \left( \mathbb{E}\left[Y_{it}\mid \gamma_t\right] - \mathbb{E}\left[Y_{it}\right] \right) \\ \nonumber
    &+ \left( \mathbb{E}\left[Y_{it}\mid \alpha_i,\gamma_t\right] - \mathbb{E}\left[Y_{it}\mid \alpha_i\right] - \mathbb{E}\left[Y_{it}\mid \gamma_t\right] + \mathbb{E}\left[Y_{it}\right]\right) + \left( Y_{it}-\mathbb{E}\left[Y_{it}\mid \alpha_i,\gamma_t\right] \right) \\ \nonumber
    \eqcolon & b+ a_i + g_t + v_{it} + e_{it}
\end{align}
the sample analogs are:
\begin{align*}
    \hat{a}_i &\coloneq \frac{1}{T}\sum^T_{t=1}Y_{it}-\bar{Y}_{NT} & \hat{g}_t &\coloneq \frac{1}{N}\sum^N_{i=1}Y_{it} - \bar{Y}_{NT} & \hat{w}_{it}\coloneq & Y_{it} - \hat{a}_i-\hat{g}_t -\bar{Y}_{NT}
\end{align*}

\paragraph*{Evaluating bootstrap performance} it is crucial at what rates these estimators are consistent depending on the extent of clustering in the true DGP.
The variance of the projection terms are: 
\begin{align*}
    \mathrm{Var}\left(\hat{a}_i\right) &= \sigma^2_a + \frac{\sigma^2_w}{T} & \mathrm{Var}\left(\hat{g}_t\right) &= \sigma^2_g + \frac{\sigma^2_w}{N}
\end{align*}
s.t. the \textbf{convolution error} depending on $\sigma^2_w$ dominates in the degenerate case. Therefore, to correct for the contribution of the row/column averages of $w_{it}$, consider the scalar for the distribution of $\hat{a}_i,\hat{g}_t$ by 
\begin{align*}
    \lambda_a &= \frac{T\sigma^2_a}{T\sigma^2_a +\sigma^2_w} & \lambda_g &= \frac{N\sigma^2_g}{N\sigma^2_g +\sigma^2_w} 
\end{align*}

\paragraph*{Component variance estimator} let 
\begin{align*}
    \hat{s}^2_a &\coloneq \frac{1}{N-1}\sum^N_{i=1} \left(\hat{a}_i - \bar{Y}_{NT}\right)^2 \\
    \hat{s}^2_g &\coloneq \frac{1}{T-1}\sum^T_{t=1} \left(\hat{g}_t - \bar{Y}_{NT}\right)^2 \\
    \hat{s}^2_w &\coloneq \frac{1}{NT-N-T} \sum^N_{i=1}\sum^T_{t=1} \left(Y_{it}-\hat{a}_i-\hat{g}_t -\bar{Y}_{NT}\right)^2
\end{align*}
then form the estimators as 
\begin{align}
    \hat{\sigma}^2_a &= \max \left\{0,\hat{s}_a^2 -\frac{1}{T}\hat{s}^2_w\right\} & \hat{\sigma}_g^2 &=\max \left\{0,\hat{s}^2_g -\frac{1}{N}\hat{s}_w^2\right\} & \hat{\sigma}^2_w&\coloneq \hat{s}^2_w
\end{align}
the rates of convergence for these estimators are given in the following lemma:
\begin{lemma}{Stochastic Order of Variance Estimators}{varest_stcha_order}
    Under Assumption \ref{assump:sep_exchange},
    \begin{align*}
        \hat{\sigma}^2_a - \sigma^2_a &= O_p \left(\frac{1}{\sqrt{N}}\left(\sigma_a + \frac{\sigma_e}{\sqrt{T}}\right)^2 + \frac{\sigma^2_v}{T}\right) \\
        \hat{\sigma}^2_g - \sigma^2_g &= O_p \left(\frac{1}{\sqrt{T}}\left(\sigma_g + \frac{\sigma_e}{\sqrt{N}}\right)^2 + \frac{\sigma^2_v}{N}\right) \\
        \hat{\sigma}^2_w - \sigma^2_w &= O_p \left(\frac{\sigma^2_e}{\sqrt{NT}} + \left(\frac{1}{N} + \frac{1}{T}\right)\sigma^2_v \right)
    \end{align*}
    and there exist \textbf{no estimators} for $\sigma^2_a,\sigma^2_g,\sigma^2_w$ that converge at rates faster than these rates. Specifically, $\sigma^2_a$ can \textbf{NOT} be estimated at a rate faster than $T^{-1}$ even when $\sigma^2_a=0$\footnote{See the appendix of \citet{menzel2021bootstrap} for the proof.}.
\end{lemma}
Hence, a bootstrap procedure can use a consistent pre-test for the presence of cluster dependence in the \textbf{first moment}, with the model selectors 
\begin{align*}
    \hat{D}_a(\kappa) &\coloneq \mathbf{1} \left\{T\hat{\sigma}_a^2 \geq \kappa \right\} & \hat{D}_g(\kappa) &\coloneq \mathbf{1}\left\{N\hat{\sigma}_g^2 \geq \kappa \right\}
\end{align*}
$\forall \kappa \geq 0$. And for some $\kappa_a,\kappa_g$, let 
\begin{align*}
    \hat{\lambda}_a & \coloneq \frac{\hat{D}_a (\kappa_a)T\hat{\sigma}^2_a}{\hat{D}_a (\kappa_a)T\hat{\sigma}^2_a+\hat{\sigma}_w^2} & \hat{\lambda}_g & \coloneq \frac{\hat{D}_g (\kappa_g)T\hat{\sigma}^2_g}{\hat{D}_g (\kappa_g)N\hat{\sigma}^2_g + \hat{\sigma}_w}
\end{align*}
and estimate the asymptotic variance of the sample mean as 
\begin{equation}\label{eq:asymp_variance}
    \hat{S}^2_{NT,sel}\coloneq \hat{D}_a(\kappa_a)T\hat{\sigma}^2_a + \hat{D}_g (\kappa_g) N\hat{\sigma}^2_g + \hat{\sigma}^2_w
\end{equation}

\paragraph*{Bootstrap procedures}
\citet{menzel2021bootstrap} proposed the following resampling algorithm to estimate the sampling distribution for exhaustive sampling with cluster dependence in two dimensions 
\begin{algorithm}{Resampling Algorithm}{resampling_menzel}
    \begin{itemize}
        \item[\textbf{(a)}] For the $b$-th bootstrap iteration, draw 
        \begin{align*}
            a^*_{i,b} &\coloneq \hat{a}_{k^*_b(i)} & g^*_{t,b} &\coloneq \hat{g}_{s^*_b(t)}
        \end{align*}
        where $k^*_b(i)$ and $s^*_b(t)$ are i.i.d. draws from the discrete uniform distribution on the index sets $\left\{1,\cdots,N\right\}$ and $\left\{1,\cdots,T\right\}$ respectively
        \item[\textbf{(b)}] Generate $$ w^*_{it,b} \coloneq \omega_{1i,b}\omega_{2t,b}\hat{w}_{k^*_b(i)s^*_b(t)} $$ where $\omega_{1i,b},\omega_{2t,b}$ are i.i.d. random variables with $\mathbb{E}\left[\omega\right]=0$, $\mathbb{E}\left[\omega^2\right] = \mathbb{E}\left[\omega^3\right] = 1$\footnote{Typical choices of $\omega_{1i,b},\omega_{2t,b}$ are the Gamma distribution (with shape $=4$, scale $=1/2$).}
        \item[\textbf{(c)}] Generate a bootstrap sample of draws $$ Y^*_{it,b} = \bar{Y}_{NT} + \sqrt{\hat{\lambda}_a}a^*_{i,b} + \sqrt{\hat{\lambda}_g}g^*_{t,b} + w^*_{it,b} $$ and get the bootstrapped statistic $$ \bar{Y}^*_{NT,b} \coloneq \frac{1}{NT} \sum^N_{i=1}\sum^T_{t=1}Y^*_{it,b} $$ 
        \item[\textbf{(d)}] Repeat this procedure, get a sample of $B$ replications and approximate the conditional distribution of $\bar{Y}^*_{NT}$ given the sample with the empirical distribution over the bootstrap draws $\bar{Y}^*_{NT,1},\cdots,\bar{Y}^*_{NT,B}$ 
    \end{itemize}
\end{algorithm}
For the \textbf{pivotal boostrap}, the last step uses instead the empirical distribution of the studentized bootstrap draws to approximate the distribution of $$ \sqrt{NT}\left(\bar{Y}^*_{NT}-\bar{Y}_{NT}\right)/\hat{S}^*_{NT,sel} $$
where $\hat{S}^*_{NT,sel}$ is the bootstrap analog of the variance estimator $\hat{S}_{NT,sel}$.

\begin{definition}{Bootstrap Procedures}{menzel_bootstrap_procedure}
    Consider 3 versions of the bootstrap procedure based on \ref{algm:resampling_menzel}:
    \begin{itemize}
        \item \myhl[myred]{\textbf{BS-N}} (bootstrap \textit{without} model selection): apply steps (a) - (d), and set $\kappa_a = \kappa_g = 0$
        \item \myhl[myred]{\textbf{BS-S}} (bootstrap \textit{with} model selection): apply steps (a) - (d), and set $\kappa_a,\kappa_g$ according to increasing sequences $\kappa_g,\kappa_a\rightarrow \infty$ s.t. $\kappa_a/T\rightarrow 0$ and $\kappa_g/N\rightarrow 0$
        \item \myhl[myred]{\textbf{BS-C}} (\textit{conservative} bootstrap): addition to the settings of \myhl[myred]{\textbf{BS-S}}, set 
        \begin{align*}
            \hat{\lambda}_a \coloneq& \frac{\hat{q}_a}{\hat{q}_a+\hat{\sigma}^2_w}\frac{\hat{q}_a}{T\hat{\sigma}^2_a} & \hat{\lambda}_g &\coloneq \frac{\hat{q}_g}{\hat{q}_g+\hat{\sigma}^2_w}\frac{\hat{q}_g}{N\hat{\sigma}^2_g}
        \end{align*}
        where 
        \begin{align*}
            \hat{q}_a &\coloneq \max\left\{T\hat{\sigma}^2_a,\kappa_a\right\} & \hat{q}_g &\coloneq \max\left\{N\hat{\sigma}^2_g,\kappa_g\right\}
        \end{align*}
    \end{itemize}
\end{definition}

\paragraph*{Consistency of the bootstrap procedures}
\begin{itemize}
    \item {\textbf{BS-N}} (bootstrap \textit{with} model selection): \myhl[myblue]{\textbf{pointwise} consistent} in $\sigma^2_a,\sigma^2_g,\sigma^2_w$
    \item {\textbf{BS-S}} (bootstrap \textit{without} model selection): \myhl[myblue]{\textbf{uniformly} consistent} if the limiting distribution is Gaussian
    \item {\textbf{BS-C}} (\textit{conservative} bootstrap): \myhl[myblue]{\textbf{consistent}} in the nondegerate case $\sigma^2_a + \sigma^2_g >0$, but asymptotically \myhl[myblue]{\textbf{conservative}} for the degenerate cases
\end{itemize}
To establish the consistency, define the \textbf{adaptive rate} $r_{NT}$ as\footnote{Following Eq. (\ref{eq:sample_mean_decomp}), $\mathrm{Var}(\bar{Y}_{NT}) = \mathrm{Var}(b+\bar{a}_N +\bar{g}_T + \bar{v}_{NT} + \bar{e}_{NT})$.}
\begin{equation*}
    r^{-2}_{NT} \coloneq N^{-1}\sigma^2_a + T^{-1}\sigma^2_g + (NT)^{-1}\sigma^2_w \equiv \mathrm{Var}(\bar{Y}_{NT})
\end{equation*}
then consider the limiting distribution with the respective limits of normalized sequences:
\begin{align}\label{eq:normalized_variances}
    q_{a,NT} &\coloneq r^2_{NT}N^{-1}\sigma^2_a & q_{g,NT} &\coloneq r^2_{NT}T^{-1}\sigma^2_g & q_{e,NT} &\coloneq r^2_{NT}(NT)^{-1}\sigma^2_e & q_{v,NT} &\coloneq r^2_{NT}(NT)^{-1}\sigma^2_v \\
    & & q_{ak,NT} & \coloneq r^2_{NT}N^{-1}\sigma_{ak} & q_{gk,NT} &\coloneq r^2_{NT}T^{-1}\sigma_{gk}\nonumber
\end{align}
for $k=1,2,\cdots$. Let $\varrho_{NT} \coloneq r_{NT}\left(NT\right)^{-1/2}$, then 
$$
q_{a,NT} + q_{g,NT} + q_{e,NT} + q_{v,NT} = 1
$$
stacking the sequences as the vector
$$
\mathbf{q}_{NT} \coloneq \left( q_{e,NT},q_{a,NT},q_{g,NT}, q_{a1,NT},q_{g1,NT}, q_{a2,NT},q_{g2,NT},\cdots \right)
$$
and the singular values for the spectral decomposition (\ref{eq:low-rank_approximation}):
\begin{align*}
    \mathbf{c}_{NT} &\coloneq \left(c_{1,NT},c_{2,NT},\cdots\right) \in l^2 & \text{for }&\mathbb{E}_{NT}\left[Y_{it}\mid \alpha_i,\gamma_t\right]\\
    \mathbf{c} &\coloneq \left(c_1,c_2,\cdots\right) \in l^2 & \text{for }&\mathbb{E}\left[Y_{it}\mid \alpha_i,\gamma_t\right]
\end{align*}
then for convergent sequences $\mathbf{q}_{NT},\mathbf{c}_{NT},\mathbf{c}$, denote the limits 
\begin{align*}
    q_a &\coloneq \lim_{N,T}q_{a,NT} & q_g &\coloneq \lim_{N,T}q_{g,NT} & q_e & \coloneq \lim_{N,T}q_{e,NT} & q_v & \coloneq \lim_{N,T} q_{v,NT} \\
    \mathbf{q} & \coloneq \lim_{N,T}\mathbf{q}_{NT} & \mathbf{c} & \coloneq \lim_{N,T}\mathbf{c}_{NT} & \varrho &\coloneq \lim_{N,T}\varrho_{NT}
\end{align*}
for any fixed values of $\mathbf{q}, \mathbf{c},\varrho\in[0,1]$, define 
\begin{equation}\label{eq:convergence_law}
    \mathcal{L}_0\left(\mathbf{q},\mathbf{c},\varrho\right) \coloneq \left(\sqrt{q_e}Z^e  + \sqrt{q_a}Z^a + \sqrt{q_g}Z^g\right) + \varrho V
\end{equation}
where 
\begin{align*}
    V \coloneq \sum^{\infty}_{k=1}c_k Z^{\psi}_k Z^{\phi}_k
\end{align*}
and $Z^e,Z_k^{\psi},Z_k^{\phi}$ are i.i.d. standard normal random variables, $Z^a,Z_g$ are standard normal random variables with 
\begin{align*}
    \mathrm{Cov}\left(Z^a,Z_k^{\phi}\right) &= \frac{q_{ak}}{\sqrt{q_a}} & \mathrm{Cov}\left(Z^g,Z_k^{\psi}\right) &= \frac{q_{gk}}{\sqrt{q_g}} & \mathrm{Cov}\left(Z^a,Z^g\right) = \mathrm{Cov}\left(Z^a,Z_k^{\psi}\right) = \mathrm{Cov}\left(Z^g,Z_k^{\psi}\right) &=0
\end{align*}
Then, the CLT for sampling distribution is established as
\begin{theorem}{CLT for Sampling Distribution}{clt_samp_dist}
    Under Assumption \ref{assump:integrability},
    \begin{itemize}
        \item[(\textbf{a})] along \textit{any} convergent sequence $\mathbf{q}_{NT}\rightarrow \mathbf{q}$ and fixed $\mathbf{c} = \left(c_1,c_2,\cdots\right)$ , we have 
        \begin{equation*}
            \left\Vert \mathbb{P}\left(r_{NT}\left(\bar{Y}_{NT}-\mathbb{E}[Y_{it}]\right)\right) - \mathcal{L}_0 \left(\mathbf{q},\mathbf{c},\varrho \right) \right\Vert _{\infty} \rightarrow 0
        \end{equation*}
        where $\varrho \coloneq \lim_{N,T}\varrho_{NT}$, and $\left\Vert \cdot \right\Vert _{infty}$ denotes the Kolmogorov metric; the limiting distribution $\mathcal{L}_0\left(\mathbf{q},\mathbf{c},\varrho\right)$ is continuous\footnote{The convergence is pointwise w.r.t. the conditional mean function $\mathbb{E}\left[Y_{it} \mid \alpha_i=\alpha,\gamma_t=\gamma\right]$}.
        \item[(\textbf{b})] if in addition, Assumption \ref{assump:restrictions_on_lowrankapprox} holds, (\textbf{a}) is robust under drifting sequences $\mathbf{c}_{NT}\rightarrow \mathbf{c}$\footnote{The convergence is uniform within the class of distributions satisfying Assumption \ref{assump:restrictions_on_lowrankapprox}}
    \end{itemize}
\end{theorem}

\subparagraph*{Estimating the asymptotic distribution}
Lemma \ref{lemma:varest_stcha_order} establishes the consistency of the estimation for the components vairances $\sigma^2_a,\sigma^2_g,\sigma^2_w$, but are they \textbf{fast} enough?
\begin{proposition}{Estimability of Asymptotic Distribution}{estimability_asympdist}
    Let $\hat{\mathcal{L}}_{NT}$ denote an arbitrary estimator for $\mathcal{L}_0$ based on an array of size $N,T$ form the unknown distribution, then $\exists \delta>0$ s.t. 
    \begin{equation*}
        \liminf_{N,T\rightarrow\infty} \sup_{f\in\mathcal{F}} \mathbb{P}_{f,NT} \left( \left\Vert \hat{\mathcal{L}}_{NT} - \mathcal{L}_0 \left(\mathbf{q}_{NT}(f),\mathbf{c}_{NT}(f),\varrho_{NT}(f)\right) \right\Vert _{\infty} > \delta \right) >0
    \end{equation*}
    where
    \begin{itemize}
        \item $\mathcal{F}$: the class of functions $f(\alpha,\gamma,\epsilon)$ corresponding to distributions of $Y_{it}$ satisfying Assump. \ref{assump:integrability} and \ref{assump:restrictions_on_lowrankapprox}, for i.i.d. uniform $\alpha_i,\gamma_t,\epsilon_{it}$\footnote{From the Aldous-Hoover representation}
        \item $\mathbb{P}_{f,NT}\left(\cdot\right)$: probabilities for events w.r.t. an array of size $N,T$, generated according to $f$
        \item $\mathbf{q}_{NT}(f) \coloneq \left(q_{e,NT}(f),q_{a,NT}(f),\cdots\right)$: the vector of normalized variances from Eq. \ref{eq:normalized_variances}
    \end{itemize}
\end{proposition}
Proposition \ref{prop:estimability_asympdist} states that there exists \textbf{no estimator}
\footnote{Consider the counterexample for this impossibility: for the model $$Y_{it} = \alpha_i\gamma_t$$ where $\alpha_i,\gamma_t$ are mutually independent with i.i.d. factors $\alpha_i \sim \mathcal{N}(0,1),\gamma_t\sim\mathcal{N}(\mu_{\gamma},1)$. This model satisfies Assump.\ref{assump:integrability}, hence 
Thm.\ref{thm:clt_samp_dist} gives convergence results. However, for this model 
\begin{align*}
    a_i & \coloneq \mathbb{E}\left[Y_{it}\mid \alpha_i\right] = \alpha_i\mu_{\gamma} & g_t & \coloneq \mathbb{E}\left[Y_{it}\mid \gamma_t\right] = \gamma_t\mathbb{E}\left[\alpha_i\right]\equiv 0\\
    v_{it} & =\alpha_i(\gamma_t-\mu_{\gamma}) & \sigma^2_a&=\mu^2_{\gamma} & \sigma^2_v=1
\end{align*}
here, $\mu_{\gamma}$ can \textbf{not} be estimated from the original data at a rate faster than $T^{-1/2}$, the fastest possible rate at which $\mu_{\gamma}$ can be estimated from observing $\gamma_1,\cdots,\gamma_T$ directly. Therefore, no test can consistently distinguish the model $\mu_{\gamma}=0$ (asymptotic variance $\sigma^2_v$) from a drifting sequence $\tilde{\mu}_{T,\gamma}\coloneq T^{-1/2}m_{\gamma}$ (asymptotic variance $m^2_{\gamma}+\sigma^2_v$).}
for the asymptotic distribution that achieves consistency uniformly over the space of distributions satisfying Assumption \ref{assump:integrability} and \ref{assump:restrictions_on_lowrankapprox}: 
\begin{itemize}
    \item Under Theorem \ref{thm:clt_samp_dist}, the sample mean $\bar{Y}_{NT}$ converges to a continuous limiting distribution $\mathcal{L}_0(\mathbf{q,c},\varrho)$ along sequences $f_{NT}\in \mathcal{F}$ with proper limits for $\mathbf{q}_{NT},\mathbf{c}_{NT}$
\end{itemize}

\subparagraph*{Bootstrap Consistency}
Consider the bootstrap analog of $\hat{S}_{NT,sel}$ in Eq. \ref{eq:asymp_variance}
\begin{equation*}
    \hat{S}^{2*}_{NT,sel}\coloneq \hat{D}_a(\kappa_a)T\hat{\sigma}^{2*}_a + \hat{D}_g (\kappa_g) N\hat{\sigma}^{2*}_g + \hat{\sigma}^{2*}_w
\end{equation*}
where $\hat{D}_a(\kappa_a),\hat{D}_g(\kappa_g)$ are fixed at the sample values, $\kappa_a,\kappa_g$ are chosen according to whether the bootstrap is \textbf{with} or \textbf{without} model selection. Consider 2 versions based on the studentized sample mean:
\begin{itemize}
    \item \myhl[myblue]{\textbf{non-pivotal}} bootstrap: approximating the distribution of \textbf{the sample mean} $r_{NT}\left(\bar{Y}_{NT}-\mathbb{E}\left[Y_{it}\right]\right)$ with the bootstrap distribution $r_{NT}\left(\bar{Y}^*_{NT}-\bar{Y}_{NT}\right)$
    \item \myhl[myblue]{\textbf{pivotal}} bootstrap: approximating the distribtuion of the \textbf{studentized sample mean} $\frac{(NT)^{1/2}}{\hat{S}_{NT,sel}}\left(\bar{Y}_{NT}-\mathbb{E}\left[Y_{it}\right]\right)$ with the boostrap distribution $\frac{(NT)^{1/2}}{\hat{S}^*_{NT,sel}}\left(\bar{Y}^*_{NT}- \bar{Y}_{NT}\right)$
\end{itemize}
And we can establish the consistency
\begin{theorem}{Bootstrap Consistency}{bootstrap_consistency}
    Under Assumption \ref{assump:integrability}, 
    \begin{itemize}
        \item[(a)] the bootstrap \myhl[myblue]{\textbf{with model selection}} satisfies 
        \begin{equation}\label{eq:bsconsistency_clt}
            \left\Vert \mathbb{P}^*_{NT}\left( r_{NT} \left(\bar{Y}^*_{NT}-\bar{Y}_{NT}\right) \right) - \mathbb{P}_{NT}\left( r_{NT} \left(\bar{Y}_{NT}-\mathbb{E}\left[\bar{Y}_{it}\right]\right) \right) \right\Vert _{\infty} \xrightarrow{\text{a.s.}} 0
        \end{equation}
        and its pivotal analog
        \begin{equation}\label{eq:bsconsistency_piv}
            \left\Vert  \mathbb{P}^*_{NT} \left( \sqrt{NT} \frac{\bar{Y}^*_{NT}-\bar{Y}_{NT}}{\hat{S}^*_{NT,sel}} \right) - \mathbb{P}_{NT} \left( \sqrt{NT} \frac{\bar{Y}_{NT}-\mathbb{E}\left[Y_{it}\right]}{\hat{S}_{NT,sel}} \right)   \right\Vert _{\infty} \xrightarrow{\text{a.s.}} 0
        \end{equation}
        \textbf{pointwise} for any fixed $\sigma^2_a,\sigma^2_g,\sigma^2_e,\sigma^2_v$
        \item[(b)] the bootstrap \myhl[myblue]{\textbf{without model selection}} satisfies Eq.\ref{eq:bsconsistency_clt} and Eq.\ref{eq:bsconsistency_piv} \textbf{uniformly} if $q_v=0$
        \item[(c)] the \myhl[myblue]{\textbf{conservative}} bootstrap satisfies
        \begin{equation}\label{eq:bscons_consistency_clt}
            \left\Vert \mathbb{P}^*_{NT}\left( r_{NT} \left(\bar{Y}^*_{NT}-\bar{Y}_{NT}\right) \right) - \mathcal{L}_0\left(\bar{\mathbf{q}},\mathbf{c},\varrho\right) \right\Vert _{\infty} \xrightarrow{\text{p}} 0
        \end{equation}
        and its pivotal analog
        \begin{equation}\label{eq:bscons_consistency_piv}
            \left\Vert  \mathbb{P}^*_{NT} \left( \sqrt{NT} \frac{\bar{Y}^*_{NT}-\bar{Y}_{NT}}{\hat{S}^*_{NT,sel}} \right) - \mathcal{L}_0\left(\bar{\mathbf{q}},\mathbf{c},\varrho\right) \right\Vert _{\infty} \xrightarrow{\text{p}} 0
        \end{equation}
        uniformly over the \textbf{entire parameter space}, where $\bar{q}=\left(q_c,\bar{q}_a,\bar{q}_g,0,0,\cdots\right)$, with $\bar{q}_a\coloneq \max\left\{\kappa_a/T,q_a\right\}$ and $\bar{q}_g \coloneq \max\left\{\kappa_g/T,q_g\right\}$, which increases as $N,T\rightarrow \infty$.
    \end{itemize}
\end{theorem}
Theorem \ref{thm:clt_samp_dist} gives that 
\begin{itemize}
    \item \myhl[myblue]{\textbf{bootstrap with model selection}}: pointwise valid asymptotically
    \item \myhl[myblue]{\textbf{bootstrap without model selection}}: valid uniformly w.r.t. clustering in means, but \textbf{inconsistent} if $q_v>0$
    \item \myhl[myblue]{\textbf{conservative bootstrap}}: uniformly valid without any qualifications. In degenerate cases $\left(q_e+q_v>0\right)$, the scale of the estimated asymptotic distribution \textbf{diverges} at a rate $\kappa_a/T + \kappa_g/N$
\end{itemize}
Notice that $\mathcal{L}_0\left(\bar{\mathbf{q}},\mathbf{c},\varrho\right)$ in Thm. \ref{thm:bootstrap_consistency} is a mean-preserving spread of $\mathcal{L}_0\left( \mathbf{q,c},\varrho\right)$ in Thm. \ref{thm:clt_samp_dist}, hence estimates of percentiles from the conservative bootstrap are \textbf{biased outwards} away from 0,
leading to asymptotic conservative CIs.

\subparagraph*{Refinements} Using standard results on Edgeworth expansions, get
\begin{proposition}{Refinements}{bscons_refinement}
    Under Assumption \ref{assump:integrability} for any $0 < \delta < \infty$, and the distributions of $a_i$ and $g_t$ satisfy Cramer's condition\footnote{Cramer's condition states that $\mathbf{X}$ has a non-degenerate, absolutely continuous component.}
    $$ \limsup_{\left\Vert t \right\Vert \rightarrow \infty} \left\vert \mathbb{E}\left[\exp(i\mathbf{t'X})\right] \right\vert <1 $$
    then if $\sigma^2_a + \sigma^2_g \geq C$ for some $C>0$, we have 
    \begin{equation*}
        \left\Vert \mathbb{P}^*_{NT}\left(\sqrt{NT} \frac{\bar{Y}^*_{NT}-\bar{Y}_{NT}}{\hat{S}^*_{NT,sel}} - \mathbb{P}_{NT}\left(\sqrt{NT}\frac{\bar{Y}_{NT}-\mathbb{E}[Y_{it}]}{\hat{S}_{NT,sel}}\right) \right) \right\Vert _{\infty} = O_p\left(r^{-2}_{NT} \vee (NT)^{-1/2} \right)
    \end{equation*}
    for all three versions of the bootstrap.
\end{proposition}

\subsubsection{Inference in Regression Models}
Consider the linear projection model
\begin{equation}\label{eq:menzel_linearmodel}
    y_{it} = \mathbf{x}'_{it}\boldsymbol{\beta} + u_{it}
\end{equation}
with the dependent variable $y_{it}$ and the vector of $k$ regressors $\mathbf{x}_{it}\in \mathbb{R}^k$. Consider LS estimator 
\begin{equation*}
    \hat{\boldsymbol{\beta}}_{LS} \coloneq \left(\mathbf{X'X}\right)^{-1}\mathbf{X'y} = \boldsymbol{\beta} + \left(\mathbf{X'X}\right)^{-1}\left(\frac{1}{NT}\sum^N_{i=1}\sum^T_{t=1}\mathbf{x}_{it}u_{it}\right)
\end{equation*}
assume $\left(\mathbf{x}_{it}u_{it}\right)_{i,t}$ constitute a dissociated, separately exchangeable array, then we can have the Aldous-Hoover representation 
\begin{equation*}
    \mathbb{z}_{it} \coloneq \mathbf{x}_{it}u_{it} = f\left(\alpha_i,\gamma_t,\epsilon_{it}\right)
\end{equation*}
then denote
\begin{align*}
    \mathbf{a}_i & \coloneq \mathbb{E}\left[\mathbf{x}_{it}u_{it}\mid\alpha_i \right] & \mathbf{g}_t & \coloneq \mathbb{E}\left[\mathbf{x}_{it}u_{it}\mid\gamma_t\right] \\
    \mathbf{v}_{it} &\coloneq \mathbb{E}\left[\mathbf{x}_{it}u_{it}\mid\alpha_i,\gamma_t\right] - \mathbf{a}_i - \mathbf{g}_t & \mathbf{e}_{it} &\coloneq \mathbf{x}_{it}u_{it} - \mathbb{E}\left[\mathbf{x}_{it}u_{it}\mid \alpha_i,\gamma_t\right] \\
    \mathbf{w}_{it} & \coloneq \mathbf{x}_{it}u_{it} - \mathbf{a}_i -\mathbf{g}_t = \mathbf{v}_{it} + \mathbf{e}_{it}
\end{align*}
and the unconditional component variances as $\sigma^2_{al},\sigma^2_{gl},\sigma^2_{vl},\sigma^2_{el},\sigma^2_{wl}=\sigma^2_{vl} + \sigma^2_{el}$. The empirical analog of this decomposition is given by 
\begin{align*}
    \hat{\mathbf{a}}_i & \coloneq \frac{1}{T}\sum^T_{t=1}\mathbf{x}_{it}\hat{u}_{it} & \hat{\mathbf{g}_{t}} &\coloneq \frac{1}{N}\sum^N_{i=1}\mathbf{x}_{it}\hat{u}_{it} \\
    \hat{\mathbf{w}}_{it} & \coloneq \mathbf{x}_{it}\hat{u}_{it} - \hat{\mathbf{a}}_i -\hat{\mathbf{g}}_t
\end{align*}
for each $l=1,\cdots,k$, then construct
\subparagraph*{Bootstrap procedure for regression}
\begin{itemize}
    \item for the $b$th bootstrap iteration, draw $\mathbf{a}^*_{i,b}\coloneq \hat{\mathbf{a}}_{k^*_b(i)}$ and $\mathbf{g}^*_{t,b}\coloneq \hat{\mathbf{g}}_{s^*_b(i)}$ where $k^*_b(i)$ and $s^*_b(t)$ are i.i.d. draws from the discrete uniform distribution on the index sets $\left\{1,\cdots,N\right\}$ and $\left\{1,\cdots,T\right\}$, respectively
    \item generate $\mathbf{w}^*_{it,b} \coloneq \omega_{1i,b}\omega_{2t,b} \hat{\mathbf{w}}_{k^*_b(i)s^*_b(i)}$, where $\omega_{1i,b},\omega_{2t,b}$ are i.i.d. random variables with $\mathbb{E}\left[\omega\right] =0$, $\mathbb{E}[\omega^2]=\mathbb{E}[\omega^3]=1$
    \item simulate values of $\mathbf{z}^*_{it,b}=\left(z^*_{it1,b},\cdots,z^*_{itk,b}\right)'$, where the $l$th component is given by $$ z^*_{itl,b} \coloneq \sqrt{\hat{\lambda}_{al}}a^*_{il,b}+\sqrt{\hat{\lambda}_{gl}}g^*_{tl,b} +w^*_{itl,b}$$
    where the scalars are $\hat{\lambda}_{al}\coloneq \frac{\hat{D}_{al}(\kappa_a)T\hat{\sigma}^2_{al}}{\hat{D}_{al}(\kappa_a)T\hat{\sigma}^2_{al}+\hat{\sigma}^2_{wl}}$ and $\hat{\lambda}_{gl}\coloneq \frac{\hat{D}_{gl}(\kappa_g)N\hat{\sigma}^2_{gl}}{\hat{D}_{gl}(\kappa_g)N\hat{\sigma}^2_{gl}+\hat{\sigma}^2_{wl}}$
    \item then compute $$ \hat{\boldsymbol{\beta}}^*_{LS,b} = \hat{\boldsymbol{\beta}}_{LS} + \left(\mathbf{X'X}\right)^{-1}\left( \frac{1}{NT}\sum^N_{i=1}\sum^T_{t=1}\mathbf{z}^*_{it,b} \right) $$ for each bootstrap sample.
\end{itemize}
Next, we can approximate the asymptotic distribution of $r_{NT}\left(\hat{\boldsymbol{\beta}}_{LS}-\boldsymbol{\beta}\right)$ with the simulated distribution of $r_{NT}\left(\hat{\boldsymbol{\beta}}^*_{LS,b}-\hat{\boldsymbol{\beta}}_{LS}\right)$.
\begin{assumption}{Regression}{bootstrap_regression}
    Assume the model in Eq.(\ref{eq:menzel_linearmodel}) with $\mathbf{x}_{it}u_{it}=f(\alpha_i,\gamma_t,\epsilon_{it})$ and $\alpha_i,\gamma_t,\epsilon_{it}$ are i.i.d. uniform on $[0,1]$. And 
    \begin{itemize}
        \item[(a)] $\mathbf{X}$ has full column rank
        \item[(b)] $\forall l=1,\cdots,k$ and some $\delta>0$, the $(4+\delta)$th absolute moments of $x_{itl}$ are bounded, and the $(4+\delta)$th conditional moments of each component $\frac{a_{il}}{\sqrt{\mathrm{Var}\left(a_{il}\mid\mathbf{X}\right)}}$, $\frac{g_{tl}}{\sqrt{\mathrm{Var}\left(g_{tl}\mid\mathbf{X}\right)}}$, $\frac{v_{itl}}{\sqrt{\mathrm{Var}\left(v_{itl}\mid\mathbf{X}\right)}}$ and $\frac{e_{itl}}{\sqrt{\mathrm{Var}\left(e_{itl}\mid\mathbf{X}\right)}}$ given $\mathbf{X}$ are bounded whenever the conditional variance of either component is strictly positive.
        \item[(c)] unconditional variance: $\mathrm{Var}(a_{il}) + \mathrm{Var}(g_{tl})>0$ or $\mathrm{Var}\left(w_{itl}\right)>0$ for each $l=1,\cdots,k$.
        \item[(d)] For each component of $\mathbf{z}_{it}=\mathbf{x}_{it}u_{it}$, there exists a spectral representation satisfying Assumption \ref{assump:restrictions_on_lowrankapprox} 
    \end{itemize}
\end{assumption}
then analogous to the sample mean inference, we have 
\begin{proposition}{Regression Inference}{regression_inf}
    Under Assumption \ref{assump:bootstrap_regression}, then 
    \begin{itemize}
        \item $\hat{\boldsymbol{\beta}}_{LS}$ is consistent at the $r_{NT}$ rate 
        \item The bootstrap with model selection satisfies Eq.(\ref{eq:bsconsistency_clt}) and (\ref{eq:bsconsistency_piv}) pointwise as $\sigma^2_{al},\sigma^2_{gl},\sigma^2_{el},\sigma^2_{vl}$ are held fixed for all $l=1,\cdots,k$
        \item The bootstrap without mode selection satisfies Eq.(\ref{eq:bsconsistency_clt}) and (\ref{eq:bsconsistency_piv}) uniformly if $q_{vl}=0$ for all $l=1,\cdots,k$
        \item The conservative bootstrap satisfies Eq.(\ref{eq:bscons_consistency_clt}) and (\ref{eq:bscons_consistency_piv}) uniformly over the entire parameter space
    \end{itemize}
\end{proposition}

\subparagraph*{Asymptotic Gaussian of the LS estimator} for conditional asymptotic normality of bilinear forms $V_k\coloneq \mathbf{Z}_{1k}'\mathbf{XZ}_{2k}$ of random vectors $\mathbf{Z}_{1k},\mathbf{Z}_{2k}$ given the matrix $\mathbf{X}$. 
Under the conditions of this paper, $V_k$ is asymptotically Gaussian if $\ddot{\mathbf{x}}_{it},\ddot{\mathbf{x}}_{js}$ are mean-independent for any $(j,s)\neq (i,t)$\footnote{For difference-in-differences designs with a regressor $x_{it1}\coloneq\mathbf{1}\left\{t\geq T_i\right\}$ for unit-specific intervention date $T_i$, or when $\mathbf{x}_{it}\coloneq\mathbf{x}\left(\boldsymbol{\xi}_i,\boldsymbol{\zeta}_t\right)$ are a non-additive function of row- and column-level attributes $\boldsymbol{\xi}_i$ and $\boldsymbol{\zeta}_t$, respectively, these conditions need not hold in general.}.

\section{Latest Development}
\subsection{LLN and CLT for Exchangeable Arrays}
\citet{davezies2021empirical} establish uniform LLN and CLT to show consistency and asymptotic normality of \textbf{nonlinear} estimators under weak regularity conditions.

\subsubsection{Set up}
\paragraph*{Notations}
For any $A\subset \mathbb{R}$, $B\subset \mathbb{R}^k$ for some $k\geq 2$, then let 
\begin{align*}
    A^+ &= A \cap \left(0,\infty\right)\\
    \bar{B} &= \left\{ b=\left(b_1,\cdots,b_k\right)\in B: \forall (i,j)\in\left\{1,\cdots,k\right\}^2,i\neq j, b_i\neq b_j \right\}
\end{align*}
and let 
\begin{itemize}
    \item $\mathbb{I}_k = \overline{\mathbb{N}^{+k}}$ denote the set of $k$-tuples of $\mathbb{N}^+$ \textbf{without} repetition
    \item for any $n\in \mathbb{N}^+$, let $\mathbb{I}_{n,k} = \overline{\left\{1,\cdots,n\right\}^k}$
    \item for any $\mathbf{i}=\left(i_1,\cdots,i_k\right),\mathbf{j}=\left(j_1,\cdots,j_k\right)$ in $\mathbb{N}^k$, let $\mathbf{i}\odot \mathbf{j} = \left(i_1j_1,\cdots,i_kj_k\right)$, and denote the distinct elements of $\mathbf{i}$ as $\left\{\mathbf{i}\right\}$
    \item for any $r\in \left\{1,\cdots,k\right\}$, let $$ \mathcal{E}_r = \left\{ \left(e_1,\cdots,e_k\right) \in\left\{0,1\right\}^k:\sum^k_{j=1}e_j = r \right\} $$
    \item for any $A\subset \mathbb{N}^+$, let $\mathfrak{S}(A)$ denote the set of permutations on $A$, then for any $\mathbf{i}=(i_1,\cdots,i_k)\in\mathbb{N}^{+k}$ and $\pi \in \mathfrak{S}(\mathbb{N}^+)$, let $\pi(\mathbf{i}) = \left(\pi(i_1),\cdots,\pi(i_k)\right)$
\end{itemize}
\paragraph*{Polyadic data}
For random variables $Y_{\mathbf{i}}$ indexed by $\mathbf{i}\in\mathbb{I}_k$\footnote{Some examples are: $Y_{i_1,i_2}$ corresponds to export flows from country $i_1$ to $i_2$, or whether there is a link between node $i_1$ and $i_2$ in a network. $\left\{i_1,\cdots,i_k\right\}$ can also correspond to the different dimensions of clustering.}, it's assumed that the random variables are generated according to a \myhl[myblue]{\textbf{jointly exchangeable}} and \myhl[myblue]{\textbf{dissociated}} array: 
\begin{assumption}{Jointly Exchangeable and Dissociated Arrays}{jexch_diss_array}
    For any $\pi \in \mathfrak{S}\left(\mathbb{N}^+\right)$, $$ \left(Y_{\mathbf{i}}\right)_{\mathbf{i}\in\mathbb{I}_k} \overset{\mathrm{d}}{=} \left(Y_{\pi(\mathbf{i})}\right)_{\mathbf{i}\in\mathbb{I}_k} $$
    and for any disjoint subsets of $\mathbb{N}^+$, $A,B$, with $\min\left(\left\vert A\right\vert,\left\vert B\right\vert\right)\geq k$, $\left(Y_{\mathbf{i}}\right)_{\mathbf{i}\in\overline{A^k}}$ is \textbf{independent} of $\left(Y_{\mathbf{i}}\right)_{\mathbf{i}\in\overline{B^k}}$
\end{assumption}
The assumption implies that 
\begin{itemize}
    \item \textbf{\underline{jointly exchangeability}}: the joint distribution of the data remains identical under any possible permutation of labels, i.e., labeling conveys no information
    \item \textbf{\underline{dissociation}}: the variables are indepednent if they have \textbf{no unit} in common, that is $Y_{i_1,i_2}$ must be independent of $Y_{j_1,j_2}$ if $\left\{i_1,i_2\right\} \cap \left\{j_1,j_2\right\} = \varnothing$
\end{itemize}
the dependence structure under such assumptions are 
\begin{lemma}{Key Dependence Structure}{davezies_depstruct}
    Assumption \ref{assump:jexch_diss_array} holds \textbf{if and only if} there exists i.i.d. variables $\left(U_J\right)_{J\subset \mathbb{N}^+,1\leq \left\vert J \right\vert \leq k}$ and a measurable function $\tau$ s.t. almost surely
    $$
    Y_{\mathbf{i}} = \tau\left( \left(U_{\left\{\mathbf{i}\odot \mathbf{e}\right\}^+}\right) _{\mathbf{e}\in\bigcup^k_{r=1}\mathcal{E}_r} \right),\forall \mathbf{i}\in\mathbb{I}_k
    $$
    this result is referred to as the AHK representation (Aldous 1981, Hoover 1979, Kallenberg 1989)\footnote{Consider dyadic data ($k=2$), then for every $i_1<i_2$ (the ranking is precise), $ Y_{i_1,i_2} = \tau\left(U_{i_1},U_{i_2},U_{ \left\{i_1,i_2\right\} }\right) $, that is, the outcome $Y$ depends of factors specific to $i_1$ and $i_2$, and factors relating both.}.
\end{lemma}

\subsubsection{Uniform LLN and CLT}
Let $\mathcal{F}$ denote a class of real-valued functions admitting a first moment w.r.t. the distribution $P$, let $Pf$ denote the corresponding moment $\mathbb{E}\left[f(Y_{\mathbf{1}})\right]$, with $\mathbf{1}$ as the $k-$tuple $(1,\cdots,k)$. Assume that
\begin{assumption}{Measurability Assumption}{measurability_assumption}
    $\exists$ a countable subclass $\mathcal{G}\subset \mathcal{F}$ s.t. elements of $\mathcal{F}$ are pointwise limits of sequences of elements of $\mathcal{G}$
\end{assumption}
Consider
\begin{align*}
    \mathbb{P}_n f &= \frac{(n-k)!}{n!} \sum_{\mathbf{i}\in \mathbb{I}_{n,k}}f(Y_{\mathbf{i}}) \\
    \mathbb{G}_n f &= \sqrt{n}\left(\mathbb{P}_n f - Pf\right)
\end{align*}
and the restrictions on $\mathcal{F}$: for any $\eta>0$ and any seminorm $\left\Vert \cdot \right\Vert$ on a space containing $\mathcal{F}$, let
\begin{itemize}
    \item $N\left(\eta, \mathcal{F}, \left\Vert \cdot \right\Vert\right)$: the minimal number of $\left\Vert \cdot \right\Vert$-closed balls of radius $\eta$ with centers in $\mathcal{F}$ needed to cover $\mathcal{F}$
    \item $N_{[]}\left(\eta, \mathcal{F}, \left\Vert \cdot \right\Vert\right)$: the minimal number of $\eta-$brackets needed cover $\mathcal{F}$, where an $\eta-$bracket for $f\in\mathcal{F}$ is a pair of functions $\left(l,u\right)$ s.t. $l\leq f\leq u$ and $\left\Vert u-l \right\Vert<\eta$
\end{itemize}
\citet{davezies2021empirical} considered the seminorms $\left\Vert f \right\Vert _{\mu,r}=\left(\int \left\vert f \right\vert^t \mathrm{d}\mu\right)^{1/r}$ for any $r\geq 1$ and probability measure (cdf) $\mu$. An envelope of $\mathcal{F}$ is measurable function $F$ satisfying $F(u) \geq \sup_{f\in\mathcal{F}}\left\vert f(u) \right\vert$, satisfying 
\begin{assumption}{Assumptions of $\mathcal{F}$}{assumptions_on_f}
    \begin{itemize}
        \item[\textbf{A}] The class $\mathcal{F}$  
        \begin{itemize}
            \item[(i)] either admits an envelope $F$ with $PF<\infty$ and $\forall \eta >0$, $$ \sup_{Q\in\mathcal{Q}}N\left( \eta\left\Vert F \right\Vert _{Q,1},\mathcal{F},\left\Vert \cdot \right\Vert _{Q,1} \right) <\infty $$ 
            \item[(ii)] or satisfies $N_{[]}\left(\eta,\mathcal{F},\left\Vert\cdot \right\Vert _{L_1\left(P\right)}\right)<\infty$ for all $\eta>0$ 
        \end{itemize}
        \item[\textbf{B}] and it 
        \begin{itemize}
            \item[(i)] \myhl[myblue]{\textbf{uniform entropy integral}}: either admits an envelope $F$ with $PF^2<\infty$ and $$ \int^{\infty}_0 \sup_{Q\in\mathcal{Q}} \sqrt{\log N\left(\eta \left\Vert F \right\Vert _{Q,2},\mathcal{F},\left\Vert \cdot \right\Vert _{Q,2} \right)} \mathrm{d}\eta <\infty $$
            \item[(ii)] \myhl[myblue]{\textbf{bracketing entropy integral}}: or satisfies $\int^{\infty}_0 \sqrt{\log N_{[]} \left(\eta,\mathcal{F},\left\Vert \cdot \right\Vert _{L_2(P)}\right)} \mathrm{d}\eta < \infty$ 
        \end{itemize}
    \end{itemize}
\end{assumption}
Assumption \ref{assump:assumptions_on_f} are same as the conditions imposed on i.i.d. data for uniform LLNs and CLTs. Under these assumptions, \citet{davezies2021empirical} established the uniform LLNs and CLTs as 
\begin{theorem}{Uniform LLNs and CLTs}{uniform_lln_clt}
    Under Assumption \ref{assump:jexch_diss_array} and \ref{assump:measurability_assumption},
    \begin{itemize}
        \item if (\textbf{A}) of Assumption \ref{assump:assumptions_on_f} holds, $\sup_{f\in\mathcal{F}} \left\vert \mathbb{P}_nf - Pf \right\vert \xrightarrow{\mathbf{a.s.}} 0$ and in $L^1$
        \item if (\textbf{B}) of Assumption \ref{assump:assumptions_on_f} holds, $\mathbb{G}_n$ converges weakly in $l^{\infty}(\mathcal{F})$ to a centered Gaussian process $\mathbb{G}$ on $\mathcal{F}$ as $n\rightarrow \infty$, the convariance kernel $K$ of $\mathbb{G}$ satisfies 
        $$
        K(f_1,f_2) = \frac{1}{(k-1)!^2} \sum_{\left(\pi,\pi'\right) \in \mathfrak{S}\left(\left\{\mathbf{1}\right\}\right)\times \mathfrak{S}\left(\left\{\mathbf{1}'\right\}\right) }\mathrm{Cov}\left( f_1 \left(\mathbf{Y}_{\pi(\mathbf{1})}\right), f_2 \left(\mathbf{Y}_{\pi '(\mathbf{1}')}\right) \right)
        $$
    \end{itemize}
\end{theorem}
Here, (\textbf{A}) of Assumption \ref{assump:assumptions_on_f} is stronger than necessary to obtain the uniform LLNs. To establish the exact characterization, consider the norms:
\begin{align*}
    \left\Vert f \right\Vert _{1,1} &= \frac{1}{n} \sum^n_{i_1=1} \left\vert \frac{1}{n-1} \sum_{i_2 \neq i_1} f(Y_{i_1,i_2}) + f(Y_{i_2,i_1}) \right\vert \\
    \left\Vert f \right\Vert _{1,2} &= \frac{1}{n(n-1)} \sum_{1\leq i_1<i_2 \leq n} \left\vert \mathbb{E}\left[  f(Y_{i_1,i_2}) + f(Y_{i_2,i_1}) \right] \mid U_{\left\{ i_1,i_2 \right\}} \right\vert
\end{align*}
and the exact characterization is established as 
\begin{proposition}{Exact Characterization of Uniform LLNs}{uni_lln_exactchar}
    Under Assumption \ref{assump:jexch_diss_array} and \ref{assump:measurability_assumption}, and $\mathcal{F}$ admits an envelop $F$ with $PF<\infty$, then $$\sup_{f\in\mathcal{F}} \left\vert \mathbb{P}_n f -Pf \right\vert \xrightarrow{\mathrm{a.s.}} 0$$ \textbf{if and only if} both $\log N\left(\epsilon, \mathcal{F}, \left\Vert \cdot \right\Vert _{1,2}\right)/n^2$ and $\log N\left(\epsilon, \mathcal{F}, \left\Vert \cdot \right\Vert _{1,1}\right)/n$ tend to 0 in outer probability.
\end{proposition}
and 2 aspects of dissociated, exchangeable arrays are emphasized:
\begin{itemize}
    \item \textbf{i.i.d. variations}: through the random entropy term related to $\left\Vert \cdot \right\Vert _{1,2}$, which only involves $\left(U_{\left\{i_1,i_2\right\}}\right)_{\mathbf{i}\in\mathbb{I}_{n,2}}$
    \item \textbf{U-statistic}: through the random entropy term related to $\left\Vert \cdot \right\Vert _{1,1}$, up to negligible terms, $\left\Vert f \right\Vert _{1,1}$ only depends on $\left(U_{i_1}\right)_{1\leq i_1 \leq n}$
\end{itemize}

\subsubsection{Convergence of the bootstrap process} \citet{davezies2021empirical}, extending the pigeonhole bootstrap \citep{mccullagh2000resampling,owen2007pigeonhole}, established the following bootstrap process:
\begin{itemize}
    \item[1] $n$ units are sampled independently in $\left\{1,\cdots,n\right\}$ with replacement and equal probability, $W_i$ denotes the number of times unit $i$ is sampled.
    \item[2] the $k-$tuple $\mathbf{i}= \left(i_1,\cdots,i_k\right)\in\mathbb{I}_{n,k}$ is then selected $W_i=\prod^k_{j=1}W_{i_j}$ times in the bootstrap sample 
\end{itemize}
then consider $\mathbb{P}^*_n$ and $\mathbb{G}^*_n$ defined on $\mathcal{F}$ by 
\begin{align*}
    \mathbb{P}^*_n f &= \frac{(n-k)!}{n!}\sum_{\mathbf{i}\in\mathbb{I}_{n,k}}W_{\mathbf{i}} f(Y_{\mathbf{i}})\\
    \mathbb{G}^*_n f &= \sqrt{n}\left(\mathbb{P}^*_{n} f - \mathbb{P}_n f\right)
\end{align*}
the validity of the bootstrap is then established as:
\begin{theorem}{Bootstrapping Validity}{bootstrap_validity}
    Under Assumption \ref{assump:jexch_diss_array} and \ref{assump:measurability_assumption}, if (\textbf{B-i}) of Assumption \ref{assump:assumptions_on_f} also holds, the process $\mathbb{G}^*_n$ converges weakly in $l^{\infty}(\mathcal{F})$ to $\mathbb{G}$, conditional on $\left(Y_{\mathbf{i}}\right)_{\mathbf{i}\in\mathbb{I}_k}$ and outer almost surely.
\end{theorem}
The proof of this theorem boils down to proving
$$
\sup_{h\in BL_1} \left\vert \mathbb{E}\left(h\left(\mathbb{G}^*_n\right) \mid \left(Y_{\mathbf{i}}\right)_{\mathbf{i}\in \mathbb{I}_k}\right)  - \mathbb{E}\left(h(\mathbb{G})\right) \right\vert \xrightarrow{\mathrm{a.s.}_{outer}} 0 
$$
where $BL_1$ is the set of bounded and Lipschitz functions from $l^{\infty}(\mathcal{F})$ to $[0,1]$. With the standard bootstrap for i.i.d. data
$$
\mathbb{E} \left[\mathbb{P}^*_n(f)\mid \left(Y_{\mathbf{i}}\right)_{\mathbf{i}\in\mathbb{I}_k}\right] = \frac{1}{n^k} \sum_{\mathbf{i}\in\mathbb{I}_{n,k}} f(Y_{\mathbf{i}}) \xrightarrow{n\rightarrow \infty} \mathbb{P}_n f
$$
hence, \citet{davezies2021empirical} established the a.s. conditional convergence of $\sqrt{n}\left( \mathbb{P}^*_n f - \frac{1}{n^k} \sum_{\mathbf{i}\in\mathbb{I}_{n,k}} f(Y_{\mathbf{i}}) \right)$.

\subsubsection{Nonlinear estimators}
\citet{davezies2021empirical} considered 2 classes of estimators: \textbf{Z-estimators} and \textbf{smooth functionals of the empirical cdf}:
\paragraph*{Z-estimators} 
Let 
\begin{itemize}
    \item $\Theta$ denote a normed space, endowed with norm $\left\Vert \cdot \right\Vert _{\Theta}$
    \item $\left(\psi_{\theta,h}\right)_{(\theta,h)\in\Theta\times \mathcal{H}}$ denote a class of real, measurable functions
    \item $\Psi(\theta)(h)=P\psi_{\theta,h}$, $\Psi_n(\theta)(h) = \mathbb{P}_n \psi_{\theta,h}$, $\Psi^*_n(\theta)(h)=\mathbb{P}^*_n\psi_{\theta,h}$
    \item for any real function $g$ on $\mathcal{H}$, $\left\Vert g \right\Vert _{\mathcal{H}} = \sup_{h\in\mathcal{H}}\left\vert g(h) \right\vert$
\end{itemize}
The parameter of interest $\theta_0$, satisfying $\Psi(\theta_0) = 0$, is estimated by $\hat{\theta} = \arg\min_{\theta\in\Theta} \left\Vert \Psi_n(\theta) \right\Vert _{\mathcal{H}}$ , define the bootstrap counterpart of $\hat{\theta}$ as $$ \hat{\theta}^* = \arg\min_{\theta\in\Theta} \left\Vert \Psi^*_n\left(\theta\right) \right\Vert _{\mathcal{H}} $$
then we have the convergence 
\begin{theorem}{Convergence of Z-estimators Bootstrap}{z_est_boot_conv}
    Under Assumption \ref{assump:jexch_diss_array}, if also
    \begin{itemize}
        \item[1] $\left\Vert \Psi(\theta_m) \right\Vert _{\mathcal{H}} \rightarrow 0 \Rightarrow \left\Vert \theta_m-\theta_0 \right\Vert _{\Theta} \rightarrow 0$, $\forall (\theta_m)_{m\in\mathbb{N}}\in \Theta$ 
        \item[2] $\left\{\psi_{\theta,h}:(\theta,h) \in \Theta \times \mathcal{H}\right\}$ satisfies Assumption \ref{assump:measurability_assumption} and (A) of \ref{assump:assumptions_on_f}, with $PF< \infty$
        \item[3] $\exists\delta>0$ s.t. $\left\{\psi_{\theta,h}:\left\Vert\theta-\theta_0\right\Vert _{\Theta}<\delta,h\in\mathcal{H}\right\}$ satisfies Assumption \ref{assump:measurability_assumption} and (B) of \ref{assump:assumptions_on_f}, with $P F^2_{\delta} < \infty$
        \item[4] $\lim_{\theta\rightarrow \theta_0}\sup_{h\in\mathcal{H}} P\left(\psi_{\theta,h}-\psi_{\theta_0,h}\right)^2 =0$
        \item[5] $\forall \eta>0$, $\left\Vert \Psi_n \left(\hat{\theta}\right) \right\Vert _{\mathcal{H}} = o_p(n^{-1/2}) $ and $P\left( \left\Vert \sqrt{n} \Psi^*_n\left(\hat{\theta}^*\right) \right\Vert _{\mathcal{H}} > \eta\left\vert \left(Y_\mathbf{i}\right)_{\mathbf{i}\in\mathbb{I}_k} \right\vert \right) = o_p(1)$
        \item[6] $\theta\mapsto \Psi(\theta)$ is Frechet-differentiable at $\theta_0$, with continuously invertible derivative $\dot{\Psi}_{\theta_0}$
    \end{itemize}
    Then the convergence can be established as
    \begin{itemize}
        \item $\sqrt{n}\left(\hat{\theta}-\theta\right)$ converges in distribution to a centered Gaussian process $\mathbb{G}$
        \item conditional on $\left(Y_{\mathbf{i}}\right)_{\mathbf{i}\in\mathbb{I}_k}$, $\sqrt{n}\left(\hat{\theta}^*-\hat{\theta}\right) \xrightarrow{d} \mathbb{G}$ almost surely
    \end{itemize}
\end{theorem}

\paragraph*{Smooth functionals of $F_Y$} For the cdf of $Y_{\mathbf{i}}$, suppose that $\mathcal{Y}\subset \mathbb{R}^p$ for some $p\in\mathbb{N}^+$ and $\theta_0 = g(F_Y)$, where $g$ is Hadamard differentiable\footnote{No linearity assumed under Hadamard differentiability.}. Estimate $\theta_0$ with $\hat{\theta}=g\left(\hat{F}_Y\right)$, where $\hat{F}_Y$ denotes the empirical cdf of $\left(Y_{\mathbf{i}}\right)_{\mathbf{i}\in\mathbb{I}_{n,k}}$, and let $\hat{\theta}^*$ denote the bootstrap counterpart of $\hat{\theta}$.
\citet{davezies2021empirical} established the convergence results as 
\begin{theorem}{Convergence of Smooth Functionals of the Empirical CDF}{conv_smoothcdf}
    Suppose that $g$ is Hadamard differentiable at $F_Y$ tangentially to a set $\mathbb{D}_0$, with derivative equal to $g'_{F_Y}$. Under Assumption \ref{assump:jexch_diss_array},
    \begin{itemize}
        \item $\sqrt{n}\left( \hat{F}_Y-F_Y \right)$ converges weakly, as a process indexed by $y$, to a Gaussian process $\mathbb{G}$ with kernel $K$ satisfying 
        $$
        K\left(y_1,y_2\right) = \frac{1}{(k-1)!^2}\sum_{\left(\pi,\pi'\right)\in \mathfrak{S}\left(\left\{\mathbf{1}\right\}\right) \times \mathfrak{S}\left(\left\{\mathbf{1}'\right\}\right)}\mathrm{Cov}\left( \mathbf{1}_{\left\{Y_{\pi(\mathbf{1})\leq y_1}\right\}},\mathbf{1}_{\left\{Y_{\pi '(\mathbf{1}')\leq y_2}\right\}} \right)
        $$
        \item If $\mathbb{G}\in \mathbb{D}_0$\footnote{In practice, $\mathbb{D}_0$ often corresponds to the set of functions that are continuous everywhere or at a certain point $y_0$.} with probability 1, $$ \sqrt{n}\left(\hat{\theta}-\theta_0\right) \xrightarrow{\mathrm{d}} \mathcal{N}\left(0, \mathbb{V}\left(g'_{F_Y}\left(\mathbb{G}\right)\right)\right) $$
    \end{itemize}
    conditional on $\left(Y_{\mathbf{i}}\right)_{\mathbf{i}\in\mathbb{I}_k}$, $\sqrt{n}\left(\hat{\theta}^*-\hat{\theta}\right)\xrightarrow{\mathrm{d}} \mathcal{N}\left(0, \mathbb{V}\left(g'_{F_Y}\left(\mathbb{G}\right)\right)\right)$, almost surely.
\end{theorem}

\subsubsection{Extensions}
\citet{davezies2021empirical} also considered several extensions of the main results:
\begin{itemize}
    \item \myhl[myblue]{\textbf{Degenerate cases}}: consider the simple $k=2$ situations where $K(f,f)=0, \forall f\in\mathcal{F}$. 
    Generally, when $K(f,f)=0$, the rate of convergence of $\mathbb{P}_nf -Pf$ is $n^{-1}$ rather than $n^{-1/2}$, and the asymptotic distribution is not necessarily normal. $\forall (i_1,i_2)\in\mathbb{I}_2$, let $Y_{i_1,i_2}=\tau\left( U_{i_1},U_{i_2},U_{\left\{i_1,i_2\right\}} \right)$ be the Aldous-Hoover-Kallenberg representation. WLoG, assume $U_{\cdot}$ to be uniform on $[0,1]$.
    %\footnote{This degeneracy appears if the variables in the array are actually \textbf{i.i.d.} in which case $\sqrt{n}\mathbb{G}_n$ converges to a Guassian process with covariance kernel $K(f_1,f_2)=\mathrm{Cov}\left(f_1(Y_{1,2}),f_2(Y_{1,2})\right)$. \citet{menzel2021bootstrap} assumes a special case where $Y_{i_1,i_2}=X_{i_1}X_{i_2}$, with $\left(X_i\right)_{i\in\mathbb{N}^+}$ i.i.d. with $\mathbb{E}\left(X_1\right)=0$, $\mathbb{V}\left(X_1\right)=1$, also $\mathcal{F}=\left\{ f_{\lambda}(x)=\lambda x,\lambda\in I \right\}$ for a compact $I\subset \mathbb{R}$. In this case, $\sqrt{n}\mathbb{G}_n$ converges weakly in $l^{\infty}\left(\mathcal{F}\right)$ to $\mathbb{G}\left(f_{\lambda}\right)=\lambda\left(Z^2-1\right)$ with a standard normal $Z$.}
    
    Under a more stringent version of (\textbf{B}-i) Assumption \ref{assump:assumptions_on_f}, that is, $\mathcal{F}$ admits an envelope $F$ with $PF^2< \infty$ and 
    $$
    \int^{\infty}_0 \sup_{Q\in\mathcal{Q}} \log N\left(\eta \left\Vert F \right\Vert _{Q,2},\mathcal{F}, \left\Vert \cdot \right\Vert _{Q,2}\right)\mathrm{d}\eta <\infty
    $$
    \citet{davezies2021empirical} showed that for $k=2$, $K(f,f)=0$ for all $f\in\mathcal{F}$, $\sqrt{n}\mathbf{G}_n$ converges \textbf{weakly} in $l^{\infty}(\mathcal{F})$ to $\mathbb{G}^d$. However, the proposed bootstrap process does \textbf{not} generally converge to $\mathbb{G}^d$.
    \item \myhl[myblue]{\textbf{An alternative bootstrap process}}: \citet{davezies2021empirical} established that under Assumption \ref{assump:jexch_diss_array} and \ref{assump:measurability_assumption}, $Pf^2<\infty$ for all $f\in \mathcal{F}$ and $\mathcal{F}$ admits an envelope $F$ s.t. $PF^{1+\delta}<\infty$ for some $\delta>0$, then if conditional on $\left(Y_{\mathbf{i}}\right)_{\mathbf{i}\in\mathbb{I}_k}$, the process $\mathbb{G}_n^*$ outer almost surely converges weakly in $l^{\infty}\left(\mathcal{F}\right)$ to a centered Gaussian process $\mathbb{G}$, the process $\mathbb{G}_n$ also converges weakly in $l^{\infty}(\mathcal{F})$ to $\mathbb{G}$. 
    Combined with Theorem \ref{thm:bootstrap_validity}, we have $$ \mathbb{G}_n \rightarrow \mathbb{G} \Leftrightarrow \mathbb{G}^*_n \xrightarrow{\mathrm{a.s.}_{outer}} \mathbb{G} $$
    consider an alternative, the multiplier bootstrap process adapted to jointly exchangeable arrays. Let $\left(\xi_i\right)^n_{i=1}$ be a sequence of i.i.d. random variables, independent from the original data $\left(Y_{\mathbf{i}}\right)_{\mathbf{i}\in \mathbb{I}_{n,2}}$, then the process 
    \begin{equation*}
        \mathbb{G}^{m*}_{n}:f \mapsto \frac{1}{\sqrt{n}}\sum^n_{i_1=1} \xi_{i_1} \left(\frac{1}{n-1}\sum_{1\leq i_2\neq i_1\leq n} \left[f(Y_{i_1,i_2})+f(Y_{i_2,i_1})\right] - 2\mathbb{P}_n f\right)
    \end{equation*}
    also outer almost surely converges weakly in $l^{\infty}(\mathcal{F})$ to $\mathbb{G}$, just as the proposed process $\mathbb{G}_n^*$.
    \item \myhl[myblue]{\textbf{Separately exchangeable arrays}} For the case of separately exchangeable arrays (the $n$ units stem from $k$ \textbf{different} populations, or multiway clustering as in \citet{menzel2021bootstrap}), we must assume a stronger version of Assumption \ref{assump:jexch_diss_array}
    \begin{assumption}{Stronger assumptions for separately exchangeable arrays}{strong_sep_exc_array}
        Consider randome variables $Y_{\mathbf{i}}$ where $\mathbf{i}=\left(i_1,\cdots,i_k\right) \in \mathbb{N}^{+k}$, implying that repetitions are allowed. Then assume that for any $\left(\pi_1,\cdots,\pi_k\right)\in\mathfrak{S}\left(\mathbb{N}^+\right)^k$,
        $$ \left(Y_{\mathbf{i}}\right)_{\mathbf{i}\in\mathbb{N}^{+k}} \overset{d}{=} \left(Y_{\pi_1(i_1),\cdots,\pi_k(i_k)}\right)_{\mathbf{i}\in\mathbb{N}^{+k}} $$
        and for any $A,B$, disjoint subsets of $\mathbb{N}^+$, $\left(Y_{\mathbf{i}}\right)_{\mathbf{i}\in A^k}$ is independent of $\left(Y_{\mathbf{i}}\right)_{\mathbf{i}\in B^k}$.
    \end{assumption}
    Under this stronger assumption, we have equality in distribution even for $\pi_1 = \cdots =\pi_k$. Here, denote 
    \begin{itemize}
        \item $\mathbf{1} = \left(1,\cdots,1\right)$
        \item $\mathbf{n} = \left(n_1,\cdots,n_k\right)$, where $n_j\geq 1$ denotes the number of units observed in population (or cluster) $j$, in general, $n_j\neq n_{j'}$ for $j\neq j'$
        \item sample at hand: $\left(Y_{\mathbf{i}}\right)_{\mathbf{1\leq i\leq n}}$, where $\mathbf{i}\geq \mathbf{i}'$ means that $i_j\geq i'_j$, $\forall j=1,\cdots,k$.
        \item $\underbar{n} = \min\left(n_1,\cdots,n_k\right)$
    \end{itemize}
    then the empirical measure and empirical process for separately exchagneable arrays are 
    \begin{align*}
        \mathbb{P}_nf &=\frac{1}{\prod^k_{j=1}n_j}\sum_{\mathbf{1\leq i\leq n}}f(Y_{\mathbf{i}}) & \mathbb{G}_n f& = \sqrt{\underbar{n}}\left(\mathbb{P}_n f-Pf\right)
    \end{align*}
    Consider the \textbf{pigeonhole bootstrap} process \citep{mccullagh2000resampling}, which is close the bootstrap in Theorem \ref{thm:bootstrap_validity}, except that weights are now independent from one coordinate to another:
    \begin{itemize}
        \item[1] For each $j\in \left\{1,\cdots,k\right\}$, $n_j$ elements are sampled with replacement and equal probability in the set $\left\{1,\cdots,n_j\right\}$. For each $i_j$, let $W_{i_j}^j$ denote the number of times $i_j$ is selected 
        \item[2] $k$-tuple $\mathbf{i}=\left(i_1,\cdots,i_k\right)$ is then selected $W_{\mathbf{i}}= \prod^k_{j=1} W_{i_j}^j$ times in the bootstrap sample 
    \end{itemize}
    the bootstrap process $\mathbb{G}^*_{\mathbf{n}}$ is then defined on $\mathcal{F}$ by 
    \begin{equation*}
        \mathbb{G}^*_{\mathbf{n}} f = \sqrt{\underbar{n}} \left(\frac{1}{\prod^k_{j=1}n_j} \sum_{\mathbf{1\leq i\leq n}}\left(W_{\mathbf{i}}-1\right) f(Y_{\mathbf{i}}) \right)
    \end{equation*}
    as with multisample U-statistics, assume an index $m\in \mathbb{N}^+$ and increasing functions $g_1,\cdots,g_k$ s.t. for all $j$, $n_j=g_j\left(m\right)\xrightarrow{m\rightarrow\infty} \infty$, and w.l.o.g., $\forall m\in\mathbb{N}^+$, $\exists j$ s.t. $g_j(m+1)>g_j(m)$, then  
    \begin{theorem}{Bootstrap Convergence: Separately Exchangeable Arrays}{bootstrap_conv_sepexcharray}
        Under Assumption \ref{assump:measurability_assumption} and \ref{assump:strong_sep_exc_array} and for every $j=1,\cdots,k$, $\exists\lambda_j\geq0$ s.t. $\underbar{n}/n_j \rightarrow \lambda_j \geq 0$, then 
        \begin{itemize}
            \item[1] If (A) of Assumption \ref{assump:assumptions_on_f} holds, $\sup_{f\in\mathcal{F}} \left\vert \mathbb{P}_{\mathbf{n}}f - Pf \right\vert \xrightarrow{\mathrm{a.s.}} 0$ and in $L^1$
            \item[2] If (B-i) of Assumption \ref{assump:assumptions_on_f} holds, the process $\mathbb{G}_n$ coverges weakly in $l^{\infty}(\mathcal{F})$ to a centered Gaussian process $\mathbb{G}_{\lambda}$ on $\mathcal{F}$ as $n\rightarrow\infty$, and the covariance kernel $K_{\lambda}$ of $\mathbb{G}_{\lambda}$ satisfies 
            \begin{equation*}
                K_{\lambda}\left(f_1,f_2\right) = \sum^k_{j=1}\lambda_j \mathrm{Cov}\left(f_1\left(Y_{\mathbf{1}}\right),f_1\left(Y_{\mathbf{2}_j}\right)\right)
            \end{equation*}
            where $\mathbf{2}_j$ is the $k$-tuple with 2 in each entry but 1 in entry $j$.
            \item[3] If (B-i) of Assumption \ref{assump:assumptions_on_f} holds, $\mathbb{G}^*_n \rightarrow \mathbb{G}_{\lambda}$ weakly, conditional on $\left(Y_{\mathbf{i}}\right)_{\mathbf{i}\in \mathbb{N}^{+k}}$ and outer almost surely
        \end{itemize}
    \end{theorem}
    Here, the case where $\lambda_j=0$ for some $j$ corresponds to \textbf{strongly unbalanced} designs with different rates of convergence to $\infty$ along the different dimensions of the array. In such case, only the dimensions with the slowest rate of convergence contribute to the asymptotic distribution.
\end{itemize}

\subsection{CLT for the estimator with heterogeneous clusters}
\citet{yap2023general} provides general conditions such that the plug-in mean estimator is asymptotically normal, and the \citet{cameron2011robust} variance estimator is consistent even when clusters are heterogeneous. The conditions mimic one-way clustering conditions, assuming that two observations are independent when they do not share any cluster, and \textbf{\textcolor{red}{not}} assuming separate exchangeability.

Consider for vectors $\left\{\mathbf{W}_i\right\}^n_{i=1}$, where $\mathbf{W}_i \coloneq  ( W_{i1},\cdots, W_{iK} )'\in \mathbb{R}^K $ and $i$ is the unit of observation, for the population of size $n$. The goal is to establish a central list theorem (CLT) for a 
weighted sum of the random vector, $\sum_i \omega_i \mathbf{W}_i$ where $\omega_i$ are non-stochastic scalar weights as $n\rightarrow \infty$.

\paragraph*{Notation}
Consider 2 clustering dimension $G$ and $H$, then 
\begin{itemize}
    \item $g(i),h(i)$: the cluster where observations $i$ belongs on the $G$ and $H$ dimensions, respectively
    \item \myhl[myblue]{\textbf{partition}}: For $C\in \left\{G,H\right\}$, let $\mathcal{N}^C_c$ denote the set of observations in cluster $c$ on dimension $C$
    \item $N^C_c = \left\vert \mathcal{N}^C_c \right\vert$: the cluster size for $C\in \left\{G,H\right\}$, and $N_{gh}\coloneq \left\vert \mathcal{N}^G_g \cap \mathcal{N}^H_h \right\vert$
\end{itemize}

\paragraph*{Assumptions}
Several assumptions are imposed to establish the main result
\begin{assumption}{Assumptions of \citet{yap2023general}}{assump_yap2023}
    \begin{itemize}
        \item[1] {\textbf{dependence structure}}: $\mathbf{W}_i \perp \mathbf{W}_j $ if $g(i)\neq g(j)$ and $h(i)\neq h(j)$
        \item[2] {\textbf{additional assumptions}}: For $C\in \left\{G,H\right\}$ and $k\in \left\{ 1,2,\cdots,K \right\}$, $\exists K_0<\infty$ s.t. 
        \begin{itemize}
            \item $\forall i, \mathbb{E}\left[W_{ik}^4\right] \leq K_0$: bounded fourth moment, stronger than the one-way clustering condition.
            \item $\frac{1}{\lambda_n} \max_c \left( \sum_{i\in\mathcal{N}^C_c}\left\vert \omega_i \right\vert\right)^2  \rightarrow 0$: the cluster with the largest weight to have relatively \textbf{small} contribution to the total variance, s.t. removing one cluster does not change the variance substantively, allowing the ratio of the size of any 2 clusters to diverge.
            \item $\frac{1}{\lambda_n} \sum_c \sum_{i,j\in \mathcal{N}^C_c} A_{ij} \left\vert \omega_i \omega_j \right\vert \leq K_0$: ruling out the purely interactive model in \citet{menzel2021bootstrap}
        \end{itemize}
    \end{itemize}
\end{assumption}
The key feature of Assumption \ref{assump:assump_yap2023} (1) is that it is agnostic about the dependent structure when $W_i$ and $W_j$ share at least one cluster. The DGP can be arbitrarily heterogeneous across different clusters. Consider a $0-1$ indicator:
$$
A_{ij} \coloneq \mathbf{1}\left[\mathbf{W}_i \not\perp \mathbf{W}_j\right]
$$
then we have $A_{ij}=A_{ji}$ and $A_{ii}=1$. \citet{yap2023general} established the following results 
\begin{theorem}{Asymptotic Normality of \citet{yap2023general}}{asymp_norm_yap2023}
    Under Assumption \ref{assump:assump_yap2023}, $$Q_n^{-1/2} \sum^n_{i=1} \omega_i \left(\mathbf{W}_i-\mathbb{E}\left[\mathbf{W}_i\right]\right) \xrightarrow{\mathrm{d}} \mathcal{N}\left(0,\mathbf{I}_K\right) $$ further 
    \begin{itemize}
        \item If $\mathbb{E}\left[\mathbf{W}_i\right] = 0,\forall i$, then $Q_n^{-1}\hat{Q}_n\xrightarrow{\mathrm{p}} \mathbf{I}_K$, where $\hat{Q}_n\coloneq \sum_i\sum_{j\in\mathcal{N}_i}\omega_i\omega_j \mathbf{W}_i \mathbf{W}_j'$
        \item If $\mathbb{E}\left[\mathbf{W}_i\right] = \mu, \forall i$\footnote{Here, the variance estimator need not be consistent, or even conservative. See Remark 2 of \citet{yap2023general} for details.} and $\frac{1}{\lambda_n} \sum_c\sum_{i,j\in\mathcal{N}^C_c}\left\vert \omega_i\omega_j \right\vert \leq K_0$ for some $K_0<\infty$, then
        \begin{align*}
            \bar{\mathbf{W}}&\xrightarrow{\mathrm{p}}\mu & Q_n^{-1}\hat{Q}_n\xrightarrow{\mathrm{p}}\mathbf{I}_K
        \end{align*}
        for $\hat{\mathbf{W}} = \left(\sum_i \omega_i \mathbf{W}_i\right)/\left(\sum_j\omega_j\right)$ and $\hat{Q}_n\coloneq \sum_i\sum_{j\in\mathcal{N}_i}\omega_i\omega_j \left(\mathbf{W}_i-\bar{\mathbf{W}}\right) \left(\mathbf{W}_j-\bar{\mathbf{W}}\right)'$
    \end{itemize}
    one-way clustering is a special case of these results when one dimension is weakly nested within the other.
\end{theorem}

\subsection{One-way robust clustering with large clusters}
\citet{sasaki2022non} established robust clustering for the cases wher the distribution of cluster sizes follows a power law with exponent less than 2, the conventional clustering method fails\footnote{That is, $\sup_g N_g/N\not\rightarrow 0$. An example is that California consists of 10\% of total sample among the 51 US states.}. Conssider again the linear model
\begin{align*}
    Y_{gi} &= \mathbf{X}_{gi}'\boldsymbol{\theta} + U_{gi} & \mathbb{E}\left[\mathbf{U}_{g}\mid \mathbf{X}_g\right]=0
\end{align*}
with a clustered sample $\left\{ \left\{ \left(Y_{gi},\mathbf{X}'_{gi}\right)' \right\}^{N_g}_{i=1} \right\}^G_{g=1}$. The OLS estimator gives 
\begin{equation*}
    \hat{\boldsymbol{\theta}} = \left(\sum^G_{g=1}\sum^{N_g}_{i=1}\mathbf{X}_gi\mathbf{X}'_{gi}\right)^{-1} \left(\sum^G_{g=1}\sum^{N_g}_{i=1}\mathbf{X}_{gi}Y_{gi}\right) = \left(\sum^G_{g=1}\mathbf{X}'\mathbf{X}\right)^{-1}\left(\sum^G_{g=1}\mathbf{X}'_g\mathbf{Y}_g\right)
\end{equation*}
and the commonly cluster-robust variance estimators take the form of 
\begin{equation*}
    \hat{V}_{\hat{\boldsymbol{\theta}}}^{\mathrm{CR}} = a_n \left(\sum^G_{g=1} \mathbf{X}'_g\mathbf{X}_g \right)^{-1} \left(\sum^G_{g=1} \hat{\mathbf{S}}_g \hat{\mathbf{S}}'_g \right) \left(\sum^G_{g=1} \mathbf{X}'_g\mathbf{X}_g \right)^{-1}
\end{equation*}
where $a_n\rightarrow 1$ is a suitable finite-sample adjustment, and $\hat{\mathbf{S}}_g =\sum^{N_g}_i=1\mathbf{X}_{gi}\hat{U}_{gi}$ with $\hat{U}_{gi} = Y_{gi} - \mathbf{X}'_{gi}\hat{\boldsymbol{\theta}}$. Conventionally, the adjustment is 
$$
a_n = \left( \frac{\sum^G_{g=1}N_g-1}{\sum^G_{g=1} N_g -p} \right) \left(\frac{G}{G-1}\right)
$$
with $p$ denoting the dimension of $\mathbf{X}_{gi}$. Asymptotically,
\begin{align*}
    \sqrt{G}\left( \hat{\boldsymbol{\theta}} - \boldsymbol{\theta} \right) = \left( \underbrace{\frac{1}{G}\sum^G_{g=1}\sum^{N_g}_{i=1}\mathbf{X}_{gi}\mathbf{X}'_{gi}}_{\xrightarrow{\mathrm{p}}\mathbf{Q}} \right)^{-1} \left( \underbrace{ \frac{1}{\sqrt{G}} \sum^G_{g=1} \sum^{N_g}_{i=1} \mathbf{X}_{gi}U_{gi} }_{\xrightarrow{\mathrm{d}} \mathcal{N}\left(0,\mathbf{V}\right)} \right) \xrightarrow{\mathrm{d}} \mathcal{N} \left(\mathbf{0},\mathbf{Q^{-1}VQ^{-1}}\right)
\end{align*}
where $\mathbf{Q} = \mathbb{E}\left[ \sum^{N_g}_{i=1}\mathbf{X}_{gi}\mathbf{X}'_{gi} \right]$, $\mathbf{V} = \mathrm{Var}\left[\sum^{N_g}_{i=1} \mathbf{X}_{gi}U_{gi}\right]$ and $\sum^{N_g}_{i=1} \mathbf{X}_{gi}U_{gi}$ have finite second moments. 

\paragraph*{Main results} for a given $k\in \left\{1,\cdots, p\right\}$, let $\sum_g$ and $Z_{gi}$ be the $k$-th coordinate of $\sum^{N_g}_{i=1} \mathbf{X}_{gi}U_{gi}$ and the $k-$th coordinate of $ \mathbf{X}_{gi}U_{gi}$ respectively.
Consider the following property, \myhl[myblue]{\textbf{regularly varying (RV)}}, of a distribution function $F$,
$$
\frac{1-F(xt)}{1-F(t)} \xrightarrow{t\rightarrow \infty} x^{-\alpha},\ \forall x>0
$$
for some constant $\alpha >0$, which is referred to as the \textbf{tail exponent}, measuring the tail heaviness of $F$. Then, let 
\begin{itemize}
    \item $F$ denote the marginal distribution of $Z_{gi}$
    \item $C^n$ denote, for each $n\geq 2$, the copula s.t. $$ \mathbb{P}\left(Z_{g1}\leq z_1, \cdots, Z_{gn}\leq z_n\right) = C^n \left(F(z_1),\cdots, F(z_n)\right) $$
\end{itemize}
\citet{sasaki2022non} make the following assumptions 
\begin{assumption}{Assumptions on the Distribution of $\left\{Z_{gi}\right\}_{g,i}$}{sasakiwang2022assump1}
    \begin{itemize}
        \item $\left\{Z_{gi}\right\}_{g,i}$ is identically distributed with CDF $F$, which is \textbf{RV} at infinity with $\alpha >1$\footnote{$Z_{gi}$ should have a finite mean, and a regularly varying tail (satisfied by many common heavy-tailed distributions including Pareto, Student-t, Cauchy, F). The tail condition cahracterize the moment conditions as \begin{align*}\mathbb{E}\left[\left\vert Z_{gi}\right\vert^r\right]=\infty &\forall r< \alpha & \mathbb{E}\left[\left\vert Z_{gi}\right\vert^r\right]<\infty &\forall r> \alpha \end{align*} }
        \item The copula density $c(u_1,\cdots,u_n) = \partial^n C(u_1,\cdots, u_n)/\partial u_1\cdots \partial u_n$ exists and is \textbf{uniformly bounded}\footnote{Allowing dependence among $Z_{g1},\cdots, Z_{gN_g}$ within each cluster $g$}
        \item $N_g$ is independent of $\left(Z_{g1},Z_{g2},\cdots\right)$ and its distribution $H$ is \textbf{RV} at infinity with $\beta>1$\footnote{The distribution of the cluster size $N_g$ is also regularly varying. One way to think about this is considering $N_g$ as the integer part of some continuous random variable with a regularly varying tail.}
    \end{itemize}
\end{assumption}
Under Assumption \ref{assump:sasakiwang2022assump1}, \citet{sasaki2022non} establish that 
\begin{theorem}{When Conventional Robust Clustering Fails}{sasakiwang2022thm1}
    If $\beta<\alpha$, then $\forall z>0$, as $t\rightarrow\infty$ $$ \frac{\mathbb{P}\left(\sum_g > zt\right)}{\mathbb{P}\left(\sum_g >t\right)} = z^{-\beta} $$
\end{theorem}
here, the tail of the distribution of $N_g$ dominates that of $Z_{gi}$, the tail heaviness of the summation $\sum_g = \sum^{N_g}_{i=1}Z_{gi}$ is dictated by that of $N_g$. Therefore, even if $Z_{gi}$ has a finite $r-$th moment for $r<\alpha$,
the $r-$th moment of $\sum_g$ might still be infinite if $r>\beta$. Hence, the conventional robust clustering may fail to exist even if $Z_{gi}$ has a bounded second moment. \citet{sasaki2022non} illustrated this problem with state clustering of PSID and 
some recent studies published on Econometrica. The way they choose to show that $\beta<2$ is by using the Hill plot \citep{drees2000make}.

\paragraph*{Proposed alternative estimators} \citet{sasaki2022non} proposed an simple fix as 
\begin{align*}
    \hat{\boldsymbol{\theta}}^{\mathrm{WCR}} &= \left(\sum^G_{g=1}N^{-1}_{g}\sum^{N_g}_{i=1}\mathbf{X}_gi\mathbf{X}'_{gi}\right)^{-1} \left(\sum^G_{g=1}N^{-1}_{g}\sum^{N_g}_{i=1}\mathbf{X}_{gi}Y_{gi}\right) = \left(\sum^G_{g=1}N^{-1}_{g}\mathbf{X}'\mathbf{X}\right)^{-1}\left(\sum^G_{g=1}N^{-1}_{g}\mathbf{X}'_g\mathbf{Y}_g\right) \\
    \hat{V}_{\hat{\boldsymbol{\theta}}}^{\mathrm{WCR}} &= a_n \left(\sum^G_{g=1} N^{-1}_{g}\mathbf{X}'_g\mathbf{X}_g \right)^{-1} \left(\sum^G_{g=1} N^{-2}_{g}\hat{\mathbf{S}}_g \hat{\mathbf{S}}'_g \right) \left(\sum^G_{g=1}N^{-1}_{g} \mathbf{X}'_g\mathbf{X}_g \right)^{-1}
\end{align*}
where $a_n\rightarrow 1$. Again, for a given $k\in \left\{1,\cdots, p\right\}$, let $\sum_g$ and $Z_{gi}$ be the $k$-th coordinate of $\sum^{N_g}_{i=1} \mathbf{X}_{gi}U_{gi}$ and the $k-$th coordinate of $ \mathbf{X}_{gi}U_{gi}$ respectively, and then similar to Thm.\ref{thm:sasakiwang2022thm1},
\begin{theorem}{\citet{sasaki2022non}'s Modification}{sasakiwang2022thm2}
    Under Assumption \ref{assump:sasakiwang2022assump1}, $\forall z>0$, as $t\rightarrow\infty$, $ \frac{\mathbb{P}\left(\tilde{\sum}_g > zt\right)}{\mathbb{P}\left(\tilde{\sum}_g >t\right)} = z^{-\beta} $ where $\tilde{\sum}_g = \sum_g/N_g = \sum^{N_g}_{i=1}Z_{gi}$
\end{theorem}
Thm.\ref{thm:sasakiwang2022thm2} established that $\tilde{\sum}_g$ has the same tailthe original score $Z_{gi}$. It requires the second moments of the score $\mathbf{X}_{gi}U_{gi}$ is \textbf{bounded} regardless of whether the second moment of $N_g$ if finite or not.

\begin{theorem}{CLT of \citet{sasaki2022non}'s Clustering}{sasakiwang2022thm3}
    Under Assumption \ref{assump:sasakiwang2022assump1}, and in addition, assume that $\left(N_g,\mathbf{X}_g,\mathbf{X}_g\right)$ is i.i.d. across $g$\footnote{Requiring the i.i.d. sampling \textbf{across} clusters, while allowing for arbitrary dependence within each cluster.}, and $\mathbb{E}\left[N^{-1}_g \mathbf{X}_g' \mathbf{X}_g\right]$ is non-singular\footnote{Ruling out multi-collinearity.}. With $\alpha>2$, then 
    $$
    \sqrt{G} \left(\hat{\boldsymbol{\theta}}^{\mathrm{WCR}} - \boldsymbol{\theta}\right) \xrightarrow{\mathrm{p}} \mathcal{N}\left(\mathbf{0},\mathbf{V}^{\mathrm{WCR}}\right)
    $$
    as $G\rightarrow \infty$, where 
    $$
    \mathbf{V}^{\mathrm{WCR}} = \left(\mathbb{E}\left[N_g^{-1}\mathbf{X}_g'\mathbf{X}_g\right]\right)^{-1} \left(\mathbb{E}\left[N_g^{-2}\mathbf{X}_g'\mathbf{X}_g \mathbf{U}_g^2\right]\right) \left(\mathbb{E}\left[N_g^{-1}\mathbf{X}_g'\mathbf{X}_g\right]\right)^{-1}
    $$
    and $G\hat{\mathbf{V}}^{\mathrm{WCR}} \xrightarrow{\mathrm{p}} \mathbf{V}^{\mathrm{WCR}}_{\hat{\boldsymbol{\theta}}}$ as $G\rightarrow \infty$.
\end{theorem}
Thm.\ref{thm:asymp_norm_yap2023} gives that the proposed clustering approach based on $\hat{\boldsymbol{\theta}}^{\mathrm{WCR}}$ works as far as $\alpha>2$ is true, regardless of whether $\beta<2$ is true or not.

\subsection{Two-Way Robust Clustering: Justification}
\citet{chiang2023using} developed a CLT for means of two-way clustered triangular arrays under mild conditions, which provides a theoretical justification for the asymptotic gaussianity of the statistics commonly occurs under two-way clustering.

Consider
$$
D_{it} = f_{NT}\left(\alpha_i,\gamma_t,\epsilon_{it}\right)
$$
normalize it, w.l.o.g., to $\mathbb{E}\left[D_{it}\right]=0$, define 
$$
\hat{\theta}_{NT} = \frac{1}{NT} \sum^N_{i=1}\sum^T_{t=1}D_{it}
$$
the goal is to show the \textbf{asymptotic Gaussianity} of $\hat{\theta}_{NT}$. The Hoeffding-type decomposition gives 
\begin{equation*}
    \hat{\theta}_{NT} = \underbrace{\sum^N_{i=1}a_i + \sum^T_{t=1}b_t}_{\coloneq L_{NT}} + \underbrace{\sum^N_{i=1}\sum^T_{t=1}w_{it}}_{\coloneq W_{NT}} + \underbrace{\sum^N_{i=1}\sum^T_{t=1}r_{it}}_{\coloneq R_{NT}}
\end{equation*}
where 
\begin{align*}
    a_i &= \frac{1}{N}\mathbb{E}\left[D_{it}\mid \alpha_i\right] & b_t &= \frac{1}{T}\mathbb{E}\left[D_{it}\mid \gamma_t\right] \\
    w_{it} &= \frac{1}{NT}\left(\mathbb{E}\left[D_{it}\mid \alpha_i,\gamma_t\right]-\mathbb{E}\left[D_{it}\mid \alpha_i\right] - \mathbb{E}\left[D_{it}\mid \gamma_t\right]\right) & r_{it} &=\frac{1}{NT}\left(D_{it}-\mathbb{E}\left[D_{it}\mid \alpha_i,\gamma_t\right]\right)
\end{align*}
straightforwardly,
\begin{align*}
    \mathbb{E}\left[a_i\right]=\mathbb{E}\left[b_t\right] = \mathbb{E}\left[w_{it}\right] = \mathbb{E}\left[r_{it}\right]&=0\\
    \mathbb{E}\left[w_{it}\mid \alpha_i\right] = \mathbb{E}\left[w_{it}\mid \gamma_t\right] &=0 \\
    \mathbb{E}\left[a_iw_{it}\right] = \mathbb{E}\left[a_ir_{it}\right] = \mathbb{E}\left[\gamma_tw_{it}\right] = \mathbb{E}\left[\gamma_t r_{it}\right] &=0
\end{align*}
here, $W_{NT}=\sum^N_{i=1}\sum^T_{t=1}w_{it}$ is the potentially non-guassian part, as discussed by \citet{menzel2021bootstrap}. \citet{chiang2023using} argue that if the DGP is treated as a triangular array, $W_{NT}$ is often asymptotically gaussian. For a simple, generic DGP
\begin{equation*}
    D_{it} = \alpha_{i0} + \gamma_{t0} + \sum^J_{j=1}\lambda_j \alpha_{ij} \gamma_{tj} + \epsilon_{ij}
\end{equation*}
where $\left\{\alpha_{ij}\right\}^J_{j=0}$ are $i-$specific latent factors, $\left\{\gamma_{tj}\right\}^J_{j=0}$ are $t-$specific latent factors, $\epsilon_{it}$ is an idiosyncratic component. Suppose that the factor loading $\lambda_j$ are non-zero, then \citet{chiang2023using} provide a 
summary on when to use the TWCR standard error for $\hat{\theta}=\left(NT\right)^{-1}\sum^N_{i=1}\sum^T_{t=1}D_{it}$:
\begin{table}[h!]
    \caption{\citet{chiang2023using}'s Summary on TWCR SE Validity}
    \begin{center}
        \begin{tabular}{ccccc}
        Small $J$ & $\alpha_{i0}$ & $\gamma_{t0}$ & $\epsilon_{it}$ & TWCR SE valid? \\ \hline
        Yes  & Degenerate & Degenerate & Degenerate & \textbf{No} \\
        \multicolumn{4}{c}{Other cases} & \textbf{Yes}
        \end{tabular}
    \end{center}
\end{table}
Except for the extreme case where \myhl[myblue]{the number of factors $J$ is \textbf{small}}, latent factors on both dimension $i$ and $t$ are degenerate, the idiosyncratic component $\epsilon_{it}$ is also degenerate.

\section{Clustering in Experiments}
\subsection{When to cluster}
\citet{abadie2023should} highlight 3 common misconceptions on clustering adjustments:
\begin{itemize}
    \item clustering when there is a nonzero correlation between residuals for units within the same cluster
    \item when clustering makes a difference, one should cluster
    \item either fully adjust for clustering, or not at all
\end{itemize}
and they propose a new design-based clustering framework to address the overuse and overconvservatism of the commonly used clustering.

They propose a framework with \textbf{3 sources} of sampling variations:
\begin{itemize}
    \item[1] variation across samples in which units are observed in each cluster 
    \item[2] variation in which clusters are observe
    \item[3] variation in the treatment assignment across units
\end{itemize}
Standard framework for clustering focuses on the first 2 sources of uncertainty. How much the 3 sources of variations matter depends on \myhl[myblue]{\textbf{the sampling process}}, \myhl[myblue]{\textbf{the assignment process}}, and  \myhl[myblue]{\textbf{the heterogeneity}} in the treatment effects across clusters.

\subsubsection{Sampling process}
Consider a \textbf{sequence of populations} indexed by $k$, and 
\begin{itemize}
    \item the $k$-th population has $n_k$ units, indexed by $i=1,\cdots,n_k$
    \item the population is partitioned into $m_k$ clusters, and $m_{k,i}\in \left\{1,\cdots,m_k\right\}$ denote the cluster to which unit $i$ of population $k$ belongs 
    \item number of units in cluster $m$ of population $k$ is $n_{k,m}\geq 1$
    \item 2 potential outcomes: treated $y_{k,i}(1)$, control $y_{k,i}(0)$
\end{itemize}
Thus, the population is characterized by triples $\left(m_{k,i},y_{k,i}(0),y_{k,i}(1)\right)$, for units $1,\cdots,n_k$ and clusters $1,\cdots,m_k$. Then, 
\begin{itemize}
    \item \myhl[myblue]{\textbf{population ATE}} $$ \tau _k = \frac{1}{n_k}\sum^{n_k}_{i=1} \left(y_{k,i}(1) - y_{k,0}(0)\right) $$
    \item \myhl[myblue]{\textbf{population ATE by clutser}} $$ \tau_{k,m} = \frac{1}{n_{k,m}} \sum^{n_k}_{i=1}\mathbf{1}\left\{m_{k,i}=m\right\} \left(y_{k,i}(1)-y_{k,i}(0)\right) $$
\end{itemize}
naturally. we have $$ \tau_k = \sum^{m_k}_{m=1}\frac{n_{k,m}}{n_k} \tau_{k,m} $$
For unit $i$ in population $k$, there are 2 components of the stochastic nature: 
\begin{itemize}
    \item \myhl[myblue]{\textbf{Sampling process}}: a random variable $R_{k,i}=1$ if unit $i$ belongs to the sample, $0$ if not. The sampling process is independent of the potential outcomes and the assignments, and consists of 2 stages  
    \begin{itemize}
        \item[\textbf{i}] clusters are sampled with cluster sampling probability $q_k\in(0,1]$ 
        \begin{itemize}
            \item $q_k=1$: sample all clusters, i.e. \textbf{random sampling}
            \item $q_k<1$: \textbf{clustered sampling}
            \item $q_k\rightarrow 0$: only a small fraction of the clusters sampled
        \end{itemize}
        \item[\textbf{ii}] units are sampled from the subpopulation consisting of all the sampled clusters, with probability $p_k\in (0,1]$
        \begin{itemize}
            \item $p_k=1$: sample all units in the population
            \item $p_k\rightarrow 0$: only a small sample of units sampled
        \end{itemize}
    \end{itemize}
    \item \myhl[myblue]{\textbf{Assignment process}} the stochastic treatment indicator $W_{k,i}\in\left\{0,1\right\}$ is determined by a 2-stage process as well:
    \begin{itemize}
        \item[\textbf{i}] for cluster $m$ in population $k$, \textbf{randomly} draw an assignment probability $A_{k,m}\in\left[0,1\right]$ from a distribution with $\mu_k$ (bounded away from 0 and 1 uniformly in $k$), variance $\sigma^2_k$, \textbf{independently} for each cluster
        \begin{itemize}
            \item $\sigma^2_k=0$: $A_{k,m}$ is the same across all clusters, i.e., \textbf{random assignment}
            \item $\sigma^2_k>0$: assignment probabilities depend on clustering, and
            \begin{itemize}
                \item[-] \textbf{\underline{clustered assi}g\underline{nment}}, $\sigma^2_k=\mu_k(1-\mu_k)$: \textbf{no within-cluster} variation in $W_{k,i}$\footnote{This is the upper bound of $\sigma^2_k$, attained when $A_{k,m}$ can only take the values $0$ or $1$}
                \item[-] \textbf{p\underline{artiall}y \underline{clustered assi}g\underline{nment}}, $0<\sigma^2_k< \mu_k(1-\mu_k)$: assignment depends on cluster but not all units in the same cluster necessarily have the same value of $W_{k,i}$
            \end{itemize}
        \end{itemize}
        \item[\textbf{ii}] each unit in cluster $m$ is assigned to the treatment independently, with cluster-specific probability $A_{k,m}$
    \end{itemize}
\end{itemize}

\subsubsection{LS estimator and variance}
Let 
\begin{align*}
    N_{k,1} &= \sum^{n_k}_{i=1}R_{k,i}W_{k,i} & N_{k,0} &= \sum^{n_k}_{i=1} R_{k,i} (1-W_{k,i})
\end{align*}
be the number of \textbf{treated} and \textbf{untreated} units in the sample (both randomly variables), total sample size is then $N_k=N_{k,1}+N_{k,0}$, then for the regression, 
$$
Y_{k,i} = \alpha + \tau_k W_{k,i} + \epsilon_{k,i}
$$
the OLS estimator of $\beta$ is equal to the difference in means\footnote{$\leftarrow$ $a\vee b = \max\left\{a,b\right\}$}:
\begin{equation*}
    \hat{\tau}_k = \frac{1}{N_{k,1}\vee 1}\sum^{n_k}_{i=1}R_{k,i}W_{k,i}Y_{k,i} - \frac{1}{N_{k,0}\vee 1}\sum^{n_k}_{i=1}R_{k,i}(1-W_{k,i})Y_{k,i}
\end{equation*}
\citet{abadie2023should} assume that 
\begin{itemize}
    \item $m_kq_k\rightarrow \infty$: the expected number of sampled clusters goes to infinity
    \item $\lim\inf_{k\rightarrow \infty} p_k \min_m n_{k,m} > 0 $: average number of observations sampled per cluster (conditional on sampled) does not go to 0
    \item $\lim \sup_{k\rightarrow \infty} \frac{\max_m n_{k,m}}{\min_n n_{k,m}} < \infty$: the imbalance between the number of units across clusters is bounded
\end{itemize}

\paragraph*{Large $k$ distribution of $\hat{\tau}_k$} 
Let
\begin{align*}
    \alpha_k &= \frac{1}{n_k}\sum^{n_k}_{i=1} y_{k,i}(0) & u_{k,i}(1) =& y_{k,i} - (\alpha_k+\tau_k) & u_{k,i}(0) &= y_{k,i}(0)-\alpha_k
\end{align*}
then
$$
\frac{\sqrt{N_k} \left(\hat{\tau}_k-\tau_k\right)}{\sqrt{v_k}} \xrightarrow{d} \mathcal{N}\left(0,1\right)
$$
where 
\begin{align*}
    v_k =& \frac{1}{n_k} \sum^{n_k}_{i=1}\left( \frac{u_{k,i}^2(1)}{\mu_k} + \frac{u^2_{k,i}(0)}{1-\mu_k} \right)\\
    &-p_k \frac{1}{n_k} \sum^{n_k}_{i=1}\left(u_{k,i}(1)-u_{k,i}(0)\right)^2 - p_k\sigma^2_k \frac{1}{n_k} \sum^{n_k}_{i=1} \left( \frac{u_{k,i}(1)}{\mu_k} + \frac{u_{k,i}(0)}{1-\mu_k} \right)^2 \\
    &+ p_k\left(1-q_k\right) \frac{1}{n_k}\sum^{m_k}_{m=1}\left(\sum^{n_k}_{i=1} \mathbf{1}\left\{m_{k,i}=m\right\} \left(u_{k,i}(1)-u_{k,i}(0)\right) \right)^2\\
    &+ p_k\sigma^2_k \frac{1}{n_k}\sum^{m_k}_{m=1} \left(\sum^{n_k}_{i=1} \mathbf{1}\left\{m_{k,i}=m\right\} \left(\frac{u_{k,i}(1)}{\mu_k} + \frac{u_{k,i}(0)}{1-\mu_k}\right)\right)^2
\end{align*}
Under the following cases:
\begin{itemize}
    \item \myhl[myblue]{\textbf{random sampling} ($q_k=1$)} and \myhl[myblue]{\textbf{random assignment} ($\sigma^2_k=0$)}: then simply $v_k$ as
    \begin{equation*}
        v_k = \underbrace{\frac{1}{n_k} \sum^{n_k}_{i=1}\left( \frac{u_{k,i}^2(1)}{\mu_k} + \frac{u^2_{k,i}(0)}{1-\mu_k} \right)}_{\text{robust variance estimator}} - \underbrace{p_k \frac{1}{n_k} \sum^{n_k}_{i=1}\left(u_{k,i}(1)-u_{k,i}(0)\right)^2}_{\text{finite-sample correction}}
    \end{equation*} 
    here, the finite-sample correction vanishes if there is either \textbf{no heterogeneity} in the treatment effects ($u_{k,i}(1)- u_{k,i}(0) = y_{k,i}(1)-y_{k,i}(0) - \tau_k=0$), or the sample is \textbf{small} ($p_k\simeq 0$)
    \item \myhl[myblue]{\textbf{clustered sampling} ($q_k<1$)}, the component 
    \begin{equation*}
        p_k\left(1-q_k\right) \frac{1}{n_k}\sum^{m_k}_{m=1}\left(\sum^{n_k}_{i=1} \mathbf{1}\left\{m_{k,i}=m\right\} \left(u_{k,i}(1)-u_{k,i}(0)\right) \right)^2 = p_k(1-q_k) \frac{1}{n_k} \sum^{m_k}_{m=1} n^2_{k,m} \left(\tau_{k,m}-\tau_k\right)^2
    \end{equation*}
    this term vanishes when there is no heterogeneity in the average treatment effect across clusters. Information on whether to include this term to adjust for clustered sampling ($q_k<1$) must come from \textbf{outside the sample}.
    \item \myhl[myblue]{\textbf{clustered assignment} ($\sigma^2_k>0$)}, then 2 more terms added 
    \begin{align*}
        -p_k\sigma^2_k \frac{1}{n_k} \sum^{n_k}_{i=1} \left( \frac{u_{k,i}(1)}{\mu_k} + \frac{u_{k,i}(0)}{1-\mu_k} \right)^2 + p_k\sigma^2_k \frac{1}{n_k}\sum^{m_k}_{m=1} \left(\sum^{n_k}_{i=1} \mathbf{1}\left\{m_{k,i}=m\right\} \left(\frac{u_{k,i}(1)}{\mu_k} + \frac{u_{k,i}(0)}{1-\mu_k}\right)\right)^2
    \end{align*}
    the sign of it depends on the amount of variation in potential outcomes that can be explained by the clusters. In contrast to the lack of sample information about the need to adjust for clustered sampling, the sample is potentially informative about the need to account for clustered assignment.
\end{itemize}

The 5 terms of $v_k$ might be of \textbf{different} orders: 
\begin{itemize}
    \item $\frac{1}{n_k} \sum^{n_k}_{i=1}\left( \frac{u_{k,i}^2(1)}{\mu_k} + \frac{u^2_{k,i}(0)}{1-\mu_k} \right)$: average of bounded terms, of \textbf{order} $O(1)$
    \item $p_k \frac{1}{n_k} \sum^{n_k}_{i=1}\left(u_{k,i}(1)-u_{k,i}(0)\right)^2 + p_k\sigma^2_k \frac{1}{n_k} \sum^{n_k}_{i=1} \left( \frac{u_{k,i}(1)}{\mu_k} + \frac{u_{k,i}(0)}{1-\mu_k} \right)^2 $: \textbf{at most} of the same order as the first term. If $p_k\simeq 0$ (the sample is small relative to the population of sampled clusters), dominated by the first term
    \item the order of the last 2 terms depends on asymptotic of cluster sizes
    \begin{align*}
        & p_k\left(1-q_k\right) \frac{1}{n_k}\sum^{m_k}_{m=1}\left(\sum^{n_k}_{i=1} \mathbf{1}\left\{m_{k,i}=m\right\} \left(u_{k,i}(1)-u_{k,i}(0)\right) \right)^2\\
    &+ p_k\sigma^2_k \frac{1}{n_k}\sum^{m_k}_{m=1} \left(\sum^{n_k}_{i=1} \mathbf{1}\left\{m_{k,i}=m\right\} \left(\frac{u_{k,i}(1)}{\mu_k} + \frac{u_{k,i}(0)}{1-\mu_k}\right)\right)^2
    \end{align*}
    \begin{itemize}
        \item if cluster sizes are bounded as $k$ increases: they are also order $O(1)$
        \item if cluster sizes increase with $k$: they can be of higher order and dominate the variance
    \end{itemize}
    it also depends on $p_k$, clustering in sampling, clustering in assignment, heterogeneity in potential outcomes.
\end{itemize}

\paragraph*{Robust and cluster robust variance estimators} Let $$ \hat{U}_{k,i} = Y_{k,i}-\hat{\alpha}_k - \hat{\tau}_kW_{k,i} $$ be the residuals from the regression of $Y_{k,i}$ on a constant and $W_{k,i}$, then the 2 common estimators of the variance of $\sqrt{N_k}\left(\hat{\tau}_k-\tau_k\right)$ are
\begin{itemize}
    \item \myhl[myblue]{\textbf{robust variance estimator}}: 
    \begin{equation*}
        \hat{V}_k^{\mathrm{robust}} = \frac{1}{\bar{W}^2_k\left(1-\bar{W}_k\right)^2} \left\{ \frac{1}{N_k}\sum^{n_k}_{i=1} R_{k,i}\hat{U}_{k,i}^2 \left(W_{k,i}-\bar{W}_k\right)^2 \right\}
    \end{equation*}
    where $ \bar{W}_k = \frac{1}{N_k\vee 1} \sum^{n_k}_{i=1}R_{k,i}W_{k,i} $, let $v_k^{\text{robust}} = \frac{1}{n_k} \sum^{n_k}_{i=1}\left(\frac{u^2_{k,i}(1)}{\mu_k} + \frac{u^2_{k,i}(0)}{1-\mu_k}\right)$, then under some regularity conditions, 
    $$ \frac{\hat{V}_k^{\text{robust}}}{v_k} = \frac{v_k^{\text{robust}}}{v_k} + o_p(1) $$
    notice that $v_k^{\text{robust}} - v_k$ can be positive or negative in general, so the robust variance estimator can be invalid in large samples.
    \item \myhl[myblue]{\textbf{cluster variance estimator}}: 
    \begin{equation*}
        \hat{V}^{\text{cluster}}_k = \frac{1}{\bar{W}^2_k(1-\bar{W}_k)^2} \times \left\{ \frac{1}{N_k} \sum^{m_k}_{m=1}\left( \sum^{n_k}_{i=1} \mathbf{1}\left\{m_{k,i}=m\right\} R_{k,i}\hat{U}_{k,i} \left(W_{k,i}-\bar{W}_k\right) \right)^2 \right\}
    \end{equation*}
    define
    \begin{align*}
        v_k^{\text{cluster}} =& \frac{1}{n_k} \sum^{n_k}_{i=1}\left( \frac{u_{k,i}^2(1)}{\mu_k} + \frac{u^2_{k,i}(0)}{1-\mu_k} \right)\\
        &-p_k \frac{1}{n_k} \sum^{n_k}_{i=1}\left(u_{k,i}(1)-u_{k,i}(0)\right)^2 - p_k\sigma^2_k \frac{1}{n_k} \sum^{n_k}_{i=1} \left( \frac{u_{k,i}(1)}{\mu_k} + \frac{u_{k,i}(0)}{1-\mu_k} \right)^2 \\
        &+ p_k \frac{1}{n_k}\sum^{m_k}_{m=1}\left(\sum^{n_k}_{i=1} \mathbf{1}\left\{m_{k,i}=m\right\} \left(u_{k,i}(1)-u_{k,i}(0)\right) \right)^2\\
        &+ p_k\sigma^2_k \frac{1}{n_k}\sum^{m_k}_{m=1} \left(\sum^{n_k}_{i=1} \mathbf{1}\left\{m_{k,i}=m\right\} \left(\frac{u_{k,i}(1)}{\mu_k} + \frac{u_{k,i}(0)}{1-\mu_k}\right)\right)^2
    \end{align*}
    then $\hat{V}^{\text{cluster}}_k$ is close to $v_k^{\text{cluster}}$ s.t. $$ \frac{\hat{V}_k^{\text{cluster}}}{v_k} = \frac{v_k^{\text{cluster}}}{v_k} + o_p(1) $$
    and $v_k^{\text{cluster}}-v_k$ is always \textbf{nonnegative}. For large $k$, the cluster variance can be conservative.
\end{itemize}

\paragraph*{Comparison}
Comparing $v^{\mathrm{robust}}_k$ and $v_k$, we have 
\begin{align*}
    v_k^{\text{robust}} - v_k =&  p_k \frac{1}{n_k} \sum^{n_k}_{i=1}\left(u_{k,i}(1)-u_{k,i}(0)\right)^2 - p_k\left(1-q_k\right) \frac{1}{n_k}\sum^{m_k}_{m=1}\left(\sum^{n_k}_{i=1} \mathbf{1}\left\{m_{k,i}=m\right\} \left(u_{k,i}(1)-u_{k,i}(0)\right) \right)^2\\
    &+ p_k\sigma^2_k \frac{1}{n_k} \sum^{n_k}_{i=1} \left( \frac{u_{k,i}(1)}{\mu_k} + \frac{u_{k,i}(0)}{1-\mu_k} \right)^2 - p_k\sigma^2_k \frac{1}{n_k}\sum^{m_k}_{m=1} \left(\sum^{n_k}_{i=1} \mathbf{1}\left\{m_{k,i}=m\right\} \left(\frac{u_{k,i}(1)}{\mu_k} + \frac{u_{k,i}(0)}{1-\mu_k}\right)\right)^2
\end{align*}

which consists of 2 terms: the first term
$$p_k\frac{1}{n_k}\left[ \sum^{n_k}_{i=1}\left(u_{k,i}(1)-u_{k,i}(0)\right)^2 - \left(1-q_k\right)\sum^{m_k}_{m=1}n^2_{k,m}\left(\tau_{k,m}-\tau_k\right)^2 \right]$$
\begin{itemize}
    \item $=0$ with \textbf{homogeneous treatment effects}: $u_{k,i}(1)-u_{k,i}(0)=0$, for $i=1,\cdots,n_k$ and $\tau_{k,m}-\tau_k=0$ for all $m=1,\cdots,m_k$
    \item $>0$ with \textbf{all clusters sampled}, $q_k=1$, and \textbf{heterogeneous treatment effects}
    \item \myhl[myblue]{\textbf{possibly} $<0$}: $q_k<1$ (only a fraction of clusters are sampled), and $n^2_{k,m}$ large enough.
\end{itemize}
and the second term
\begin{align*}
    &p_k\sigma^2_k \frac{1}{n_k} \sum^{n_k}_{i=1} \left( \frac{u_{k,i}(1)}{\mu_k} + \frac{u_{k,i}(0)}{1-\mu_k} \right)^2 - p_k\sigma^2_k \frac{1}{n_k}\sum^{m_k}_{m=1} \left(\sum^{n_k}_{i=1} \mathbf{1}\left\{m_{k,i}=m\right\} \left(\frac{u_{k,i}(1)}{\mu_k} + \frac{u_{k,i}(0)}{1-\mu_k}\right)\right)^2 \\
    =& p_k\sigma^2_k\sum^{m_k}_{m=1} \frac{n_{k,m}}{n_k} \left[ \frac{1}{n_{k,m}}\sum^{n_k}_{i=1}\mathbf{1}\left\{m_{k,i}=m\right\} \left(\frac{u_{k,i}(1)}{\mu_k} + \frac{u_{k,i}(0)}{1-\mu_k}\right)^2 \right. \\
    & - \left. n_{k,m} \left( \frac{1}{n_{k,m}}\sum^{n_k}_{i=1} \mathbf{1}\left\{m_{k,i}=m\right\} \left(\frac{u_{k,i}(1)}{\mu_k} + \frac{u_{k,i}(0)}{1-\mu_k}\right)\right)^2 \right]
\end{align*}
\begin{itemize}
    \item $=0$ if there is no clustered assignment: $\sigma^2_k=0$
    \item close to $0$ if the heterogeneity in potential outcomes is small: $u_{k,i}(1),u_{k,i}(0)$ close to $0$
    \item $>0$ if there is heterogeneity in potential outcomes, but average potential outcomes are \textbf{nearly constant} across clusters
    \item $<0$ if the clusters explain enough heterogeneity in potential outcomes, could be very large if $n_{k,m}$ is large
\end{itemize}

Compare $v^{\text{cluster}}_k$ and $v_k$, we have 
\begin{align*}
    v_k^{\text{cluster}} - v_k &= p_kq_k \frac{1}{n_k} \sum^{m_k}_{m=1}\left(\sum^{n_k}_{i=1} \mathbf{1}\left\{m_{k,i}=m\right\} \left(u_{k,i}(1)-u_{k,i}(0)\right) \right)^2\\
    &= \left(\frac{p_kn_k}{m_k}\right) q_k \left\{ \frac{1}{m_k} \sum^{m_k}_{m=1} \left(\frac{n_{k,m}m_k}{n_k}\right)^2 \left(\tau_{k,m}-\tau_k\right)^2 \right\}
\end{align*}
hence, 
\begin{itemize}
    \item $v^{\text{cluster}}_k-v_k\geq 0$: cluster standard errors are (very) conservative in general
    \item $v^{\text{cluster}}-v_k\simeq 0$ when $q_k$ is small (the expected fraction of clusters in the sample), or when the average treatment effect is nearly \textbf{constant} between clusters.
\end{itemize}

\subsubsection{New variance estimators}
\citet{abadie2023should} proposed 2 estimators of the variance of $\hat{\tau}_k$, one analytic based on a correction to $\hat{V}^{\text{cluster}}_k$, one based on resampling.

\paragraph*{Random sampling: $q_k=1$}
When all clusters are observed ($q_k=1$), but allowing for general $p_k$, let 
$$
U_{k,i} = W_{k,i} u_{k,i}(1) + (1-W_{k,i})u_{k,i}(0)
$$
first, approximate the normalized error of $\hat{\tau}_k$ by a normalized sample average over clusters 
\begin{equation*}
    \frac{\sqrt{N_k}\left(\hat{\tau}_k-\tau_k\right)}{\sqrt{v_k}} = \frac{1}{\sqrt{n_kp_kv_k}\mu_k\left(1-\mu_k\right)}\sum^{m_k}_{m=1}C_{k,m} + o_p(1)
\end{equation*}
where $C_{k,m}=\sum^{n_k}_{i=1}\mathbf{1}\left\{m_{k,i}=m\right\} R_{k,i}\left(W_{k,i}-\mu_k\right) U_{k,i}$ are independent across clusters. Then 
For the term $C_{k,m}$, its expectation $\mathbb{E}\left[C_{k,m}\right]$ is \textbf{not} 0 in general for each cluster separately, but equals to 0 when sum over \textbf{all} clusters:
\begin{align*}
    \mathbb{E}\left[C_{k,m}\right] &= n_{k,m}p_k\mu_k \left(1-\mu_k\right)\left(\tau_{k,m}-\tau_k\right) & \sum^{m_k}_{m=1}\mathbb{E}\left[C_{k,m}\right] &= p_k\mu_k \left(1-\mu_k\right) \sum^{m_k}_{m=1} n_{k,m}\left(\tau_{k,m}-\tau_k\right) = 0
\end{align*}
Now, since we have 
\begin{equation*}
    \frac{\hat{V}^{\text{cluster}}_k}{v_k} = \frac{1}{n_kp_kv_k} \left(\frac{1}{\mu_k(1-\mu_k)}\right)^2 \sum^{m_k}_{m=1}C^2_{k,m} + o_p(1)
\end{equation*}
and $\mathrm{var}\left(C_{k,m}\right) \leq \mathbb{E}\left[C^2_{k,m}\right]$, the expectation of $\frac{1}{n_kp_kv_k} \left(\frac{1}{\mu_k(1-\mu_k)}\right)^2 \sum^{m_k}_{m=1}C^2_{k,m}$ is  $\hat{V}^{\text{cluster}}_k$ is bigger than the variance of $\frac{\sqrt{N_k}\left(\hat{\tau}_k-\tau_k\right)}{\sqrt{v_k}}$, leading to $\hat{V}^{\text{cluster}}_k$ being conservative.

Plug in $\sum^{m_k}_{m=1}\mathbb{E}\left[C_{k,m}\right]=0$ back into normalized $\hat{\tau}_k$, get 
\begin{align*}
    \frac{\sqrt{N_k}\left(\hat{\tau}_k-\tau_k\right)}{\sqrt{v_k}} &= \frac{1}{\sqrt{n_kp_kv_k}\mu_k\left(1-\mu_k\right)}\sum^{m_k}_{m=1}C_{k,m} + o_p(1) \\
    &= \frac{1}{\sqrt{n_kp_kv_k}\mu_k\left(1-\mu_k\right)}\sum^{m_k}_{m=1}\left(C_{k,m}-\mathbb{E}\left[C_{k,m}\right]\right) + o_p(1) \\
    &= \frac{1}{\sqrt{n_kp_kv_k}\mu_k\left(1-\mu_k\right)}\sum^{m_k}_{m=1}\left( C_{k,m,1}+C_{k,m,2} \right) + o_p(1)
\end{align*}
where 
\begin{align*}
    C_{k,m,1} &= \sum^{n_k}_{i=1} \mathbf{1}\left\{m_{k,i}=m\right\}\left(R_{k,i}-p_k\right) \left(\tau_{k,m}-\tau_k\right) \mu_k \left(1-\mu_k\right) \\
    C_{k,m,2} &= \sum^{n_k}_{i=1} \mathbf{1}\left\{m_{k,i}=m\right\} R_{k,i} \left[\left(W_{k,i}-\mu_k\right)U_{k,i} - \left(\tau_{k,m}-\tau_k\right)\mu_k\left(1-\mu_k\right) \right]
\end{align*}
$C_{k,m,1}$ and $C_{k,m,2}$ are \textbf{mean 0} and \textbf{uncorrelated}, and \textbf{uncorrelated across clusters}, and
\begin{itemize}
    \item variance of $\frac{\sum^{m_k}_{m=1}C_{k,m,1}}{\sqrt{n_kp_k}\mu_k \left(1-\mu_k\right)}$ is $$ \left(1-p_k\right) \sum^{m_k}_{m=1}\frac{n_{k,m}}{n_k}\left(\tau_{k,m}-\tau_k\right)^2 $$
    \item a direct estimator of the variance of $\sum^{m_k}_{m=1} C_{k,m,2}$ is $$ \sum^{m_k}_{m=1}\left( \sum^{n_k}_{i=1}\mathbf{1}\left\{m_{k,i}=m\right\} R_{k,i}\left(\left(W_{k,i}-\bar{W}_k\right)\hat{U}_{k,i} -\left(\hat{\tau}_{k,m}-\hat{\tau}_k\right)\bar{W}_k\left(1-\bar{W}_k\right) \right) \right)^2 $$
    where $\hat{\tau}_{k,m}$ is the estimated treatment effect (sample average) in cluster $m$. This estimator could be biased from the correlation between the estimation errors of its components (due to $\bar{W}_k$), to address this, \citet{abadie2023should} proposed a sample-splitting estimation procedure:
    \begin{itemize}
        \item[\textbf{1}] split the sample randomly into 2 subsamples, indexed by $Z_{k,i}\in \left\{0,1\right\}$, where $Z_{k,i}=1$ for the second subsample. Let $\bar{Z}_k$ be the mean of $Z_{k,i}$
        \item[\textbf{2}] use the first subsample $Z_{k,i}=0$, get estimates $\hat{\tau}^*_{k,m}$, $\hat{\alpha}^*_k$ and $\hat{\tau}^*_k$ 
        \item[\textbf{3}] for the second subsample $Z_{k,i}=1$, calculate the residuals $\hat{U}^*_{k,i} = Y_{k,i} - \hat{\alpha}^*_k -\hat{\tau}^*_k W_{k,i}$
        \item[\textbf{4}] estimate the normalized variance (for $q_k=1$) as 
        \begin{align*}
            \hat{V}^{\text{CCV}}_k(1) =& \frac{1}{N_k \bar{W}^2_k \left(1-\bar{W}_k\right)^2} \times \\
            & \sum^{m_k}_{m=1} \left[ \frac{1}{\bar{Z}^2_k} \left( \sum^{n_k}_{i=1} \mathbf{1}\left\{m_{k,i}=m\right\}R_{k,i}Z_{k,i}\times \left(\left(W_{k,i}-\bar{W}_k\right)\hat{U}^*_{k,i} - \left(\hat{\tau}^*_{k,m}-\hat{\tau}^*_k\right) \bar{W}_k\left(1-\bar{W}_k\right) \right)^2 \right. \right. \\
            & \left. - \frac{1-\bar{Z}_k}{\bar{Z}^2_k} \sum^{n_k}_{i=1} \mathbf{1}\left\{m_{k,i}=m\right\} R_{k,i}Z_{k,i}\left( \left(W_{k,i}-\bar{W}_k\right)\hat{U}^*_{k,i} - \left(\hat{\tau}^*_{k,m}-\hat{\tau}^*_k\right)\bar{W}_k \left(1-\bar{W}_k\right) \right)^2 \right] \\
            & + \left(1-p_k\right) \sum^{m_k}_{m=1}\frac{\bar{N}_{k,m}}{N_k} \left(\hat{\tau}_{k,m}-\hat{\tau}_k\right)^2
        \end{align*}
        where $\bar{N}_{k,m}$ is the size of the sample in cluster $m$. For clusters with no variation in the treatment variable, replace $\hat{\tau}_{k,m}$ with $\hat{\tau}_k$, for clusters with no variation in the treatment variable for a particular subsample, replace $\hat{\tau}^*_{k,m}$ with $\hat{\tau}^*_k$.
    \end{itemize}
    for more precise $\hat{V}^{\text{CCV}}_k(1)$, do the sample splitting multiple times and take the average of all variance estimators.
\end{itemize}

\paragraph*{Not all clusters are sampled: $q_k<1$}
First, notice that the variance for the general $q_k$ case is a convex combination of the true variance at $q_k=1$ and the cluster variance:
\begin{align*}
    v_k \left(q_k\right) - v^{\text{cluster}}_k &= q_k\times \left(v_k(1) - v_k^{\text{cluster}}\right) \\
    \Rightarrow v_k \left(q_k\right) &= q_k \times v_k(1) + \left(1-q_k\right)\times v^{\text{cluster}}_k
\end{align*}
then naturally, the variance estimator is 
\begin{equation*}
    \hat{V}^{\text{CCV}}_k = \hat{q}_k \times \hat{V}^{\text{CCV}}_k(1) + \left(1-\hat{q}_k\right) \times \hat{V}^{\text{cluster}}_k
\end{equation*}
where computing $\hat{q}_k$ requires knowledge of $m_k$, the total number of clusters.

\paragraph*{Bootstrap estimator}
\citet{abadie2023should} proposed a bootstrap procedure where units (clusters) have different assignments and assignment probabilities from the original sample: 
\begin{itemize}
    \item \myhl[myblue]{\textbf{Input}}:
    \begin{itemize}
        \item Sample $\left(Y_{k,i},W_{k,i},m_{k,i}\right)$
        \item Fraction sampled clusters $q_k$
        \item Number of bootstrap replications $B$
    \end{itemize}
    \item \myhl[myblue]{\textbf{Stage 1}}:
    \begin{itemize}
        \item[\textbf{a}] create pseudo population by replicating each cluster $1/q_k$ times 
        \item[\textbf{b}] for each cluster in the pseudo population, calculate the \textit{assignment probability} $\bar{W}_{k,m}$
        \item[\textbf{c}] create a bootstrap sample of clusters by \textit{randomly drawing clusters} from the pseudo population from \textbf{a}, where cluster $m$ is sampled with probability $q_k$
        \item[\textbf{d}] for each sampled cluster, draw an \textit{assignment probability} $A_{k,m}$ from the empirical distribution of the $\bar{W}_{k,m}$ from \textbf{b}
    \end{itemize}
    \item \myhl[myblue]{\textbf{Stage 2}}:
    \begin{itemize}
        \item[\textbf{a}] randomly draw from the set of treated units in cluster $m$, 
    \end{itemize}
\end{itemize}

\newpage
\bibliographystyle{plainnat}
\bibliography{ref.bib}

\end{document}