\documentclass[twoside]{article}
\setlength{\oddsidemargin}{0 in}
\setlength{\evensidemargin}{0 in}
\setlength{\topmargin}{-0.6 in}
\setlength{\textwidth}{6.5 in}
\setlength{\textheight}{8.5 in}
\setlength{\headsep}{0.75 in}
\setlength{\parindent}{0 in}
\setlength{\parskip}{0.1 in}

\usepackage{url}
\usepackage{titlesec}
\setcounter{secnumdepth}{3}
\usepackage{palatino}
\usepackage{marginnote}
\usepackage{multirow}
\usepackage{easybmat,bigdelim,arydshln}
\usepackage[authoryear,round]{natbib}
\usepackage{amssymb,amsmath,amsthm,amsfonts}
\usepackage{mathtools}
%\usepackage{nicematrix}
\usepackage{arydshln}
\usepackage{caption}
\usepackage{hyperref}
\usepackage{tcolorbox}
\tcbuselibrary{skins, breakable, theorems}
\usepackage{newpxtext,newpxmath}
\usepackage{longtable}
\usepackage{enumitem}
\makeatletter

\let\bar\overline

\setlist[itemize]{topsep=0pt,leftmargin=10pt,itemsep=-0.2em}
\usepackage{xcolor}
\usepackage{tikz}
\usepackage{pgfplots}
\pgfplotsset{compat = newest}
\usetikzlibrary{patterns,decorations.pathreplacing,decorations.markings,fit,shapes.geometric,angles,quotes,arrows}
\usepgfplotslibrary{fillbetween}

\usepackage{ifthen}
\usepackage{tikz-3dplot}

\pgfdeclarelayer{ft}
\pgfdeclarelayer{bg}
\pgfsetlayers{bg,main,ft}

\hypersetup{
    colorlinks,
    citecolor=red,
    filecolor=black,
    linkcolor=violet,
    urlcolor=blue
}

\definecolor{myblue}{cmyk}{1,.72,0,.38}
\definecolor{mypurple}{cmyk}{.57,1,0,.58}
\definecolor{myred}{cmyk}{0,.88,.88,.58}
\definecolor{mygreen}{cmyk}{1,0,.69,.66}
\definecolor{myorange}{cmyk}{0,.58,100,.20}
\definecolor{glaucous}{rgb}{0.38, 0.51, 0.71}

\makeatletter
\renewcommand{\thefigure}{\thesection.\arabic{figure}}
\newtheoremstyle{indented}
  {3pt}% space before
  {3pt}% space after
  {\addtolength{\@totalleftmargin}{3.5em}
   \addtolength{\linewidth}{-3.5em}
   \parshape 1 3.5em \linewidth}% body font
  {}% indent
  {\bfseries}% header font
  {.}% punctuation
  {.5em}% after theorem header
  {}% header specification (empty for default)
\makeatother

\newcommand{\ind}{\perp\!\!\!\perp}

\theoremstyle{definition}
\newtheorem{defin}{Definition}[section] % Creates a new counter, number within section
\newtheorem{prt}[defin]{Remark} 
\newtheorem{prts}[defin]{Remarks} % Again share defin's counter
\newtheorem{exmp}[defin]{Example} % etc.
\newtheorem{exmps}[defin]{Examples}
\newtheorem*{note}{Note}
\tcbuselibrary{theorems}

% use counter*=defin to make each tcbtheorem share defin's counter

\newtcbtheorem[use counter*=defin, number within=section]{definition}{Definition}{enhanced, breakable,
    colback = white, colframe = red!55!black, colbacktitle = red!55!black, attach boxed title to top left = {yshift = -2.5mm, xshift = 3mm}, boxed title style = {sharp corners},fonttitle=\bfseries}{def}

\newtcbtheorem[use counter*=defin, number within=section]{theorem}{Theorem}{enhanced, breakable,
    colback = white, colframe = blue!45!black, colbacktitle = blue!45!black, attach boxed title to top left = {yshift = -2.5mm, xshift = 3mm}, boxed title style = {sharp corners},fonttitle=\bfseries}{thm}
    
\newtcbtheorem[use counter*=defin, number within=section]{proposition}{Proposition}{enhanced, breakable,
    colback = white, colframe = teal, colbacktitle = teal, attach boxed title to top left = {yshift = -2.5mm, xshift = 3mm}, boxed title style = {sharp corners},fonttitle=\bfseries}{prop}

\newtcbtheorem[use counter*=defin, number within=section]{lemma}{Lemma}{enhanced, breakable,
    colback = white, colframe = orange!80!black, colbacktitle = orange!80!black, attach boxed title to top left = {yshift = -2.5mm, xshift = 3mm}, boxed title style = {sharp corners},fonttitle=\bfseries}{lemma}

\newtcbtheorem[use counter*=defin, number within=section]{example}{Example}{enhanced, breakable,
    colback = white, colframe = yellow!60!black, colbacktitle = yellow!60!black, attach boxed title to top left = {yshift = -2.5mm, xshift = 3mm}, boxed title style = {sharp corners},fonttitle=\bfseries}{exmp}

\newtcbtheorem[use counter*=defin, number within=section]{assumption}{Assumption}{enhanced, breakable,
    colback = white, colframe = violet!60!white, colbacktitle = violet!60!white, attach boxed title to top left = {yshift = -2.5mm, xshift = 3mm}, boxed title style = {sharp corners},fonttitle=\bfseries}{assump}

\newtcbtheorem[use counter*=defin, number within=section]{algorithm}{Algorithm}{enhanced, breakable,
    colback = white, colframe = green!55!black, colbacktitle = green!55!black, attach boxed title to top left = {yshift = -2.5mm, xshift = 3mm}, boxed title style = {sharp corners},fonttitle=\bfseries}{algm}
%\newtcolorbox{example}[1]{enhanced, breakable, colback = white, colframe = orange!85!black, colbacktitle = orange!85!black, attach boxed title to top left = {yshift = -2.5mm, xshift = 3mm}, boxed title style = {sharp corners},fonttitle=\bfseries, title={Example: #1}}

\newtcbox{\myhl}[1][white]
  {on line, arc = 0pt, outer arc = 0pt,
    colback = #1!20!white, colframe = #1!50!black,
    boxsep = 0pt, left = 1pt, right = 1pt, top = 1pt, bottom = 1pt, boxrule = 0pt, bottomrule =0pt, toprule =0pt}
    
\newtcbox{\myhlrule}[1][white]
  {on line, arc = 0pt, outer arc = 0pt,
    colback = #1!20!white, colframe = #1!50!black,
    boxsep = 0pt, left = 1pt, right = 1pt, top = 1pt, bottom = 1pt, boxrule = 0pt, bottomrule =0.5pt, toprule =0.5pt}
%
% The following commands set up the lecnum (lecture number)
% counter and make various numbering schemes work relative
% to the lecture number.
%
\newcounter{lecnum}
\renewcommand{\thepage}{\thelecnum-\arabic{page}}
\renewcommand{\thesection}{\thelecnum.\arabic{section}}
\renewcommand{\theequation}{\thelecnum.\arabic{equation}}
\renewcommand{\thefigure}{\thelecnum.\arabic{figure}}
\renewcommand{\thetable}{\thelecnum.\arabic{table}}

\newcommand{\sidenotes}[1]{\marginnote{\raggedright\scriptsize#1}}
%
% The following macro is used to generate the header.
%
\newcommand{\lecture}[6]{
   \pagestyle{myheadings}
   \thispagestyle{plain}
   \newpage
   \setcounter{lecnum}{#1}
   \setcounter{page}{1}
   \noindent
   \begin{center}
   \framebox{
      \vbox{\vspace{2mm}
    \hbox to 6.28in { {\bf Econometrics
	\hfill \today} }
       \vspace{4mm}
       \hbox to 6.28in { {\Large \hfill Topic #1: #2  \hfill} }
       \vspace{2mm}
       \hbox to 6.28in { {\it #3 \hfill by #4} }
      \vspace{2mm}}
   }
   \end{center}
   \markboth{Week #1: #2}{Week #1: #2}

   {\bf Key points}: {#5}

   {\bf Disclaimer}: {\it #6}
   \vspace*{4mm}
}
%

\tikzset{-stealth-/.style={decoration={
  markings,
  mark=at position #1 with {\arrow{stealth}}},postaction={decorate}}}

  \tikzset{tangent/.style={
    decoration={
        markings,% switch on markings
        mark=
            at position #1
            with
            {
                \coordinate (tangent point-\pgfkeysvalueof{/pgf/decoration/mark info/sequence number}) at (0pt,0pt);
                \coordinate (tangent unit vector-\pgfkeysvalueof{/pgf/decoration/mark info/sequence number}) at (1,0pt);
                \coordinate (tangent orthogonal unit vector-\pgfkeysvalueof{/pgf/decoration/mark info/sequence number}) at (0pt,1);
            }
    },
    postaction=decorate
},
use tangent/.style={
    shift=(tangent point-#1),
    x=(tangent unit vector-#1),
    y=(tangent orthogonal unit vector-#1)
},
use tangent/.default=1}

\tikzstyle{terminator} = [rectangle, draw, thick, text centered, rounded corners, minimum height=2em]
\tikzstyle{process} = [rectangle, draw, thick, text centered, minimum height=2em]
\tikzstyle{decision} = [diamond, draw, thick, text centered, minimum width=3cm, minimum height=0.5cm]
\tikzstyle{data}=[trapezium, draw, thick, text centered, trapezium left angle=60, trapezium right angle=120, minimum height=2em]
\tikzstyle{arrow} = [thick,->,>=stealth]

\begin{document}
\lecture{13}{Non-convex Learning + Lasso}{}{Sai Zhang}{Combining the best of the two, we can use \textbf{Lasso plus Concave} method, with Lasso screening and concave component selecting variables, achieving a coordinated intrinsic two-scale learning.}{The note is built on Prof. \hyperlink{http://faculty.marshall.usc.edu/jinchi-lv/}{Jinchi Lv}'s lectures of the course at USC, DSO 607, High-Dimensional Statistics and Big Data Problems.}

We are facing a tradeoff:
\begin{itemize}
    \item \myhl[myblue]{\textbf{Convex}} methods: have appealing \underline{prediction power and oracle inequalities}, but challenging to provide tight \underline{false sign rate control}
    \item \myhl[myred]{\textbf{Concave}} methods: have good \underline{variable selection} properties, but challenging to establish \underline{global} properties and risk properties
\end{itemize}

Here, we take advantage of the linearity of Lasso (convex \textit{and} concave) and try to combine it with concave regularization to get the best of both.

\section{Model Setup}
Again, consider a linear regression model $\mathbf{y}=\mathbf{X}\boldsymbol{\beta}+\boldsymbol{\epsilon}$, where 
\begin{itemize}
    \item response vector ($n\times 1$): $\mathbf{y} = (y_1,\cdots,y_n)^{\prime}$
    \item design matrix ($n\times p$): $\mathbf{X} = (\mathbf{x}_1,\cdots,\mathbf{x}_p)$, with each column rescaled to have $L_2-$norm $n^{1/2}$
\end{itemize}
here, we consider a scenario where
\begin{itemize}
    \item $\boldsymbol{\beta}_0 = (\beta_{0,1},\cdots,\beta_{0,p})^{\prime}$ is \textit{\textbf{sparse}} (with many 0 components)
    \item ultra-\textbf{high} dimensions: $\log p=O(n^a)$, for some $0<a<1$
\end{itemize}
and consider the penalized least squares 
\begin{equation}\label{eq:penalized-LS-l1concave}
    \min_{\boldsymbol{\beta}\in\mathbb{R}^p}\left\{ (2n)^{-1}\lVert \mathbf{y}-\mathbf{X}\boldsymbol{\beta} \rVert ^2_2 + \lambda_0 \lVert \boldsymbol{\beta} \rVert _1 + \lVert p_{\lambda}(\boldsymbol{\beta}) \rVert _1  \right\}
\end{equation}
where 
\begin{itemize}
    \item $\lambda_0 = c\left(\frac{\log p}{n}\right)^{1/2}$ for some $c>0$
    \item $p_{\lambda}(\boldsymbol{\beta}) = p_{\lambda}(\lvert\boldsymbol{\beta}\rvert ) = (p_{\lambda}(\lvert \beta_1 \rvert),\cdots,p_{\lambda}(\lvert \beta_p \rvert))^{\prime}$, with $\lvert\boldsymbol{\beta} \rvert = (\lvert \beta_1 \rvert,\cdots,\lvert \beta_p \rvert)^{\prime}$; the concave penalty $p_{\lambda}(t)$ is defined on $t\in\left[0,\infty \right)$, indexed by $\lambda\geq 0$, increasing in \textbf{\underline{both}} $t$ and $\lambda$, $p_{\lambda}(0)=0$
\end{itemize}
the 2 penalty components
\begin{itemize}
    \item $L_1-$component: minimum amount of regularization for \underline{\textit{removing noise}} in prediction 
    \item concave component $\lVert p_{\lambda}(\boldsymbol{\beta}) \rVert _1$: adapt model sparsity for \underline{\textit{variable selection}}
\end{itemize}

Under this set up, we can derive the hard-thresholding property as 
\begin{proposition}{Hard-Thresholding Property}{hard-thresholding-nonconvex}
    Assume the $p_{\lambda}(t)$, $t\geq 0$, is \myhl[myblue]{\textbf{increasing and concave}} with 
    \begin{itemize}
        \item $p_{\lambda}(t)\geq p_{H,\lambda}(t) = \frac{1}{2}\left[\lambda^2-(\lambda-t)^2_+\right]$ on $[0,\lambda]$
        \item $p'_{\lambda}\left((1-c_1)\lambda\right)\leq c_1\lambda$ for some $c_1\in [0,1)$
        \item $-p''_{\lambda}(t)$ decreasing on $[0,(1-c_1)\lambda]$
    \end{itemize}
    then any \underline{local minimizer} of \ref{eq:penalized-LS-l1concave} that is also a \underline{global minimizer} in each coordinate has the \textbf{{hard-thresholding}} feature that each component is either $0$ or of magnitude \textbf{larger} than $(1-c_1)\lambda$
\end{proposition}
Such property is shared by a wide class of concave penalties, including hard-thresholding penalty $p_{H,\lambda}(t)$ with $c_1=0$, $L_0-$penalty, and SICA (with suitable $c_1$).

\paragraph*{How to \textit{\underline{understand}} this proposition?} Let $\hat{\boldsymbol{\beta}}=\left(\hat{\beta}_1,\cdots,\hat{\beta}_p\right)^{\prime}$, then \myhl[myred]{\textbf{each} $\hat{\beta}_j$} is the glocal minimizer of the corresponding univariate penalized least-square problem along the $j-$th coordinate. These univariate problems share a common form with (generally) different scalars $z$ $$ \hat{\beta}(z) = \arg\min_{\beta\in\mathbb{R}}\left\{ \frac{1}{2}(z-\beta)^2 + \lambda_0\lvert\beta\rvert + p_{H,\lambda}(\lvert \beta \rvert) \right\} $$
after we rescale all covariates to have $L_2-$norm $n^{1/2}$. The solution to these univariate problems are $$ \hat{\beta}(z) = \mathrm{sgn}(z)(\lvert z \rvert-\lambda_0)\cdot\mathbf{1}_{\lvert z\rvert > \lambda+\lambda_0} $$
these solutions have the same feature as the hard-thresholded estimator: each component is either 0 or of magnitude larger than $\lambda$. This provides a better distinction between insignificant and significant covariates then soft-thresholding by $L_1$ penalty.


With the hard-thresholding property of Prop. \ref{prop:hard-thresholding-nonconvex}, we can prove a basic constraint for the global optimum $\hat{\boldsymbol{\beta}}$ on an event with significant probability \citep{fan2014asymptotic}
$$
\lVert \boldsymbol{\delta}_2 \rVert _1 \leq 7 \lVert \boldsymbol{\delta}_1 \rVert _1
$$
where $\boldsymbol{\delta} = \hat{\boldsymbol{\beta}}-\boldsymbol{\beta}_0 = \left( \boldsymbol{\delta}^{\prime}_1,\boldsymbol{\delta}^{\prime}_2 \right)^{\prime}$, with $\boldsymbol{\delta}_1\in\mathbb{R}^s$. Where does this constraint come from? For the penalized least square quesion \ref{eq:penalized-LS-l1concave}
$$
\min_{\boldsymbol{\beta}\in\mathbb{R}^p}\left\{ (2n)^{-1}\lVert \mathbf{y}-\mathbf{X}\boldsymbol{\beta} \rVert ^2_2 + \lambda_0 \lVert \boldsymbol{\beta} \rVert _1 + \lVert p_{\lambda}(\boldsymbol{\beta}) \rVert _1  \right\}
$$
the global minimizer $\hat{\boldsymbol{\beta}}$ leads to 
\begin{align*}
    (2n)^{-1}\lVert \mathbf{y}-\mathbf{X}\hat{\boldsymbol{\beta}} \rVert ^2_2 + \lambda_0 \lVert \hat{\boldsymbol{\beta}} \rVert _1 + \lVert p_{\lambda}(\hat{\boldsymbol{\beta}}) \rVert _1 =& (2n)^{-1}\lVert \mathbf{X}\boldsymbol{\beta}_0+ \boldsymbol{\epsilon} -\mathbf{X}\hat{\boldsymbol{\beta}} \rVert ^2_2 + \lambda_0 \lVert \hat{\boldsymbol{\beta}} \rVert _1 + \lVert p_{\lambda}(\hat{\boldsymbol{\beta}}) \rVert _1 \\
    =& (2n)^{-1}\lVert \mathbf{X}\boldsymbol{\beta}_0+ \boldsymbol{\epsilon} -\mathbf{X}\hat{\boldsymbol{\beta}} \rVert ^2_2 + \lambda_0 \lVert \hat{\boldsymbol{\beta}} \rVert _1 + \lVert p_{\lambda}(\hat{\boldsymbol{\beta}}) \rVert _1 
\end{align*}




\newpage
\bibliographystyle{plainnat}
\bibliography{ref.bib}

\end{document}