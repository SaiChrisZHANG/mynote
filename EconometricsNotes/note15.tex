\documentclass[twoside]{article}
\setlength{\oddsidemargin}{0 in}
\setlength{\evensidemargin}{0 in}
\setlength{\topmargin}{-0.6 in}
\setlength{\textwidth}{6.5 in}
\setlength{\textheight}{8.5 in}
\setlength{\headsep}{0.75 in}
\setlength{\parindent}{0 in}
\setlength{\parskip}{0.1 in}

\usepackage{url}
\usepackage{titlesec}
\setcounter{secnumdepth}{3}
\usepackage{palatino}
\usepackage{marginnote}
\usepackage{multirow}
\usepackage{easybmat,bigdelim,arydshln}
\usepackage[authoryear,round]{natbib}
\usepackage{amssymb,amsmath,amsthm,amsfonts}
\usepackage{mathtools}
%\usepackage{nicematrix}
\usepackage{arydshln}
\usepackage{caption}
\usepackage{hyperref}
\usepackage{tcolorbox}
\tcbuselibrary{skins, breakable, theorems}
\usepackage{newpxtext,newpxmath}
\usepackage{longtable}
\usepackage{enumitem}
\makeatletter

\let\bar\overline

\setlist[itemize]{topsep=0pt,leftmargin=10pt,itemsep=-0.2em}
\usepackage{xcolor}
\usepackage{tikz}
\usepackage{pgfplots}
\pgfplotsset{compat = newest}
\usetikzlibrary{patterns,decorations.pathreplacing,decorations.markings,fit,shapes.geometric,angles,quotes,arrows}
\usepgfplotslibrary{fillbetween}

\usepackage{ifthen}
\usepackage{tikz-3dplot}

\pgfdeclarelayer{ft}
\pgfdeclarelayer{bg}
\pgfsetlayers{bg,main,ft}

\hypersetup{
    colorlinks,
    citecolor=red,
    filecolor=black,
    linkcolor=violet,
    urlcolor=blue
}

\definecolor{myblue}{cmyk}{1,.72,0,.38}
\definecolor{mypurple}{cmyk}{.57,1,0,.58}
\definecolor{myred}{cmyk}{0,.88,.88,.58}
\definecolor{mygreen}{cmyk}{1,0,.69,.66}
\definecolor{myorange}{cmyk}{0,.58,100,.20}
\definecolor{glaucous}{rgb}{0.38, 0.51, 0.71}

\makeatletter
\renewcommand{\thefigure}{\thesection.\arabic{figure}}
\newtheoremstyle{indented}
  {3pt}% space before
  {3pt}% space after
  {\addtolength{\@totalleftmargin}{3.5em}
   \addtolength{\linewidth}{-3.5em}
   \parshape 1 3.5em \linewidth}% body font
  {}% indent
  {\bfseries}% header font
  {.}% punctuation
  {.5em}% after theorem header
  {}% header specification (empty for default)
\makeatother

\newcommand{\ind}{\perp\!\!\!\perp}

\theoremstyle{definition}
\newtheorem{defin}{Definition}[section] % Creates a new counter, number within section
\newtheorem{prt}[defin]{Remark} 
\newtheorem{prts}[defin]{Remarks} % Again share defin's counter
\newtheorem{exmp}[defin]{Example} % etc.
\newtheorem{exmps}[defin]{Examples}
\newtheorem*{note}{Note}
\tcbuselibrary{theorems}

% use counter*=defin to make each tcbtheorem share defin's counter

\newtcbtheorem[use counter*=defin, number within=section]{definition}{Definition}{enhanced, breakable,
    colback = white, colframe = red!55!black, colbacktitle = red!55!black, attach boxed title to top left = {yshift = -2.5mm, xshift = 3mm}, boxed title style = {sharp corners},fonttitle=\bfseries}{def}

\newtcbtheorem[use counter*=defin, number within=section]{theorem}{Theorem}{enhanced, breakable,
    colback = white, colframe = blue!45!black, colbacktitle = blue!45!black, attach boxed title to top left = {yshift = -2.5mm, xshift = 3mm}, boxed title style = {sharp corners},fonttitle=\bfseries}{thm}
    
\newtcbtheorem[use counter*=defin, number within=section]{proposition}{Proposition}{enhanced, breakable,
    colback = white, colframe = teal, colbacktitle = teal, attach boxed title to top left = {yshift = -2.5mm, xshift = 3mm}, boxed title style = {sharp corners},fonttitle=\bfseries}{prop}

\newtcbtheorem[use counter*=defin, number within=section]{lemma}{Lemma}{enhanced, breakable,
    colback = white, colframe = orange!80!black, colbacktitle = orange!80!black, attach boxed title to top left = {yshift = -2.5mm, xshift = 3mm}, boxed title style = {sharp corners},fonttitle=\bfseries}{lemma}

\newtcbtheorem[use counter*=defin, number within=section]{example}{Example}{enhanced, breakable,
    colback = white, colframe = yellow!60!black, colbacktitle = yellow!60!black, attach boxed title to top left = {yshift = -2.5mm, xshift = 3mm}, boxed title style = {sharp corners},fonttitle=\bfseries}{exmp}

\newtcbtheorem[use counter*=defin, number within=section]{assumption}{Assumption}{enhanced, breakable,
    colback = white, colframe = violet!60!white, colbacktitle = violet!60!white, attach boxed title to top left = {yshift = -2.5mm, xshift = 3mm}, boxed title style = {sharp corners},fonttitle=\bfseries}{assump}

\newtcbtheorem[use counter*=defin, number within=section]{algorithm}{Algorithm}{enhanced, breakable,
    colback = white, colframe = green!55!black, colbacktitle = green!55!black, attach boxed title to top left = {yshift = -2.5mm, xshift = 3mm}, boxed title style = {sharp corners},fonttitle=\bfseries}{algm}
%\newtcolorbox{example}[1]{enhanced, breakable, colback = white, colframe = orange!85!black, colbacktitle = orange!85!black, attach boxed title to top left = {yshift = -2.5mm, xshift = 3mm}, boxed title style = {sharp corners},fonttitle=\bfseries, title={Example: #1}}

\newtcbox{\myhl}[1][white]
  {on line, arc = 0pt, outer arc = 0pt,
    colback = #1!20!white, colframe = #1!50!black,
    boxsep = 0pt, left = 1pt, right = 1pt, top = 1pt, bottom = 1pt, boxrule = 0pt, bottomrule =0pt, toprule =0pt}
    
\newtcbox{\myhlrule}[1][white]
  {on line, arc = 0pt, outer arc = 0pt,
    colback = #1!20!white, colframe = #1!50!black,
    boxsep = 0pt, left = 1pt, right = 1pt, top = 1pt, bottom = 1pt, boxrule = 0pt, bottomrule =0.5pt, toprule =0.5pt}
%
% The following commands set up the lecnum (lecture number)
% counter and make various numbering schemes work relative
% to the lecture number.
%
\newcounter{lecnum}
\renewcommand{\thepage}{\thelecnum-\arabic{page}}
\renewcommand{\thesection}{\thelecnum.\arabic{section}}
\renewcommand{\theequation}{\thelecnum.\arabic{equation}}
\renewcommand{\thefigure}{\thelecnum.\arabic{figure}}
\renewcommand{\thetable}{\thelecnum.\arabic{table}}

\newcommand{\sidenotes}[1]{\marginnote{\raggedright\scriptsize#1}}
%
% The following macro is used to generate the header.
%
\newcommand{\lecture}[6]{
   \pagestyle{myheadings}
   \thispagestyle{plain}
   \newpage
   \setcounter{lecnum}{#1}
   \setcounter{page}{1}
   \noindent
   \begin{center}
   \framebox{
      \vbox{\vspace{2mm}
    \hbox to 6.28in { {\bf Econometrics
	\hfill \today} }
       \vspace{4mm}
       \hbox to 6.28in { {\Large \hfill Topic #1: #2  \hfill} }
       \vspace{2mm}
       \hbox to 6.28in { {\it #3 \hfill by #4} }
      \vspace{2mm}}
   }
   \end{center}
   \markboth{Week #1: #2}{Week #1: #2}

   {\bf Key points}: {#5}

   {\bf Disclaimer}: {\it #6}
   \vspace*{4mm}
}
%

\tikzset{-stealth-/.style={decoration={
  markings,
  mark=at position #1 with {\arrow{stealth}}},postaction={decorate}}}

  \tikzset{tangent/.style={
    decoration={
        markings,% switch on markings
        mark=
            at position #1
            with
            {
                \coordinate (tangent point-\pgfkeysvalueof{/pgf/decoration/mark info/sequence number}) at (0pt,0pt);
                \coordinate (tangent unit vector-\pgfkeysvalueof{/pgf/decoration/mark info/sequence number}) at (1,0pt);
                \coordinate (tangent orthogonal unit vector-\pgfkeysvalueof{/pgf/decoration/mark info/sequence number}) at (0pt,1);
            }
    },
    postaction=decorate
},
use tangent/.style={
    shift=(tangent point-#1),
    x=(tangent unit vector-#1),
    y=(tangent orthogonal unit vector-#1)
},
use tangent/.default=1}

\tikzstyle{terminator} = [rectangle, draw, thick, text centered, rounded corners, minimum height=2em]
\tikzstyle{process} = [rectangle, draw, thick, text centered, minimum height=2em]
\tikzstyle{decision} = [diamond, draw, thick, text centered, minimum width=3cm, minimum height=0.5cm]
\tikzstyle{data}=[trapezium, draw, thick, text centered, trapezium left angle=60, trapezium right angle=120, minimum height=2em]
\tikzstyle{arrow} = [thick,->,>=stealth]

\begin{document}
\lecture{15}{Sparse Orthogonal Factor Regression}{}{Sai Zhang}{Sparcity and dimensionality reduction for Multivariate Linear Regression models.}{The note is built on Prof. \hyperlink{http://faculty.marshall.usc.edu/jinchi-lv/}{Jinchi Lv}'s lectures of the course at USC, DSO 607, High-Dimensional Statistics and Big Data Problems.}
%\footnotetext{These notes are partially based on those of Nigel Mansell.}

\section{Motivation}
Consider a Mutlivariate Linear Regression (MLR) model
\begin{align*}
    \underset{n\times q}{\mathbf{Y}} = \underset{n\times p}{\mathbf{X}} \cdot \underset{p\times q}{\mathbf{C}}+ \underset{n\times q}{\mathbf{E}}
\end{align*}
How to apply regularization methods to this model? There are several approaches to consider 
\begin{itemize}
    \item \myhl[myblue]{\textbf{Shrinkage}}: ridge regression to overcome multicollinearity
    \item \myhl[myblue]{\textbf{sparsity}}: variable selection in multivariate setting 
    \item \myhl[myblue]{\textbf{Reduced-rank}}
    \begin{itemize}
        \item[-] \textbf{\underline{Dimension reduction}} via reducing rank of $\mathbf{C}$
        \item[-] $\min \lVert \mathbf{Y}-\mathbf{XC} \rVert^2_F$ s.t. $\mathrm{rank}(\mathbf{C})\leq r$  
    \end{itemize}
    \item \myhl[myblue]{\textbf{Combinations}}
    \item \myhl[myblue]{\textbf{Low-rank}} plus \myhl[myblue]{\textbf{sparse decomposition}}: robust PCA, latent variable graphical models, covariance estimation 
    \item \myhl[myblue]{\textbf{Regularized matrix}} or \myhl[myblue]{\textbf{tensor regression}}
\end{itemize}
Or, we can introduce a very attractive sparsity structure to achieve simultaneous dimension reduction and variable selection. This structure should be characterized by
\begin{itemize}
    \item Having a few \textbf{distinct} channels/pathways relating responses and predictors
    \item Each of such associations may involve only \textbf{a smaller subset}, but not all of the responses and predictors  
\end{itemize}
that is 
\begin{align*}
    \mathbf{Y} &= \mathbf{X}\mathbf{C} +\mathbf{E}\\
    &= \mathbf{X}\cdot \begin{pmatrix}
        c_{11} & c_{12} & \cdots & c_{1q} \\
        c_{21} & c_{22} & \cdots & c_{2q} \\
        \vdots & \vdots & \ddots & \vdots \\
        c_{p1} & c_{p2} & \cdots & c_{pq} 
    \end{pmatrix} + \mathbf{E} \\
    &= \mathbf{X}\cdot \begin{pmatrix}
        \textcolor{myblue}{0} & \textcolor{myred}{u_{12}} & \cdots & \textcolor{myorange}{u_{1r}} \\
        \textcolor{myblue}{u_{21}} & \textcolor{myred}{0} & \cdots & \textcolor{myorange}{c_{2r}} \\
        \textcolor{myblue}{\vdots} & \textcolor{myred}{\vdots} & \ddots & \textcolor{myorange}{\vdots} \\
        \textcolor{myblue}{u_{p1}} & \textcolor{myred}{u_{p2}} & \cdots & \textcolor{myorange}{u_{pr}} 
    \end{pmatrix}\cdot \begin{pmatrix}
        \textcolor{myblue}{d_1} & & & \\
         & \textcolor{myred}{d_2} & & \\
         & & \ddots & \\
         & & & \textcolor{myorange}{0}
    \end{pmatrix} \cdot \begin{pmatrix}
        \textcolor{myblue}{0} & \textcolor{myblue}{0} & \cdots & \textcolor{myblue}{v_{q1}} \\
        \textcolor{myred}{v_{12}} & \textcolor{myred}{v_{22}} & \cdots & \textcolor{myred}{0} \\
        \vdots & \vdots & \ddots & \vdots \\
        \textcolor{myorange}{v_{1r}} & \textcolor{myorange}{v_{2r}} & \cdots & \textcolor{myorange}{v_{qr}} 
    \end{pmatrix}
      + \mathbf{E}
\end{align*}
This way, we can have 
\begin{itemize}
    \item \myhl[myblue]{\textbf{Sparsity}}: selection of both \textbf{\underline{latent}} and \textbf{\underline{original}} variables
    \item \myhl[myblue]{\textbf{Low-rank SVD}}: different subsets of responses allowed to be associated with different subsets of predictors
\end{itemize}
Consider an example:
\begin{example}{Dimension Reduction and Variable Selection via Sparse SVD}{sparse_svd}
    Consider the case where $p=1000,q=100$, then $C$, as a $p\times q$ matrix, contains 100000 coefficients. Meanwhile, for a rank-3 SVD model:
    $$
     \mathbf{C} = d_1\mathbf{u}_1\mathbf{v}_1' + d_2\mathbf{u}_2\mathbf{v}_2' + d_3\mathbf{u}_3\mathbf{v}_3'
    $$
    where $\mathbf{u}_1,\mathbf{u}_2,\mathbf{u}_3$ are all $p\times 1$, $\mathbf{v}_1,\mathbf{v}_2,\mathbf{v}_3$ are all $q\times 1$, $d_1,d_2,d_3$ are all scalars. Hence, there are only $3\times (1000+100+1) = 3303$ paramaters to estimate. If futher assume sparcity, the dimension would be even lower.
\end{example}
Now let's develop a scalable procedure for this idea.

\section{Sparse Orthogonal Factor Regression}
Consider the sigular value decomposition of $\mathbf{C}$
$$
\mathbf{C} = \mathbf{UDV}'=\sum^r_{k=1}d_k\mathbf{u}_k\mathbf{v}_k'
$$
where $\mathbf{U}$ and $\mathbf{V}$ are both \myhl[myblue]{\textbf{orthonormal}}: $\mathbf{UU}' = \mathbf{VV}'=\mathbf{I}$. Then we can achieve dimension reduction via \textbf{low-dimensional latent model} $$ \tilde{\mathbf{Y}} = \tilde{\mathbf{X}}\mathbf{D}+\tilde{\mathbf{E}} $$ where 
\begin{itemize}
    \item $\tilde{\mathbf{Y}} = \mathbf{YV}$: $\mathbf{V}$ sparsity leads to \textbf{\underline{response}} variable selection
    \item $\tilde{\mathbf{X}} = \mathbf{XU}$: $\mathbf{U}$ sparsity leads to \textbf{\underline{predictor}} variable selection
\end{itemize}

How consider 
\begin{align}
    \left( \hat{\mathbf{D}},\hat{\mathbf{U}},\hat{\mathbf{V}} \right) &= \arg\min_{\mathbf{D,U,V}}\left\{ \frac{1}{2}\left\Vert \mathbf{Y-XUDV}' \right\Vert _F^2 + \lambda_d \lVert \mathbf{D} \rVert _1 + \lambda_a \rho_a (\mathbf{UD}) + \lambda_b \rho_b (\mathbf{VD})\right\} & \text{s.t. } \mathbf{U'U}=\mathbf{V'V}=\mathbf{I}_m 
\end{align}
where 
\begin{itemize}
    \item $\rho_a(\cdot),\rho_b(\cdot)$ are penalty functions with regularization parameters $\lambda_d,\lambda_a,\lambda_b\geq 0$. These sparsity penalizations on $\mathbf{UD}$ and $\mathbf{VD}$ can be thought as \textbf{importance weighting}
    \item $\lVert \cdot \rVert _F$ is the nuclear norm, defined as the \textbf{sum} of its singular values $\lVert \mathbf{A}\rVert _F =\sum_i\sigma_i(\mathbf{A})$. It encourages sparsity among singular values and achieve \textbf{\underline{rank reduction}}
    \item The orthgonality on $\mathbf{U,V}$ allow a flexible form of sparsity-inducing penalties
\end{itemize}



Two applications are
\begin{itemize}
    \item Biclustering with sparse SVD 
    \begin{align}
        \left( \hat{\mathbf{D}},\hat{\mathbf{U}},\hat{\mathbf{V}} \right) &= \arg\min_{\mathbf{D,U,V}}\left\{ \frac{1}{2}\left\Vert \mathbf{X-UDV}' \right\Vert _F^2 + \lambda_d \lVert \mathbf{D} \rVert _1 + \lambda_a \rho_a (\mathbf{UD}) + \lambda_b \rho_b (\mathbf{VD})\right\} & \text{s.t. } \mathbf{U'U}=\mathbf{V'V}=\mathbf{I}_m 
    \end{align}
    \item Sparse PCA (sparsity in loadings of principla components)
    \begin{align}
        \left( \hat{\mathbf{A}},\hat{\mathbf{V}} \right) &= \arg\min_{\mathbf{A,V}}\left\{ \frac{1}{2}\left\Vert \mathbf{X-XAV}' \right\Vert _F^2  + \lambda_a \rho_a (\mathbf{A}) \right\} & \text{s.t. } \mathbf{V'V}=\mathbf{I}_m 
    \end{align}
\end{itemize}

%\newpage
%\bibliographystyle{plainnat}
%\bibliography{ref.bib}

\end{document}